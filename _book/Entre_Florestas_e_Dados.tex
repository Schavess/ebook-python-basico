% Options for packages loaded elsewhere
\PassOptionsToPackage{unicode}{hyperref}
\PassOptionsToPackage{hyphens}{url}
%
\documentclass[
]{book}
\usepackage{amsmath,amssymb}
\usepackage{lmodern}
\usepackage{iftex}
\ifPDFTeX
  \usepackage[T1]{fontenc}
  \usepackage[utf8]{inputenc}
  \usepackage{textcomp} % provide euro and other symbols
\else % if luatex or xetex
  \usepackage{unicode-math}
  \defaultfontfeatures{Scale=MatchLowercase}
  \defaultfontfeatures[\rmfamily]{Ligatures=TeX,Scale=1}
\fi
% Use upquote if available, for straight quotes in verbatim environments
\IfFileExists{upquote.sty}{\usepackage{upquote}}{}
\IfFileExists{microtype.sty}{% use microtype if available
  \usepackage[]{microtype}
  \UseMicrotypeSet[protrusion]{basicmath} % disable protrusion for tt fonts
}{}
\makeatletter
\@ifundefined{KOMAClassName}{% if non-KOMA class
  \IfFileExists{parskip.sty}{%
    \usepackage{parskip}
  }{% else
    \setlength{\parindent}{0pt}
    \setlength{\parskip}{6pt plus 2pt minus 1pt}}
}{% if KOMA class
  \KOMAoptions{parskip=half}}
\makeatother
\usepackage{xcolor}
\usepackage{color}
\usepackage{fancyvrb}
\newcommand{\VerbBar}{|}
\newcommand{\VERB}{\Verb[commandchars=\\\{\}]}
\DefineVerbatimEnvironment{Highlighting}{Verbatim}{commandchars=\\\{\}}
% Add ',fontsize=\small' for more characters per line
\usepackage{framed}
\definecolor{shadecolor}{RGB}{248,248,248}
\newenvironment{Shaded}{\begin{snugshade}}{\end{snugshade}}
\newcommand{\AlertTok}[1]{\textcolor[rgb]{0.94,0.16,0.16}{#1}}
\newcommand{\AnnotationTok}[1]{\textcolor[rgb]{0.56,0.35,0.01}{\textbf{\textit{#1}}}}
\newcommand{\AttributeTok}[1]{\textcolor[rgb]{0.77,0.63,0.00}{#1}}
\newcommand{\BaseNTok}[1]{\textcolor[rgb]{0.00,0.00,0.81}{#1}}
\newcommand{\BuiltInTok}[1]{#1}
\newcommand{\CharTok}[1]{\textcolor[rgb]{0.31,0.60,0.02}{#1}}
\newcommand{\CommentTok}[1]{\textcolor[rgb]{0.56,0.35,0.01}{\textit{#1}}}
\newcommand{\CommentVarTok}[1]{\textcolor[rgb]{0.56,0.35,0.01}{\textbf{\textit{#1}}}}
\newcommand{\ConstantTok}[1]{\textcolor[rgb]{0.00,0.00,0.00}{#1}}
\newcommand{\ControlFlowTok}[1]{\textcolor[rgb]{0.13,0.29,0.53}{\textbf{#1}}}
\newcommand{\DataTypeTok}[1]{\textcolor[rgb]{0.13,0.29,0.53}{#1}}
\newcommand{\DecValTok}[1]{\textcolor[rgb]{0.00,0.00,0.81}{#1}}
\newcommand{\DocumentationTok}[1]{\textcolor[rgb]{0.56,0.35,0.01}{\textbf{\textit{#1}}}}
\newcommand{\ErrorTok}[1]{\textcolor[rgb]{0.64,0.00,0.00}{\textbf{#1}}}
\newcommand{\ExtensionTok}[1]{#1}
\newcommand{\FloatTok}[1]{\textcolor[rgb]{0.00,0.00,0.81}{#1}}
\newcommand{\FunctionTok}[1]{\textcolor[rgb]{0.00,0.00,0.00}{#1}}
\newcommand{\ImportTok}[1]{#1}
\newcommand{\InformationTok}[1]{\textcolor[rgb]{0.56,0.35,0.01}{\textbf{\textit{#1}}}}
\newcommand{\KeywordTok}[1]{\textcolor[rgb]{0.13,0.29,0.53}{\textbf{#1}}}
\newcommand{\NormalTok}[1]{#1}
\newcommand{\OperatorTok}[1]{\textcolor[rgb]{0.81,0.36,0.00}{\textbf{#1}}}
\newcommand{\OtherTok}[1]{\textcolor[rgb]{0.56,0.35,0.01}{#1}}
\newcommand{\PreprocessorTok}[1]{\textcolor[rgb]{0.56,0.35,0.01}{\textit{#1}}}
\newcommand{\RegionMarkerTok}[1]{#1}
\newcommand{\SpecialCharTok}[1]{\textcolor[rgb]{0.00,0.00,0.00}{#1}}
\newcommand{\SpecialStringTok}[1]{\textcolor[rgb]{0.31,0.60,0.02}{#1}}
\newcommand{\StringTok}[1]{\textcolor[rgb]{0.31,0.60,0.02}{#1}}
\newcommand{\VariableTok}[1]{\textcolor[rgb]{0.00,0.00,0.00}{#1}}
\newcommand{\VerbatimStringTok}[1]{\textcolor[rgb]{0.31,0.60,0.02}{#1}}
\newcommand{\WarningTok}[1]{\textcolor[rgb]{0.56,0.35,0.01}{\textbf{\textit{#1}}}}
\usepackage{longtable,booktabs,array}
\usepackage{calc} % for calculating minipage widths
% Correct order of tables after \paragraph or \subparagraph
\usepackage{etoolbox}
\makeatletter
\patchcmd\longtable{\par}{\if@noskipsec\mbox{}\fi\par}{}{}
\makeatother
% Allow footnotes in longtable head/foot
\IfFileExists{footnotehyper.sty}{\usepackage{footnotehyper}}{\usepackage{footnote}}
\makesavenoteenv{longtable}
\usepackage{graphicx}
\makeatletter
\def\maxwidth{\ifdim\Gin@nat@width>\linewidth\linewidth\else\Gin@nat@width\fi}
\def\maxheight{\ifdim\Gin@nat@height>\textheight\textheight\else\Gin@nat@height\fi}
\makeatother
% Scale images if necessary, so that they will not overflow the page
% margins by default, and it is still possible to overwrite the defaults
% using explicit options in \includegraphics[width, height, ...]{}
\setkeys{Gin}{width=\maxwidth,height=\maxheight,keepaspectratio}
% Set default figure placement to htbp
\makeatletter
\def\fps@figure{htbp}
\makeatother
\setlength{\emergencystretch}{3em} % prevent overfull lines
\providecommand{\tightlist}{%
  \setlength{\itemsep}{0pt}\setlength{\parskip}{0pt}}
\setcounter{secnumdepth}{5}
\ifLuaTeX
\usepackage[bidi=basic]{babel}
\else
\usepackage[bidi=default]{babel}
\fi
\babelprovide[main,import]{brazilian}
% get rid of language-specific shorthands (see #6817):
\let\LanguageShortHands\languageshorthands
\def\languageshorthands#1{}
\usepackage{fvextra}       % Extensão para controlar blocos de código
\usepackage{listings}

\lstset{
  breaklines=true,           % Quebra automática de linha
  breakatwhitespace=false,    % Quebra onde necessário, não apenas em espaços
  basicstyle=\ttfamily\small, % Fonte de código menor
  columns=flexible,           % Ajusta o espaçamento para evitar overflow
  keepspaces=true             % Preserva os espaços no código
}

% Configuração específica do fvextra
\DefineVerbatimEnvironment{Highlighting}{Verbatim}{breaklines,commandchars=\\\{\}}

\usepackage{etoolbox}
\patchcmd{\tableofcontents}{\thispagestyle{plain}}{\thispagestyle{empty}}{}{}
\ifLuaTeX
  \usepackage{selnolig}  % disable illegal ligatures
\fi
\usepackage[]{natbib}
\bibliographystyle{plainnat}
\IfFileExists{bookmark.sty}{\usepackage{bookmark}}{\usepackage{hyperref}}
\IfFileExists{xurl.sty}{\usepackage{xurl}}{} % add URL line breaks if available
\urlstyle{same} % disable monospaced font for URLs
\hypersetup{
  pdftitle={Entre Florestas e Dados},
  pdfauthor={Arthur Guilherme Schirmbeck Chaves},
  pdflang={pt-BR},
  hidelinks,
  pdfcreator={LaTeX via pandoc}}

\title{Entre Florestas e Dados}
\author{Arthur Guilherme Schirmbeck Chaves}
\date{Última atualização: 2024-11-14}

\begin{document}
\maketitle

{
\setcounter{tocdepth}{1}
\tableofcontents
}
\hypertarget{sobre}{%
\chapter*{Sobre}\label{sobre}}
\addcontentsline{toc}{chapter}{Sobre}

``Entre Florestas e Dados'' é um livro digital dinâmico (sempre atualizado online) que compartilha soluções práticas para
análises de dados florestais utilizando a linguagem R.

\hypertarget{por-que-este-livro}{%
\section*{Por que este livro?}\label{por-que-este-livro}}
\addcontentsline{toc}{section}{Por que este livro?}

Este livro foi criado para preencher uma lacuna na literatura sobre análise de dados aplicada especificamente ao setor
florestal, com o intuito de capacitar profissionais a tomar decisões fundamentadas em dados de inventários, planejamento
e monitoramento ambiental.

\hypertarget{autor}{%
\section*{Autor}\label{autor}}
\addcontentsline{toc}{section}{Autor}

Olá! Sou Arthur Guilherme Schirmbeck Chaves, professor de Engenharia Florestal no IFMT. Como engenheiro florestal e
analista de sistemas com sólida experiência em planejamento florestal, inventário, silvicultura e análise de dados,
estou desenvolvendo este livro para compartilhar soluções práticas e orientadas por dados que aplico em minhas
atividades profissionais e didáticas. Meu objetivo é capacitar profissionais e estudantes a compreender e implementar
análises de dados na área florestal de forma eficaz, precisa e diretamente aplicável às demandas do setor.

\hypertarget{dendrometria}{%
\chapter{Dendrometria}\label{dendrometria}}

\hypertarget{diuxe2metros}{%
\section{Diâmetros}\label{diuxe2metros}}

\hypertarget{histogramas-de-distribuiuxe7uxe3o-diamuxe9trica}{%
\subsection{Histogramas de distribuição diamétrica}\label{histogramas-de-distribuiuxe7uxe3o-diamuxe9trica}}

Este tipo de abordagem é útil para se verificar o grau de aproximação da Normalidade dos dados pela ``formato'' das classes através da frequência de indivíduos por classe diamétrica.

Os engenheiros florestais geralmente estabelecem classes diamétricas em intervalos fixos de 1,5 ou 2,0 cm para poderem comparar entre si as inúmeras parcelas; bem como para acompanhar a mudança de classe das árvores ao longo do tempo.

\hypertarget{criando-dados-para-demonstrauxe7uxe3o}{%
\subsubsection{Criando dados para demonstração}\label{criando-dados-para-demonstrauxe7uxe3o}}

\begin{Shaded}
\begin{Highlighting}[]
\CommentTok{\# Definir número de árvores por parcela}
\NormalTok{n\_arvores }\OtherTok{\textless{}{-}} \DecValTok{50}

\CommentTok{\# Gerar dados de 3 parcelas com diâmetros distribuídos normalmente}
\FunctionTok{set.seed}\NormalTok{(}\DecValTok{123}\NormalTok{)  }\CommentTok{\# Para garantir reprodutibilidade}

\NormalTok{parcela1 }\OtherTok{\textless{}{-}} \FunctionTok{rnorm}\NormalTok{(n\_arvores, }\AttributeTok{mean =} \DecValTok{25}\NormalTok{, }\AttributeTok{sd =} \DecValTok{5}\NormalTok{)  }\CommentTok{\# Parcela 1 com média 25 cm e desvio padrão 5 cm}
\NormalTok{parcela2 }\OtherTok{\textless{}{-}} \FunctionTok{rnorm}\NormalTok{(n\_arvores, }\AttributeTok{mean =} \DecValTok{30}\NormalTok{, }\AttributeTok{sd =} \DecValTok{7}\NormalTok{)  }\CommentTok{\# Parcela 2 com média 30 cm e desvio padrão 7 cm}
\NormalTok{parcela3 }\OtherTok{\textless{}{-}} \FunctionTok{rnorm}\NormalTok{(n\_arvores, }\AttributeTok{mean =} \DecValTok{35}\NormalTok{, }\AttributeTok{sd =} \DecValTok{6}\NormalTok{)  }\CommentTok{\# Parcela 3 com média 35 cm e desvio padrão 6 cm}

\CommentTok{\# Criar um data frame com os dados das parcelas}
\NormalTok{dados\_inventario }\OtherTok{\textless{}{-}} \FunctionTok{data.frame}\NormalTok{(}
  \AttributeTok{Parcela =} \FunctionTok{rep}\NormalTok{(}\FunctionTok{c}\NormalTok{(}\StringTok{"1"}\NormalTok{, }\StringTok{"2"}\NormalTok{, }\StringTok{"3"}\NormalTok{), }\AttributeTok{each =}\NormalTok{ n\_arvores),}
  \AttributeTok{Diametro =} \FunctionTok{c}\NormalTok{(parcela1, parcela2, parcela3)}
\NormalTok{)}
\end{Highlighting}
\end{Shaded}

\hypertarget{construindo-os-histogramas}{%
\subsubsection{Construindo os histogramas}\label{construindo-os-histogramas}}

\begin{Shaded}
\begin{Highlighting}[]
\CommentTok{\# Definir as classes diamétricas com intervalo de 2 cm}
\NormalTok{intervalo }\OtherTok{\textless{}{-}} \DecValTok{2}
\NormalTok{min\_diametro }\OtherTok{\textless{}{-}} \FunctionTok{floor}\NormalTok{(}\FunctionTok{min}\NormalTok{(dados\_inventario}\SpecialCharTok{$}\NormalTok{Diametro))  }\CommentTok{\# Valor mínimo de diâmetro arredondado para baixo}
\NormalTok{max\_diametro }\OtherTok{\textless{}{-}} \FunctionTok{ceiling}\NormalTok{(}\FunctionTok{max}\NormalTok{(dados\_inventario}\SpecialCharTok{$}\NormalTok{Diametro))  }\CommentTok{\# Valor máximo de diâmetro arredondado para cima}
\NormalTok{intervalos }\OtherTok{\textless{}{-}} \FunctionTok{seq}\NormalTok{(min\_diametro }\SpecialCharTok{{-}}\NormalTok{ intervalo, max\_diametro }\SpecialCharTok{+}\NormalTok{ intervalo, }\AttributeTok{by =}\NormalTok{ intervalo)  }\CommentTok{\# Definir as classes com intervalo de 2 cm}

\CommentTok{\# Criar layout para os gráficos}
\FunctionTok{par}\NormalTok{(}\AttributeTok{mfrow =} \FunctionTok{c}\NormalTok{(}\DecValTok{1}\NormalTok{, }\DecValTok{3}\NormalTok{))  }\CommentTok{\# Define 3 gráficos em uma linha}

\CommentTok{\# Definir lista de parcelas}
\NormalTok{parcelas }\OtherTok{\textless{}{-}} \FunctionTok{unique}\NormalTok{(dados\_inventario}\SpecialCharTok{$}\NormalTok{Parcela)}

\CommentTok{\# Loop para plotar histogramas de cada parcela}
\ControlFlowTok{for}\NormalTok{ (parcela }\ControlFlowTok{in}\NormalTok{ parcelas) \{}
  \CommentTok{\# Selecionar os dados da parcela atual}
\NormalTok{  dados\_parcela }\OtherTok{\textless{}{-}}\NormalTok{ dados\_inventario[dados\_inventario}\SpecialCharTok{$}\NormalTok{Parcela }\SpecialCharTok{==}\NormalTok{ parcela, }\StringTok{"Diametro"}\NormalTok{]}
  \CommentTok{\# Plotar o histograma}
  \FunctionTok{print}\NormalTok{(parcela)}
  \FunctionTok{hist}\NormalTok{(dados\_parcela,}
       \AttributeTok{breaks =}\NormalTok{ intervalos,}
       \AttributeTok{main =} \FunctionTok{paste}\NormalTok{(}\StringTok{"Histograma {-} parcela "}\NormalTok{, parcela),}
       \AttributeTok{xlab =} \StringTok{"Diâmetro (cm)"}\NormalTok{, }
       \AttributeTok{ylab =} \StringTok{"Frequência"}\NormalTok{,}
       \AttributeTok{ylim =} \FunctionTok{c}\NormalTok{(}\DecValTok{0}\NormalTok{, }\DecValTok{10}\NormalTok{),}
       \AttributeTok{col =} \StringTok{"skyblue"}\NormalTok{,}
       \AttributeTok{border =} \StringTok{"black"}\NormalTok{)}
\NormalTok{\}}
\end{Highlighting}
\end{Shaded}

\begin{verbatim}
## [1] "1"
\end{verbatim}

\begin{verbatim}
## [1] "2"
\end{verbatim}

\begin{verbatim}
## [1] "3"
\end{verbatim}

\includegraphics{Entre_Florestas_e_Dados_files/figure-latex/unnamed-chunk-2-1.pdf}

\hypertarget{alturas}{%
\section{Alturas}\label{alturas}}

\hypertarget{altura-dominante}{%
\subsection{Altura Dominante}\label{altura-dominante}}

A altura dominante, que representa a média das alturas das árvores mais altas e/ou mais grossas de um determinado número de árvores por hectare, é menos influenciada por variações de densidade e competições locais entre árvores menores, tornando-a uma medida confiável da qualidade do sítio. Com ela, é possível estimar o potencial de crescimento da floresta, modelar curvas de crescimento e projetar a produção futura, auxiliando no planejamento sustentável e na tomada de decisões estratégicas no manejo florestal.

Dados exemplo:
\href{data/DesafioCalcHdom.csv}{Baixar dados}

\begin{table}
\centering
\begin{tabular}{r|r|r|r|r|l}
\hline
A.Parc. & Parc & n & DAP & Ht & Hdom\\
\hline
720 & 711 & 1 & 19.74 & 14.65 & NA\\
\hline
720 & 711 & 2 & NA & NA & NA\\
\hline
720 & 711 & 3 & 19.19 & 14.44 & NA\\
\hline
720 & 711 & 4 & 17.83 & 13.89 & NA\\
\hline
720 & 711 & 5 & 18.46 & 14.15 & NA\\
\hline
720 & 711 & 6 & 20.05 & 14.78 & NA\\
\hline
\end{tabular}
\end{table}

\hypertarget{calculando-altura-dominante-por-parcela-assman}{%
\subsubsection{Calculando altura dominante por parcela (Assman)}\label{calculando-altura-dominante-por-parcela-assman}}

\begin{Shaded}
\begin{Highlighting}[]
\NormalTok{BD }\OtherTok{\textless{}{-}} \FunctionTok{read.csv2}\NormalTok{(}\StringTok{"data/DesafioCalcHdom.csv"}\NormalTok{)}
\FunctionTok{colnames}\NormalTok{(BD)[}\DecValTok{1}\NormalTok{] }\OtherTok{=} \FunctionTok{c}\NormalTok{(}\StringTok{"areaParc"}\NormalTok{)}
\NormalTok{BD }\OtherTok{\textless{}{-}}\NormalTok{ BD[, }\DecValTok{1}\SpecialCharTok{:}\DecValTok{5}\NormalTok{]}
\NormalTok{BD }\OtherTok{\textless{}{-}}\NormalTok{ BD[}\FunctionTok{order}\NormalTok{(BD}\SpecialCharTok{$}\NormalTok{Parc, BD}\SpecialCharTok{$}\NormalTok{Ht, }\AttributeTok{decreasing =} \ConstantTok{TRUE}\NormalTok{),]}
\FunctionTok{row.names}\NormalTok{(BD) }\OtherTok{\textless{}{-}} \ConstantTok{NULL}

\NormalTok{Parc }\OtherTok{\textless{}{-}} \FunctionTok{unique}\NormalTok{(BD}\SpecialCharTok{$}\NormalTok{Parc)}
\NormalTok{Hdom }\OtherTok{\textless{}{-}} \FunctionTok{unique}\NormalTok{(BD}\SpecialCharTok{$}\NormalTok{Parc)}

\NormalTok{j }\OtherTok{\textless{}{-}} \DecValTok{1}
\ControlFlowTok{for}\NormalTok{ (i }\ControlFlowTok{in}\NormalTok{ Parc) \{}
\NormalTok{  a }\OtherTok{\textless{}{-}} \FunctionTok{round}\NormalTok{(}\FunctionTok{mean}\NormalTok{(BD[}\FunctionTok{which}\NormalTok{(BD}\SpecialCharTok{$}\NormalTok{Parc}\SpecialCharTok{==}\NormalTok{i),}\DecValTok{1}\NormalTok{])}\SpecialCharTok{*}\DecValTok{100}\SpecialCharTok{/}\DecValTok{10000}\NormalTok{, }\DecValTok{0}\NormalTok{)}
\NormalTok{  H }\OtherTok{\textless{}{-}}\NormalTok{ BD[}\FunctionTok{which}\NormalTok{(BD}\SpecialCharTok{$}\NormalTok{Parc}\SpecialCharTok{==}\NormalTok{i),}\DecValTok{5}\NormalTok{]}
\NormalTok{  Hdom[j] }\OtherTok{\textless{}{-}} \FunctionTok{round}\NormalTok{(}\FunctionTok{mean}\NormalTok{(H[}\DecValTok{1}\SpecialCharTok{:}\NormalTok{a]), }\DecValTok{2}\NormalTok{)}
\NormalTok{  j }\OtherTok{\textless{}{-}}\NormalTok{ j}\SpecialCharTok{+}\DecValTok{1}
\NormalTok{\}}

\NormalTok{BD2 }\OtherTok{\textless{}{-}} \FunctionTok{as.data.frame}\NormalTok{(}\FunctionTok{cbind}\NormalTok{(Parc, Hdom))}
\NormalTok{BD }\OtherTok{\textless{}{-}} \FunctionTok{merge}\NormalTok{(BD, BD2)}

\FunctionTok{rm}\NormalTok{(a, H, Hdom, i, j, Parc)}
\end{Highlighting}
\end{Shaded}

Resultado:

\begin{table}
\centering
\begin{tabular}{r|r}
\hline
Parc & Hdom\\
\hline
711 & 14.58\\
\hline
623 & 14.02\\
\hline
622 & 13.54\\
\hline
621 & 14.62\\
\hline
534 & 15.00\\
\hline
533 & 14.62\\
\hline
532 & 14.76\\
\hline
531 & 14.95\\
\hline
453 & 14.51\\
\hline
452 & 14.53\\
\hline
451 & 14.52\\
\hline
344 & 15.06\\
\hline
343 & 14.48\\
\hline
342 & 14.78\\
\hline
341 & 14.96\\
\hline
262 & 13.28\\
\hline
261 & 14.33\\
\hline
173 & 14.10\\
\hline
172 & 14.22\\
\hline
171 & 14.80\\
\hline
\end{tabular}
\end{table}

  \bibliography{book.bib,packages.bib}

\end{document}
