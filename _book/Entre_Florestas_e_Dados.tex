% Options for packages loaded elsewhere
\PassOptionsToPackage{unicode}{hyperref}
\PassOptionsToPackage{hyphens}{url}
%
\documentclass[
]{book}
\usepackage{amsmath,amssymb}
\usepackage{iftex}
\ifPDFTeX
  \usepackage[T1]{fontenc}
  \usepackage[utf8]{inputenc}
  \usepackage{textcomp} % provide euro and other symbols
\else % if luatex or xetex
  \usepackage{unicode-math} % this also loads fontspec
  \defaultfontfeatures{Scale=MatchLowercase}
  \defaultfontfeatures[\rmfamily]{Ligatures=TeX,Scale=1}
\fi
\usepackage{lmodern}
\ifPDFTeX\else
  % xetex/luatex font selection
\fi
% Use upquote if available, for straight quotes in verbatim environments
\IfFileExists{upquote.sty}{\usepackage{upquote}}{}
\IfFileExists{microtype.sty}{% use microtype if available
  \usepackage[]{microtype}
  \UseMicrotypeSet[protrusion]{basicmath} % disable protrusion for tt fonts
}{}
\makeatletter
\@ifundefined{KOMAClassName}{% if non-KOMA class
  \IfFileExists{parskip.sty}{%
    \usepackage{parskip}
  }{% else
    \setlength{\parindent}{0pt}
    \setlength{\parskip}{6pt plus 2pt minus 1pt}}
}{% if KOMA class
  \KOMAoptions{parskip=half}}
\makeatother
\usepackage{xcolor}
\usepackage{color}
\usepackage{fancyvrb}
\newcommand{\VerbBar}{|}
\newcommand{\VERB}{\Verb[commandchars=\\\{\}]}
\DefineVerbatimEnvironment{Highlighting}{Verbatim}{commandchars=\\\{\}}
% Add ',fontsize=\small' for more characters per line
\usepackage{framed}
\definecolor{shadecolor}{RGB}{248,248,248}
\newenvironment{Shaded}{\begin{snugshade}}{\end{snugshade}}
\newcommand{\AlertTok}[1]{\textcolor[rgb]{0.94,0.16,0.16}{#1}}
\newcommand{\AnnotationTok}[1]{\textcolor[rgb]{0.56,0.35,0.01}{\textbf{\textit{#1}}}}
\newcommand{\AttributeTok}[1]{\textcolor[rgb]{0.13,0.29,0.53}{#1}}
\newcommand{\BaseNTok}[1]{\textcolor[rgb]{0.00,0.00,0.81}{#1}}
\newcommand{\BuiltInTok}[1]{#1}
\newcommand{\CharTok}[1]{\textcolor[rgb]{0.31,0.60,0.02}{#1}}
\newcommand{\CommentTok}[1]{\textcolor[rgb]{0.56,0.35,0.01}{\textit{#1}}}
\newcommand{\CommentVarTok}[1]{\textcolor[rgb]{0.56,0.35,0.01}{\textbf{\textit{#1}}}}
\newcommand{\ConstantTok}[1]{\textcolor[rgb]{0.56,0.35,0.01}{#1}}
\newcommand{\ControlFlowTok}[1]{\textcolor[rgb]{0.13,0.29,0.53}{\textbf{#1}}}
\newcommand{\DataTypeTok}[1]{\textcolor[rgb]{0.13,0.29,0.53}{#1}}
\newcommand{\DecValTok}[1]{\textcolor[rgb]{0.00,0.00,0.81}{#1}}
\newcommand{\DocumentationTok}[1]{\textcolor[rgb]{0.56,0.35,0.01}{\textbf{\textit{#1}}}}
\newcommand{\ErrorTok}[1]{\textcolor[rgb]{0.64,0.00,0.00}{\textbf{#1}}}
\newcommand{\ExtensionTok}[1]{#1}
\newcommand{\FloatTok}[1]{\textcolor[rgb]{0.00,0.00,0.81}{#1}}
\newcommand{\FunctionTok}[1]{\textcolor[rgb]{0.13,0.29,0.53}{\textbf{#1}}}
\newcommand{\ImportTok}[1]{#1}
\newcommand{\InformationTok}[1]{\textcolor[rgb]{0.56,0.35,0.01}{\textbf{\textit{#1}}}}
\newcommand{\KeywordTok}[1]{\textcolor[rgb]{0.13,0.29,0.53}{\textbf{#1}}}
\newcommand{\NormalTok}[1]{#1}
\newcommand{\OperatorTok}[1]{\textcolor[rgb]{0.81,0.36,0.00}{\textbf{#1}}}
\newcommand{\OtherTok}[1]{\textcolor[rgb]{0.56,0.35,0.01}{#1}}
\newcommand{\PreprocessorTok}[1]{\textcolor[rgb]{0.56,0.35,0.01}{\textit{#1}}}
\newcommand{\RegionMarkerTok}[1]{#1}
\newcommand{\SpecialCharTok}[1]{\textcolor[rgb]{0.81,0.36,0.00}{\textbf{#1}}}
\newcommand{\SpecialStringTok}[1]{\textcolor[rgb]{0.31,0.60,0.02}{#1}}
\newcommand{\StringTok}[1]{\textcolor[rgb]{0.31,0.60,0.02}{#1}}
\newcommand{\VariableTok}[1]{\textcolor[rgb]{0.00,0.00,0.00}{#1}}
\newcommand{\VerbatimStringTok}[1]{\textcolor[rgb]{0.31,0.60,0.02}{#1}}
\newcommand{\WarningTok}[1]{\textcolor[rgb]{0.56,0.35,0.01}{\textbf{\textit{#1}}}}
\usepackage{longtable,booktabs,array}
\usepackage{calc} % for calculating minipage widths
% Correct order of tables after \paragraph or \subparagraph
\usepackage{etoolbox}
\makeatletter
\patchcmd\longtable{\par}{\if@noskipsec\mbox{}\fi\par}{}{}
\makeatother
% Allow footnotes in longtable head/foot
\IfFileExists{footnotehyper.sty}{\usepackage{footnotehyper}}{\usepackage{footnote}}
\makesavenoteenv{longtable}
\usepackage{graphicx}
\makeatletter
\def\maxwidth{\ifdim\Gin@nat@width>\linewidth\linewidth\else\Gin@nat@width\fi}
\def\maxheight{\ifdim\Gin@nat@height>\textheight\textheight\else\Gin@nat@height\fi}
\makeatother
% Scale images if necessary, so that they will not overflow the page
% margins by default, and it is still possible to overwrite the defaults
% using explicit options in \includegraphics[width, height, ...]{}
\setkeys{Gin}{width=\maxwidth,height=\maxheight,keepaspectratio}
% Set default figure placement to htbp
\makeatletter
\def\fps@figure{htbp}
\makeatother
\setlength{\emergencystretch}{3em} % prevent overfull lines
\providecommand{\tightlist}{%
  \setlength{\itemsep}{0pt}\setlength{\parskip}{0pt}}
\setcounter{secnumdepth}{5}
\ifLuaTeX
\usepackage[bidi=basic]{babel}
\else
\usepackage[bidi=default]{babel}
\fi
\babelprovide[main,import]{brazilian}
% get rid of language-specific shorthands (see #6817):
\let\LanguageShortHands\languageshorthands
\def\languageshorthands#1{}
% Configuração básica de fontes e margens
\usepackage{geometry}
\usepackage[table]{xcolor}
\usepackage{pdflscape}


\geometry{a4paper, margin=1in}

% Extensão para controle de blocos de código
\usepackage{fvextra}
\usepackage{listings}

% Configuração dos blocos de código
\lstset{
  breaklines=true,           % Habilita quebra automática de linha
  breakatwhitespace=false,   % Quebra onde necessário, não apenas em espaços
  basicstyle=\ttfamily\small,% Fonte menor e monoespaçada
  columns=flexible,          % Ajusta espaçamento para evitar overflow
  keepspaces=true,           % Preserva os espaços
  frame=single,              % Adiciona borda ao redor do código
  xleftmargin=5pt,           % Margem esquerda
  xrightmargin=5pt           % Margem direita
}

% Configuração para ambientes Verbatim do fvextra
\DefineVerbatimEnvironment{Highlighting}{Verbatim}{breaklines,commandchars=\\\{\}}

% Ajustes no índice (sumário)
\usepackage{tocloft}
\setlength{\cftbeforetoctitleskip}{0pt} % Remove espaço antes do título do índice
\setlength{\cftaftertoctitleskip}{0pt}  % Remove espaço após o título do índice

% Remover número de página em páginas específicas
\usepackage{etoolbox}
\patchcmd{\tableofcontents}{\thispagestyle{plain}}{\thispagestyle{empty}}{}{}

% Remover espaços adicionais após capítulos
\makeatletter
\renewcommand{\cleardoublepage}{\clearpage}
\makeatother
\ifLuaTeX
  \usepackage{selnolig}  % disable illegal ligatures
\fi
\usepackage[]{natbib}
\bibliographystyle{plainnat}
\usepackage{bookmark}
\IfFileExists{xurl.sty}{\usepackage{xurl}}{} % add URL line breaks if available
\urlstyle{same}
\hypersetup{
  pdftitle={Entre Florestas e Dados},
  pdfauthor={Arthur Guilherme Schirmbeck Chaves},
  pdflang={pt-BR},
  hidelinks,
  pdfcreator={LaTeX via pandoc}}

\title{Entre Florestas e Dados}
\author{Arthur Guilherme Schirmbeck Chaves}
\date{Última atualização: 2024-11-19}

\begin{document}
\maketitle

{
\setcounter{tocdepth}{1}
\tableofcontents
}
\chapter*{Sobre}\label{sobre}
\addcontentsline{toc}{chapter}{Sobre}

``Entre Florestas e Dados'' é um livro digital dinâmico (sempre atualizado online) que compartilha soluções práticas para
análises de dados florestais utilizando a linguagem R.

\section*{Por que este livro?}\label{por-que-este-livro}
\addcontentsline{toc}{section}{Por que este livro?}

Este livro foi criado para preencher uma lacuna na literatura sobre análise de dados aplicada especificamente ao setor
florestal, com o intuito de capacitar profissionais a tomar decisões fundamentadas em dados de inventários, planejamento
e monitoramento ambiental.

\section*{Autor}\label{autor}
\addcontentsline{toc}{section}{Autor}

Olá! Meu nome é Arthur Guilherme Schirmbeck Chaves. Sou professor de Engenharia Florestal no IFMT. Como engenheiro florestal e
analista de sistemas com sólida experiência em planejamento florestal, inventário, silvicultura e análise de dados,
estou desenvolvendo este livro para compartilhar soluções práticas e orientadas por dados para os engenheiros florestais empregarem em suas
atividades profissionais e didáticas.

\chapter{Introdução ao R}\label{introduuxe7uxe3o-ao-r}

O R é provavelmente a mais importante ferramenta para a análise de dados florestais.

\begin{center}\rule{0.5\linewidth}{0.5pt}\end{center}

\section{Instalação do R}\label{instalauxe7uxe3o-do-r}

A instalação padrão do R é feita a partir do CRAN, uma rede de servidores espalhada pelo mundo que armazena versões idênticas e atualizadas de códigos e documentações para o R.

\subsection{Windows}\label{windows}

Para instalar o R no Windows, siga os seguintes passos:

\begin{enumerate}
\def\labelenumi{\arabic{enumi}.}
\item
  Acesse o CRAN: \url{https://www.r-project.org/}
\item
  No menu à esquerda, encontre a opção \textbf{Download} e clique em \textbf{CRAN}.
\item
  Escolha a opção de servidor (mirror) mais próxima de você.
\item
  Clique em \textbf{Download R for Windows}.
\item
  Clique na opção \textbf{base}.
\item
  Na nova página, clique em Download R x.x.x for Windows, sendo x.x.x o número da versão que será baixada. Se você teve algum problema com o download, tente escolher outro servidor no passo 3.
\item
  Feito o download, clique duas vezes no arquivo baixado e siga as instruções para instalação.
\end{enumerate}

\subsection{macOS}\label{macos}

Abra o terminal: \texttt{Command\ +\ Espaço} ``terminal'' e \texttt{Enter}

\begin{Shaded}
\begin{Highlighting}[]
\ExtensionTok{brew}\NormalTok{ install }\AttributeTok{{-}{-}cask}\NormalTok{ r}
\end{Highlighting}
\end{Shaded}

\subsection{Linux}\label{linux}

Abra o terminal Linux: \texttt{ctrl\ +\ alt\ +\ t}

\begin{Shaded}
\begin{Highlighting}[]
\FunctionTok{sudo}\NormalTok{ apt update}
\FunctionTok{sudo}\NormalTok{ apt upgrade}
\FunctionTok{sudo}\NormalTok{ apt install }\AttributeTok{{-}y}\NormalTok{ r{-}base}
\end{Highlighting}
\end{Shaded}

\begin{center}\rule{0.5\linewidth}{0.5pt}\end{center}

\section{Instalação do RStudio}\label{instalauxe7uxe3o-do-rstudio}

Para instalar o RStudio, siga os passos atualizados abaixo:

Acesse a página de downloads do RStudio:

Visite \url{https://posit.co/download/rstudio-desktop/\#download}.
Verifique os requisitos do sistema:

Certifique-se de que seu sistema operacional é compatível:
Windows: Windows 10 ou superior (64 bits).
macOS: macOS 11 (Big Sur) ou superior.
Linux: Distribuições compatíveis incluem Debian 10, Ubuntu 18.04 LTS, Ubuntu 20.04 LTS, Ubuntu 22.04 LTS, Debian 9, RHEL/CentOS 7, RHEL 8 e OpenSUSE/SLES 15.
Baixe o instalador adequado:

Na seção ``All Installers'', selecione o instalador correspondente ao seu sistema operacional.
Instale o RStudio:

\subsection{Windows}\label{windows-1}

Execute o arquivo .exe baixado e siga as instruções do assistente de instalação.

\subsection{macOS}\label{macos-1}

Abra o arquivo .dmg baixado, arraste o ícone do RStudio para a pasta ``Aplicativos'' e, em seguida, ejete a imagem de disco.

\subsection{Linux}\label{linux-1}

Baixe o pacote apropriado (.deb ou .rpm) e instale-o usando as ferramentas de gerenciamento de pacotes do seu sistema.
Inicie o RStudio:

Após a instalação, abra o RStudio para começar a utilizá-lo.
Lembre-se de que o RStudio requer o R instalado previamente. Certifique-se de ter o R instalado antes de prosseguir com a instalação do RStudio.

\begin{center}\rule{0.5\linewidth}{0.5pt}\end{center}

\section{Sobre o R}\label{sobre-o-r}

\subsection{O que são algoritmos?}\label{o-que-suxe3o-algoritmos}

\textbf{Objetivo do Algoritmo}: Plantar uma árvore.

\textbf{Entradas do Algoritmo}:

\begin{itemize}
\tightlist
\item
  Uma muda de árvore.
\item
  Local de plantio escolhido.
\item
  Ferramentas para cavar (como uma pá).
\item
  Água para regar a muda.
\end{itemize}

\textbf{Passos do Algoritmo}:

\begin{enumerate}
\def\labelenumi{\arabic{enumi}.}
\item
  Selecionar o Local: Escolher um local adequado para plantar a muda, considerando a exposição solar, o tipo de solo e o espaço necessário para o crescimento da árvore.
\item
  Cavar um Buraco: Usar a pá para cavar um buraco no local escolhido. O buraco deve ter profundidade e largura suficientes para acomodar as raízes da muda.
\item
  Preparar a Muda: Remover a muda do recipiente em que veio, com cuidado para não danificar as raízes.
\item
  Plantar a Muda: Colocar a muda no buraco, ajustando a profundidade para que a base do tronco fique no nível do solo. Preencher o buraco com terra, firmemente, mas sem compactar excessivamente.
\item
  Regar a Muda: Regar a muda generosamente para umedecer o solo e ajudar a estabelecer as raízes.
\item
  Cuidados Posteriores: Incluir instruções básicas de cuidados posteriores, como rega regular, aplicação de mulch para manter a umidade, e proteção contra ervas daninhas.
\end{enumerate}

\textbf{Saída do Algoritmo}:

Uma árvore plantada e pronta para crescer no local escolhido.

\subsection{O que é uma linguagem de programação?}\label{o-que-uxe9-uma-linguagem-de-programauxe7uxe3o}

Uma linguagem de programação é uma forma padronizada de comunicar instruções a um computador. É como uma língua que permite aos programadores escreverem códigos que são convertidos em ações executáveis pelo computador, permitindo a criação de software, aplicativos, websites, e muito mais. Cada linguagem de programação tem sua própria sintaxe e regras, assim como as línguas humanas têm gramática e vocabulário.

\subsection{R 📊}\label{r}

\textbf{R} é uma linguagem de programação voltada para análise estatística e visualização de dados.\\
Ela permite:\\
- Manipulação de grandes volumes de dados 📂\\
- Criação de gráficos avançados 📈\\
- Modelagem estatística complexa 🔢

Com uma vasta comunidade e milhares de pacotes disponíveis, \textbf{R} é ideal tanto para iniciantes quanto para especialistas em ciência de dados.

\textbf{Alto Nível:}\\
R é uma linguagem de programação de alto nível projetada para abstrair detalhes técnicos e permitir que os usuários se concentrem na análise de dados e estatísticas. Ela automatiza tarefas como alocação de memória e manipulação de estruturas complexas, permitindo maior foco nos problemas analíticos e menos em detalhes de implementação.

\textbf{Foco em Análise de Dados:}\\
Diferentemente de linguagens de propósito geral, R é especialmente projetada para análise estatística, ciência de dados e visualização. É amplamente usada em áreas como bioestatística, econometria, aprendizado de máquina e pesquisa acadêmica, fornecendo ferramentas especializadas para cada etapa do fluxo de trabalho analítico.

\textbf{Visualizações Avançadas:}\\
R é reconhecida por sua capacidade de criar gráficos e visualizações de dados de alta qualidade, tanto para análises exploratórias quanto para apresentações profissionais. Com pacotes como \texttt{ggplot2} e \texttt{plotly}, é possível produzir desde gráficos básicos até visualizações interativas complexas.

\textbf{Tipagem Dinâmica e Flexibilidade:}\\
R utiliza tipagem dinâmica, onde o tipo das variáveis é determinado em tempo de execução. Isso oferece flexibilidade para manipular diferentes tipos de dados, mas requer atenção para evitar inconsistências. Além disso, sua sintaxe foi projetada para facilitar operações vetoriais e manipulação de dados em larga escala, tornando-a eficiente em tarefas analíticas.

\begin{center}\rule{0.5\linewidth}{0.5pt}\end{center}

\section{Fundamentos da linguagem}\label{fundamentos-da-linguagem}

\subsection{Boas práticas de codaR}\label{boas-pruxe1ticas-de-codar}

\subsubsection{O Básico}\label{o-buxe1sico}

Acesse no menu superior \texttt{tools} \textgreater{} \texttt{Global\ Options...} \textgreater{} \texttt{code} \textgreater{} \texttt{Display} e defina os padrões de identação e estilização que deseja. E Salve.

Agora basta selecionar o trecho de código e usar o atalho \texttt{Ctrl\ +\ shift\ +\ A}(Windows).

\subsubsection{Avançado}\label{avanuxe7ado}

Para um estudo profundo verificar:

O manual de referências de boas práticas \citep{tidyverse_style_guide}

\subsection{?Dúvidas}\label{duxfavidas}

\begin{Shaded}
\begin{Highlighting}[]
\NormalTok{?mean}
\CommentTok{\# ou}
\FunctionTok{help}\NormalTok{(mean)}
\end{Highlighting}
\end{Shaded}

\subsection{Tipos de dados}\label{tipos-de-dados}

Conhecer os tipos de dados utilizados pela linguagem é essencial para sua manipulação e análise. A seguir, estão os principais tipos de dados:

\begin{longtable}[]{@{}
  >{\raggedright\arraybackslash}p{(\columnwidth - 4\tabcolsep) * \real{0.1558}}
  >{\raggedright\arraybackslash}p{(\columnwidth - 4\tabcolsep) * \real{0.4545}}
  >{\raggedright\arraybackslash}p{(\columnwidth - 4\tabcolsep) * \real{0.3896}}@{}}
\toprule\noalign{}
\begin{minipage}[b]{\linewidth}\raggedright
Tipo
\end{minipage} & \begin{minipage}[b]{\linewidth}\raggedright
Descrição
\end{minipage} & \begin{minipage}[b]{\linewidth}\raggedright
Exemplo
\end{minipage} \\
\midrule\noalign{}
\endhead
\bottomrule\noalign{}
\endlastfoot
\texttt{numeric} & Números, incluindo inteiros e reais & \texttt{10}, \texttt{10.5} \\
\texttt{integer} & Números inteiros & \texttt{as.integer(10)} \\
\texttt{character} & Cadeias de caracteres (texto) & \texttt{"Olá"} \\
\texttt{logical} & Valores booleanos & \texttt{TRUE}, \texttt{FALSE} \\
\texttt{complex} & Números complexos & \texttt{1+2i} \\
\texttt{factor} & Variáveis categóricas & \texttt{factor(c("A",\ "B",\ "A"))} \\
\texttt{list} & Lista de elementos heterogêneos & \texttt{list(1,\ "a",\ TRUE)} \\
\texttt{data.frame} & Tabela de dados estruturados & \texttt{data.frame(x\ =\ 1:3,\ y\ =\ 4:6)} \\
\texttt{matrix} & Tabela de dados bidimensional & \texttt{matrix(1:6,\ nrow\ =\ 2)} \\
\texttt{array} & Dados com mais de duas dimensões & \texttt{array(1:8,\ dim\ =\ c(2,\ 2,\ 2))} \\
\texttt{NULL} & Representa a ausência de valor & \texttt{NULL} \\
\texttt{NA} & Valor ausente ou indefinido & \texttt{NA} \\
\end{longtable}

\subsection{Operadores}\label{operadores}

Há inúmeras operações que podem ser realizadas com os dados dentro da lógica da programação. A seguir, estão os principais operadores:

\begin{longtable}[]{@{}
  >{\raggedright\arraybackslash}p{(\columnwidth - 6\tabcolsep) * \real{0.1646}}
  >{\raggedright\arraybackslash}p{(\columnwidth - 6\tabcolsep) * \real{0.1772}}
  >{\raggedright\arraybackslash}p{(\columnwidth - 6\tabcolsep) * \real{0.3544}}
  >{\raggedright\arraybackslash}p{(\columnwidth - 6\tabcolsep) * \real{0.3038}}@{}}
\toprule\noalign{}
\begin{minipage}[b]{\linewidth}\raggedright
Categoria
\end{minipage} & \begin{minipage}[b]{\linewidth}\raggedright
Operador
\end{minipage} & \begin{minipage}[b]{\linewidth}\raggedright
Descrição
\end{minipage} & \begin{minipage}[b]{\linewidth}\raggedright
Exemplo
\end{minipage} \\
\midrule\noalign{}
\endhead
\bottomrule\noalign{}
\endlastfoot
Aritmético & \texttt{+} & Adição & \texttt{x\ +\ y} \\
& \texttt{-} & Subtração & \texttt{x\ -\ y} \\
& \texttt{*} & Multiplicação & \texttt{x\ *\ y} \\
& \texttt{/} & Divisão & \texttt{x\ /\ y} \\
& \texttt{\^{}} & Exponenciação & \texttt{x\ \^{}\ y} \\
& \texttt{\%\%} & Módulo (resto da divisão) & \texttt{x\ \%\%\ y} \\
& \texttt{\%/\%} & Divisão inteira & \texttt{x\ \%/\%\ y} \\
Comparação & \texttt{==} & Igual a & \texttt{x\ ==\ y} \\
& \texttt{!=} & Diferente de & \texttt{x\ !=\ y} \\
& \texttt{\textgreater{}} & Maior que & \texttt{x\ \textgreater{}\ y} \\
& \texttt{\textless{}} & Menor que & \texttt{x\ \textless{}\ y} \\
& \texttt{\textgreater{}=} & Maior ou igual a & \texttt{x\ \textgreater{}=\ y} \\
& \texttt{\textless{}=} & Menor ou igual a & \texttt{x\ \textless{}=\ y} \\
Lógico & \texttt{\&} & AND elemento por elemento & \texttt{x\ \&\ y} \\
& \texttt{\textbar{}} & OR elemento por elemento & \texttt{x\ \ \ \ \ \ \ \ \ \ \ \ \ \ \ \ \ \ \ \ \ \ \ \ \ \ \textbar{}\ y} \\
& \texttt{\&\&} & AND para o primeiro elemento & \texttt{x\ \&\&\ y} \\
& \texttt{\textbar{}\ \ \ \ \ \ \ \ \ \ \ \ \ \ \ \ \ \ \ \ \ \ \ \ \ \ \ \ \ \ \textbar{}} & OR para o primeiro elemento & \texttt{x\ \textbar{}\ \ \textbar{}\ y} \\
& \texttt{!} & NOT lógico & \texttt{!x} \\
Atribuição & \texttt{\textless{}-} & Atribui um valor & \texttt{x\ \textless{}-\ y} \\
& \texttt{-\textgreater{}} & Atribui um valor & \texttt{y\ -\textgreater{}\ x} \\
& \texttt{=} & Atribui um valor & \texttt{x\ =\ y} \\
Pertencimento & \texttt{\%in\%} & Pertencimento (contém) & \texttt{x\ \%in\%\ y} \\
Bit a bit & \texttt{bitwAnd()} & AND bit a bit & \texttt{bitwAnd(x,\ y)} \\
& \texttt{bitwOr()} & OR bit a bit & \texttt{bitwOr(x,\ y)} \\
& \texttt{bitwXor()} & XOR bit a bit & \texttt{bitwXor(x,\ y)} \\
& \texttt{bitwNot()} & NOT bit a bit & \texttt{bitwNot(x)} \\
& \texttt{bitwShiftL()} & Deslocamento à esquerda & \texttt{bitwShiftL(x,\ n)} \\
& \texttt{bitwShiftR()} & Deslocamento à direita & \texttt{bitwShiftR(x,\ n)} \\
\end{longtable}

\subsection{Conversão de tipos}\label{conversuxe3o-de-tipos}

Por se tratar de uma linguagem de tipagem dinâmica, é possível converter os tipos de dados:

\begin{longtable}[]{@{}
  >{\raggedright\arraybackslash}p{(\columnwidth - 6\tabcolsep) * \real{0.1532}}
  >{\raggedright\arraybackslash}p{(\columnwidth - 6\tabcolsep) * \real{0.3514}}
  >{\raggedright\arraybackslash}p{(\columnwidth - 6\tabcolsep) * \real{0.3243}}
  >{\raggedright\arraybackslash}p{(\columnwidth - 6\tabcolsep) * \real{0.1712}}@{}}
\toprule\noalign{}
\begin{minipage}[b]{\linewidth}\raggedright
Função
\end{minipage} & \begin{minipage}[b]{\linewidth}\raggedright
Descrição
\end{minipage} & \begin{minipage}[b]{\linewidth}\raggedright
Exemplo
\end{minipage} & \begin{minipage}[b]{\linewidth}\raggedright
Resultado
\end{minipage} \\
\midrule\noalign{}
\endhead
\bottomrule\noalign{}
\endlastfoot
\texttt{as.numeric()} & Converte para tipo numérico & \texttt{as.numeric("10.5")} & \texttt{10.5} (numeric) \\
\texttt{as.integer()} & Converte para inteiro & \texttt{as.integer(10.5)} & \texttt{10} (integer) \\
\texttt{as.character()} & Converte para texto (character) & \texttt{as.character(10)} & \texttt{"10"} (character) \\
\texttt{as.logical()} & Converte para lógico (booleano) & \texttt{as.logical(1)} & \texttt{TRUE} (logical) \\
& & \texttt{as.logical(0)} & \texttt{FALSE} (logical) \\
\texttt{as.complex()} & Converte para número complexo & \texttt{as.complex(10)} & \texttt{10+0i} (complex) \\
\texttt{as.factor()} & Converte para fator (categorias) & \texttt{as.factor(c("A",\ "B"))} & \texttt{Factor} com níveis \\
\texttt{as.list()} & Converte para lista & \texttt{as.list(c(1,\ 2,\ 3))} & \texttt{list(1,\ 2,\ 3)} \\
\texttt{as.matrix()} & Converte para matriz & \texttt{as.matrix(c(1,\ 2,\ 3))} & \texttt{matrix} (2D) \\
\texttt{as.data.frame()} & Converte para data frame & \texttt{as.data.frame(matrix(1:4,\ ncol=2))} & \texttt{data.frame} \\
\texttt{as.Date()} & Converte para data (classe Date) & \texttt{as.Date("2024-11-18")} & \texttt{2024-11-18} (Date) \\
\texttt{as.POSIXct()} & Converte para data/hora & \texttt{as.POSIXct("2024-11-18\ 10:00:00")} & Data/hora (POSIXct) \\
\texttt{as.vector()} & Converte para vetor & \texttt{as.vector(matrix(1:4,\ nrow=2))} & \texttt{c(1,\ 2,\ 3,\ 4)} \\
\texttt{as.array()} & Converte para array (multi-dimensional) & \texttt{as.array(1:8)} & \texttt{array} \\
\texttt{is.numeric()} & Testa se é numérico & \texttt{is.numeric(10.5)} & \texttt{TRUE} \\
\texttt{is.character()} & Testa se é texto & \texttt{is.character("text")} & \texttt{TRUE} \\
\texttt{is.logical()} & Testa se é lógico & \texttt{is.logical(TRUE)} & \texttt{TRUE} \\
\texttt{is.factor()} & Testa se é fator & \texttt{is.factor(as.factor("A"))} & \texttt{TRUE} \\
\end{longtable}

\subsection{Estruturas de Controle}\label{estruturas-de-controle}

A lógica da programação e a construção de algoritmos dependem das estruturas de controle:

\begin{longtable}[]{@{}
  >{\raggedright\arraybackslash}p{(\columnwidth - 4\tabcolsep) * \real{0.1119}}
  >{\raggedright\arraybackslash}p{(\columnwidth - 4\tabcolsep) * \real{0.4615}}
  >{\raggedright\arraybackslash}p{(\columnwidth - 4\tabcolsep) * \real{0.4266}}@{}}
\toprule\noalign{}
\begin{minipage}[b]{\linewidth}\raggedright
Estrutura
\end{minipage} & \begin{minipage}[b]{\linewidth}\raggedright
Descrição
\end{minipage} & \begin{minipage}[b]{\linewidth}\raggedright
Exemplo
\end{minipage} \\
\midrule\noalign{}
\endhead
\bottomrule\noalign{}
\endlastfoot
\texttt{if} & Executa um bloco de código se a condição for verdadeira & \texttt{if\ (x\ \textgreater{}\ 0)\ \{\ print("Positivo")\ \}} \\
\texttt{if-else} & Executa um bloco se a condição for verdadeira e outro se for falsa & \texttt{if\ (x\ \textgreater{}\ 0)\ \{\ print("Positivo")\ \}\ else\ \{\ print("Negativo")\ \}} \\
\texttt{ifelse} & Avaliação vetorizada para condicional & \texttt{result\ \textless{}-\ ifelse(x\ \textgreater{}\ 0,\ "Positivo",\ "Negativo")} \\
\texttt{for} & Itera sobre elementos de um vetor ou lista & \texttt{for\ (i\ in\ 1:5)\ \{\ print(i)\ \}} \\
\texttt{while} & Executa um bloco enquanto a condição for verdadeira & \texttt{while\ (x\ \textless{}\ 5)\ \{\ print(x);\ x\ \textless{}-\ x\ +\ 1\ \}} \\
\texttt{repeat} & Executa um bloco até encontrar um \texttt{break} & \texttt{repeat\ \{\ if\ (x\ \textgreater{}\ 5)\ break;\ print(x);\ x\ \textless{}-\ x\ +\ 1\ \}} \\
\texttt{break} & Interrompe a execução de um laço & \texttt{for\ (i\ in\ 1:5)\ \{\ if\ (i\ ==\ 3)\ break;\ print(i)\ \}} \\
\texttt{next} & Pula para a próxima iteração de um laço & \texttt{for\ (i\ in\ 1:5)\ \{\ if\ (i\ ==\ 3)\ next;\ print(i)\ \}} \\
\texttt{switch} & Seleciona uma opção com base em uma expressão & \texttt{switch(x,\ "a"\ =\ "Opção\ A",\ "b"\ =\ "Opção\ B")} \\
Funções anônimas & Executa expressões diretamente em chamadas & \texttt{sapply(1:5,\ function(x)\ x\^{}2)} \\
\texttt{try} & Captura erros em blocos de código & \texttt{try(\{\ log("a")\ \})} \\
\texttt{tryCatch} & Captura e lida com erros e mensagens & \texttt{tryCatch(log("a"),\ error\ =\ function(e)\ print("Erro"))} \\
\end{longtable}

\begin{center}\rule{0.5\linewidth}{0.5pt}\end{center}

\section{Gerenciamento de Pacotes 📦}\label{gerenciamento-de-pacotes}

O R é uma linguagem poderosa, ampliada pelo uso de pacotes. Com pacotes, é possível adicionar novas funcionalidades e acessar ferramentas específicas para diversas tarefas, como visualização de dados, manipulação e modelagem estatística.

\subsection{O que são pacotes?}\label{o-que-suxe3o-pacotes}

Pacotes no R são coleções de funções, conjuntos de dados e documentações. Eles estendem as funcionalidades do R base e são fundamentais para tarefas mais avançadas.

\begin{itemize}
\tightlist
\item
  Os pacotes podem ser obtidos do \textbf{CRAN} (repositório oficial), \textbf{GitHub} ou outras fontes.\\
\item
  Pacotes populares incluem \texttt{ggplot2} (gráficos), \texttt{dplyr} (manipulação de dados) e \texttt{shiny} (aplicações interativas).
\end{itemize}

\subsection{Instalando pacotes}\label{instalando-pacotes}

Para instalar pacotes no R, utilize a função \texttt{install.packages()}.

\subsubsection{Exemplo básico:}\label{exemplo-buxe1sico}

\begin{Shaded}
\begin{Highlighting}[]
\CommentTok{\# Instalando um pacote do CRAN}
\FunctionTok{install.packages}\NormalTok{(}\StringTok{"ggplot2"}\NormalTok{)}
\end{Highlighting}
\end{Shaded}

\subsubsection{Instalando múltiplos pacotes:}\label{instalando-muxfaltiplos-pacotes}

\begin{Shaded}
\begin{Highlighting}[]
\FunctionTok{install.packages}\NormalTok{(}\FunctionTok{c}\NormalTok{(}\StringTok{"dplyr"}\NormalTok{, }\StringTok{"tidyr"}\NormalTok{, }\StringTok{"stringr"}\NormalTok{))}
\end{Highlighting}
\end{Shaded}

\subsubsection{Instalando pacotes do GitHub:}\label{instalando-pacotes-do-github}

Pacotes disponíveis no GitHub requerem o pacote \texttt{remotes} ou \texttt{devtools} para instalação:

\begin{Shaded}
\begin{Highlighting}[]
\CommentTok{\# Instalando o pacote remotes}
\FunctionTok{install.packages}\NormalTok{(}\StringTok{"remotes"}\NormalTok{)}

\CommentTok{\# Instalando um pacote do GitHub}
\NormalTok{remotes}\SpecialCharTok{::}\FunctionTok{install\_github}\NormalTok{(}\StringTok{"tidyverse/ggplot2"}\NormalTok{)}
\end{Highlighting}
\end{Shaded}

\subsection{Carregando pacotes}\label{carregando-pacotes}

Depois de instalados, os pacotes precisam ser carregados para uso. Isso é feito com a função \texttt{library()}.

\subsubsection{Exemplo:}\label{exemplo}

\begin{Shaded}
\begin{Highlighting}[]
\CommentTok{\# Carregando o pacote ggplot2}
\FunctionTok{library}\NormalTok{(ggplot2)}
\end{Highlighting}
\end{Shaded}

Se você não quiser carregar o pacote inteiro, pode usar o operador \texttt{::} para chamar uma função específica:

\begin{Shaded}
\begin{Highlighting}[]
\CommentTok{\# Usando a função ggplot() sem carregar o pacote}
\NormalTok{ggplot2}\SpecialCharTok{::}\FunctionTok{ggplot}\NormalTok{(}\AttributeTok{data =}\NormalTok{ mtcars, }\FunctionTok{aes}\NormalTok{(}\AttributeTok{x =}\NormalTok{ mpg, }\AttributeTok{y =}\NormalTok{ hp))}
\end{Highlighting}
\end{Shaded}

\subsection{Atualizando pacotes}\label{atualizando-pacotes}

Pacotes instalados podem ser atualizados para suas versões mais recentes.

\subsubsection{Atualizando todos os pacotes:}\label{atualizando-todos-os-pacotes}

\begin{Shaded}
\begin{Highlighting}[]
\FunctionTok{update.packages}\NormalTok{()}
\end{Highlighting}
\end{Shaded}

\subsubsection{Atualizando pacotes específicos:}\label{atualizando-pacotes-especuxedficos}

Reinstale o pacote desejado:

\begin{Shaded}
\begin{Highlighting}[]
\FunctionTok{install.packages}\NormalTok{(}\StringTok{"dplyr"}\NormalTok{)}
\end{Highlighting}
\end{Shaded}

\subsection{Removendo pacotes}\label{removendo-pacotes}

Se um pacote não for mais necessário, você pode removê-lo com a função \texttt{remove.packages()}.

\subsubsection{Exemplo:}\label{exemplo-1}

\begin{Shaded}
\begin{Highlighting}[]
\FunctionTok{remove.packages}\NormalTok{(}\StringTok{"stringr"}\NormalTok{)}
\end{Highlighting}
\end{Shaded}

\subsection{Verificando pacotes instalados}\label{verificando-pacotes-instalados}

Você pode listar todos os pacotes instalados ou verificar se um pacote específico está presente.

\subsubsection{Listando todos os pacotes:}\label{listando-todos-os-pacotes}

\begin{Shaded}
\begin{Highlighting}[]
\FunctionTok{installed.packages}\NormalTok{()}
\end{Highlighting}
\end{Shaded}

\subsubsection{Verificando se um pacote está instalado:}\label{verificando-se-um-pacote-estuxe1-instalado}

\begin{Shaded}
\begin{Highlighting}[]
\StringTok{"ggplot2"} \SpecialCharTok{\%in\%} \FunctionTok{rownames}\NormalTok{(}\FunctionTok{installed.packages}\NormalTok{())}
\end{Highlighting}
\end{Shaded}

\subsection{Repositórios}\label{reposituxf3rios}

O R utiliza o \textbf{CRAN} como repositório padrão para instalação de pacotes. No entanto, você pode configurar o repositório manualmente, por exemplo, escolhendo um mirror mais rápido:

\begin{Shaded}
\begin{Highlighting}[]
\CommentTok{\# Configurando um repositório brasileiro}
\FunctionTok{options}\NormalTok{(}\AttributeTok{repos =} \FunctionTok{c}\NormalTok{(}\AttributeTok{CRAN =} \StringTok{"https://cran.rstudio.com/"}\NormalTok{))}
\end{Highlighting}
\end{Shaded}

\subsection{Ferramentas avançadas para gerenciamento}\label{ferramentas-avanuxe7adas-para-gerenciamento}

Para projetos complexos, você pode usar ferramentas para gerenciar versões de pacotes e ambientes:

\begin{itemize}
\item
  \textbf{\texttt{renv}}: Cria ambientes isolados para projetos, garantindo que as dependências permaneçam consistentes:

\begin{Shaded}
\begin{Highlighting}[]
\FunctionTok{install.packages}\NormalTok{(}\StringTok{"renv"}\NormalTok{)}
\NormalTok{renv}\SpecialCharTok{::}\FunctionTok{init}\NormalTok{()}
\end{Highlighting}
\end{Shaded}
\item
  \textbf{\texttt{packrat}}: Alternativa mais antiga ao \texttt{renv}, também para isolamento de dependências.
\item
  \textbf{\texttt{conda}}: Se você utiliza Python junto com R, o \texttt{conda} permite gerenciar pacotes R em ambientes híbridos.
\end{itemize}

\begin{center}\rule{0.5\linewidth}{0.5pt}\end{center}

\section{Importação e Exportação de Dados 📂}\label{importauxe7uxe3o-e-exportauxe7uxe3o-de-dados}

O R oferece diversas ferramentas para \textbf{importar} e \textbf{exportar dados} em diferentes formatos, permitindo que você trabalhe com arquivos como CSV, Excel, bancos de dados, arquivos de texto e muito mais.

\subsection{Importando dados}\label{importando-dados}

\subsubsection{\texorpdfstring{\textbf{1. Arquivos CSV}}{1. Arquivos CSV}}\label{arquivos-csv}

O formato CSV é amplamente utilizado e facilmente manipulável no R com a função \texttt{read.csv()} ou a função mais moderna \texttt{readr::read\_csv()}.

\begin{Shaded}
\begin{Highlighting}[]
\CommentTok{\# Usando a função base}
\NormalTok{dados }\OtherTok{\textless{}{-}} \FunctionTok{read.csv}\NormalTok{(}\StringTok{"caminho/para/seu/arquivo.csv"}\NormalTok{)}

\CommentTok{\# Usando o pacote readr (do Tidyverse)}
\NormalTok{dados }\OtherTok{\textless{}{-}}\NormalTok{ readr}\SpecialCharTok{::}\FunctionTok{read\_csv}\NormalTok{(}\StringTok{"caminho/para/seu/arquivo.csv"}\NormalTok{)}
\end{Highlighting}
\end{Shaded}

\subsubsection{\texorpdfstring{\textbf{2. Arquivos Excel}}{2. Arquivos Excel}}\label{arquivos-excel}

Para importar dados de planilhas Excel, você pode usar pacotes como \texttt{readxl} ou \texttt{openxlsx}.

\begin{Shaded}
\begin{Highlighting}[]
\CommentTok{\# Instale o pacote readxl, se necessário}
\FunctionTok{install.packages}\NormalTok{(}\StringTok{"readxl"}\NormalTok{)}

\CommentTok{\# Lendo uma planilha Excel}
\FunctionTok{library}\NormalTok{(readxl)}
\NormalTok{dados }\OtherTok{\textless{}{-}} \FunctionTok{read\_excel}\NormalTok{(}\StringTok{"caminho/para/seu/arquivo.xlsx"}\NormalTok{, }\AttributeTok{sheet =} \DecValTok{1}\NormalTok{)}
\end{Highlighting}
\end{Shaded}

\subsubsection{\texorpdfstring{\textbf{3. Arquivos de texto}}{3. Arquivos de texto}}\label{arquivos-de-texto}

Para arquivos de texto delimitados (como TSV), use \texttt{read.delim()} ou \texttt{readr::read\_tsv()}.

\begin{Shaded}
\begin{Highlighting}[]
\CommentTok{\# Arquivo delimitado por tabulação}
\NormalTok{dados }\OtherTok{\textless{}{-}} \FunctionTok{read.delim}\NormalTok{(}\StringTok{"caminho/para/seu/arquivo.txt"}\NormalTok{)}

\CommentTok{\# Com readr}
\NormalTok{dados }\OtherTok{\textless{}{-}}\NormalTok{ readr}\SpecialCharTok{::}\FunctionTok{read\_tsv}\NormalTok{(}\StringTok{"caminho/para/seu/arquivo.txt"}\NormalTok{)}
\end{Highlighting}
\end{Shaded}

\subsubsection{\texorpdfstring{\textbf{4. Bancos de Dados}}{4. Bancos de Dados}}\label{bancos-de-dados}

Para se conectar a bancos de dados, use pacotes como \texttt{DBI} e \texttt{RSQLite}.

\begin{Shaded}
\begin{Highlighting}[]
\CommentTok{\# Exemplo com SQLite}
\FunctionTok{install.packages}\NormalTok{(}\StringTok{"DBI"}\NormalTok{)}
\FunctionTok{install.packages}\NormalTok{(}\StringTok{"RSQLite"}\NormalTok{)}

\FunctionTok{library}\NormalTok{(DBI)}
\NormalTok{con }\OtherTok{\textless{}{-}} \FunctionTok{dbConnect}\NormalTok{(RSQLite}\SpecialCharTok{::}\FunctionTok{SQLite}\NormalTok{(), }\StringTok{"caminho/para/seu/banco.sqlite"}\NormalTok{)}
\NormalTok{dados }\OtherTok{\textless{}{-}} \FunctionTok{dbReadTable}\NormalTok{(con, }\StringTok{"nome\_da\_tabela"}\NormalTok{)}
\FunctionTok{dbDisconnect}\NormalTok{(con)}
\end{Highlighting}
\end{Shaded}

\subsubsection{\texorpdfstring{\textbf{5. Outros formatos populares}}{5. Outros formatos populares}}\label{outros-formatos-populares}

\begin{itemize}
\tightlist
\item
  JSON: Use o pacote \texttt{jsonlite}.\\
\item
  XML: Use o pacote \texttt{xml2}.\\
\item
  Arquivos SPSS, Stata e SAS: Use o pacote \texttt{haven}.
\end{itemize}

\begin{Shaded}
\begin{Highlighting}[]
\CommentTok{\# Exemplo de JSON}
\FunctionTok{install.packages}\NormalTok{(}\StringTok{"jsonlite"}\NormalTok{)}
\NormalTok{dados }\OtherTok{\textless{}{-}}\NormalTok{ jsonlite}\SpecialCharTok{::}\FunctionTok{fromJSON}\NormalTok{(}\StringTok{"caminho/para/seu/arquivo.json"}\NormalTok{)}
\end{Highlighting}
\end{Shaded}

\subsection{Exportando dados}\label{exportando-dados}

\subsubsection{\texorpdfstring{\textbf{1. Exportando para CSV}}{1. Exportando para CSV}}\label{exportando-para-csv}

A função \texttt{write.csv()} é usada para salvar dados em formato CSV.

\begin{Shaded}
\begin{Highlighting}[]
\CommentTok{\# Salvando um data frame como CSV}
\FunctionTok{write.csv}\NormalTok{(dados, }\StringTok{"caminho/para/saida.csv"}\NormalTok{, }\AttributeTok{row.names =} \ConstantTok{FALSE}\NormalTok{)}
\end{Highlighting}
\end{Shaded}

\subsubsection{\texorpdfstring{\textbf{2. Exportando para Excel}}{2. Exportando para Excel}}\label{exportando-para-excel}

Para salvar dados em Excel, use o pacote \texttt{openxlsx}.

\begin{Shaded}
\begin{Highlighting}[]
\CommentTok{\# Instale o pacote openxlsx, se necessário}
\FunctionTok{install.packages}\NormalTok{(}\StringTok{"openxlsx"}\NormalTok{)}

\CommentTok{\# Escrevendo dados em uma planilha Excel}
\FunctionTok{library}\NormalTok{(openxlsx)}
\FunctionTok{write.xlsx}\NormalTok{(dados, }\StringTok{"caminho/para/saida.xlsx"}\NormalTok{)}
\end{Highlighting}
\end{Shaded}

\subsubsection{\texorpdfstring{\textbf{3. Exportando para outros formatos}}{3. Exportando para outros formatos}}\label{exportando-para-outros-formatos}

O R permite salvar dados em diversos formatos com as funções correspondentes:
- Texto (\texttt{write.table()})
- JSON (\texttt{jsonlite::toJSON()})
- Bancos de dados (\texttt{dbWriteTable()}).

\textbf{Exemplo com JSON:}

\begin{Shaded}
\begin{Highlighting}[]
\NormalTok{jsonlite}\SpecialCharTok{::}\FunctionTok{toJSON}\NormalTok{(dados, }\AttributeTok{pretty =} \ConstantTok{TRUE}\NormalTok{, }\AttributeTok{file =} \StringTok{"caminho/para/saida.json"}\NormalTok{)}
\end{Highlighting}
\end{Shaded}

\subsection{Dicas e boas práticas}\label{dicas-e-boas-pruxe1ticas}

\begin{itemize}
\item
  Sempre verifique os tipos de dados importados com \texttt{str()} ou \texttt{glimpse()}.\\
\item
  Para grandes volumes de dados, considere pacotes otimizados como \texttt{data.table} ou \texttt{vroom}.\\
\item
  Configure o diretório de trabalho corretamente para evitar erros de caminho:

\begin{Shaded}
\begin{Highlighting}[]
\FunctionTok{setwd}\NormalTok{(}\StringTok{"caminho/para/seu/diretorio"}\NormalTok{)}
\end{Highlighting}
\end{Shaded}
\end{itemize}

\begin{center}\rule{0.5\linewidth}{0.5pt}\end{center}

\section{\texorpdfstring{Funções da família \texttt{apply()} no R}{Funções da família apply() no R}}\label{funuxe7uxf5es-da-famuxedlia-apply-no-r}

As funções da família \texttt{apply()} são usadas para realizar operações repetitivas em vetores, listas, matrizes e data frames de forma eficiente e vetorizada, substituindo muitos loops explícitos (\texttt{for}).

\subsection{\texorpdfstring{\textbf{1. \texttt{apply()}}}{1. apply()}}\label{apply}

Aplica uma função ao longo de uma \textbf{margem} de uma matriz ou array (linhas ou colunas).

\begin{itemize}
\tightlist
\item
  \textbf{Uso:} Para matrizes e arrays.\\
\item
  \textbf{Sintaxe:} \texttt{apply(X,\ MARGIN,\ FUN,\ ...)}

  \begin{itemize}
  \tightlist
  \item
    \texttt{X}: Matriz ou array.\\
  \item
    \texttt{MARGIN}: Margem a operar:

    \begin{itemize}
    \tightlist
    \item
      \texttt{1} para linhas.
    \item
      \texttt{2} para colunas.\\
    \end{itemize}
  \item
    \texttt{FUN}: Função a ser aplicada.
  \end{itemize}
\end{itemize}

\textbf{Exemplo:}

\begin{Shaded}
\begin{Highlighting}[]
\CommentTok{\# Soma das colunas de uma matriz}
\NormalTok{mat }\OtherTok{\textless{}{-}} \FunctionTok{matrix}\NormalTok{(}\DecValTok{1}\SpecialCharTok{:}\DecValTok{9}\NormalTok{, }\AttributeTok{nrow =} \DecValTok{3}\NormalTok{)}
\FunctionTok{apply}\NormalTok{(mat, }\DecValTok{2}\NormalTok{, sum)}
\CommentTok{\# Resultado: [1] 12 15 18}

\CommentTok{\# Média das linhas}
\FunctionTok{apply}\NormalTok{(mat, }\DecValTok{1}\NormalTok{, mean)}
\CommentTok{\# Resultado: [1] 2 5 8}
\end{Highlighting}
\end{Shaded}

\subsection{\texorpdfstring{\textbf{2. \texttt{lapply()}}}{2. lapply()}}\label{lapply}

Aplica uma função a cada elemento de uma \textbf{lista} ou \textbf{vetor} e retorna uma \textbf{lista} como resultado.

\begin{itemize}
\tightlist
\item
  \textbf{Uso:} Para listas e vetores.\\
\item
  \textbf{Sintaxe:} \texttt{lapply(X,\ FUN,\ ...)}
\end{itemize}

\textbf{Exemplo:}

\begin{Shaded}
\begin{Highlighting}[]
\CommentTok{\# Quadrado de cada elemento de um vetor}
\NormalTok{vec }\OtherTok{\textless{}{-}} \DecValTok{1}\SpecialCharTok{:}\DecValTok{5}
\FunctionTok{lapply}\NormalTok{(vec, }\ControlFlowTok{function}\NormalTok{(x) x}\SpecialCharTok{\^{}}\DecValTok{2}\NormalTok{)}
\CommentTok{\# Resultado: [[1]] 1, [[2]] 4, [[3]] 9, [[4]] 16, [[5]] 25}

\CommentTok{\# Aplicação em uma lista}
\NormalTok{lst }\OtherTok{\textless{}{-}} \FunctionTok{list}\NormalTok{(}\AttributeTok{a =} \DecValTok{1}\SpecialCharTok{:}\DecValTok{3}\NormalTok{, }\AttributeTok{b =} \DecValTok{4}\SpecialCharTok{:}\DecValTok{6}\NormalTok{)}
\FunctionTok{lapply}\NormalTok{(lst, sum)}
\CommentTok{\# Resultado: $a 6, $b 15}
\end{Highlighting}
\end{Shaded}

\subsection{\texorpdfstring{\textbf{3. \texttt{sapply()}}}{3. sapply()}}\label{sapply}

Uma versão simplificada de \texttt{lapply()} que retorna um \textbf{vetor} ou \textbf{matriz} (se possível) em vez de uma lista.

\begin{itemize}
\tightlist
\item
  \textbf{Uso:} Para listas e vetores.\\
\item
  \textbf{Sintaxe:} \texttt{sapply(X,\ FUN,\ ...)}
\end{itemize}

\textbf{Exemplo:}

\begin{Shaded}
\begin{Highlighting}[]
\CommentTok{\# Quadrado de cada elemento}
\NormalTok{vec }\OtherTok{\textless{}{-}} \DecValTok{1}\SpecialCharTok{:}\DecValTok{5}
\FunctionTok{sapply}\NormalTok{(vec, }\ControlFlowTok{function}\NormalTok{(x) x}\SpecialCharTok{\^{}}\DecValTok{2}\NormalTok{)}
\CommentTok{\# Resultado: [1] 1 4 9 16 25}

\CommentTok{\# Soma dos elementos em uma lista}
\NormalTok{lst }\OtherTok{\textless{}{-}} \FunctionTok{list}\NormalTok{(}\AttributeTok{a =} \DecValTok{1}\SpecialCharTok{:}\DecValTok{3}\NormalTok{, }\AttributeTok{b =} \DecValTok{4}\SpecialCharTok{:}\DecValTok{6}\NormalTok{)}
\FunctionTok{sapply}\NormalTok{(lst, sum)}
\CommentTok{\# Resultado: [1] 6 15}
\end{Highlighting}
\end{Shaded}

\subsection{\texorpdfstring{\textbf{4. \texttt{vapply()}}}{4. vapply()}}\label{vapply}

Semelhante a \texttt{sapply()}, mas requer que você \textbf{especifique o tipo de retorno esperado} (mais seguro).

\begin{itemize}
\tightlist
\item
  \textbf{Uso:} Para listas e vetores.\\
\item
  \textbf{Sintaxe:} \texttt{vapply(X,\ FUN,\ FUN.VALUE,\ ...)}

  \begin{itemize}
  \tightlist
  \item
    \texttt{FUN.VALUE}: Define o tipo e formato esperado do retorno.
  \end{itemize}
\end{itemize}

\textbf{Exemplo:}

\begin{Shaded}
\begin{Highlighting}[]
\CommentTok{\# Soma dos elementos em uma lista com tipo esperado (numeric)}
\NormalTok{lst }\OtherTok{\textless{}{-}} \FunctionTok{list}\NormalTok{(}\AttributeTok{a =} \DecValTok{1}\SpecialCharTok{:}\DecValTok{3}\NormalTok{, }\AttributeTok{b =} \DecValTok{4}\SpecialCharTok{:}\DecValTok{6}\NormalTok{)}
\FunctionTok{vapply}\NormalTok{(lst, sum, }\FunctionTok{numeric}\NormalTok{(}\DecValTok{1}\NormalTok{))}
\CommentTok{\# Resultado: [1]  6 15}
\end{Highlighting}
\end{Shaded}

\subsection{\texorpdfstring{\textbf{5. \texttt{mapply()}}}{5. mapply()}}\label{mapply}

Aplica uma função a múltiplos argumentos/vetores \textbf{simultaneamente} (como um \texttt{map} em Python).

\begin{itemize}
\tightlist
\item
  \textbf{Uso:} Para múltiplos vetores ou listas.\\
\item
  \textbf{Sintaxe:} \texttt{mapply(FUN,\ ...,\ MoreArgs\ =\ NULL)}
\end{itemize}

\textbf{Exemplo:}

\begin{Shaded}
\begin{Highlighting}[]
\CommentTok{\# Soma de dois vetores}
\NormalTok{vec1 }\OtherTok{\textless{}{-}} \DecValTok{1}\SpecialCharTok{:}\DecValTok{5}
\NormalTok{vec2 }\OtherTok{\textless{}{-}} \DecValTok{6}\SpecialCharTok{:}\DecValTok{10}
\FunctionTok{mapply}\NormalTok{(sum, vec1, vec2)}
\CommentTok{\# Resultado: [1]  7  9 11 13 15}

\CommentTok{\# Repetição customizada}
\FunctionTok{mapply}\NormalTok{(rep, }\DecValTok{1}\SpecialCharTok{:}\DecValTok{3}\NormalTok{, }\DecValTok{3}\SpecialCharTok{:}\DecValTok{1}\NormalTok{)}
\CommentTok{\# Resultado: [[1]] 1, [[2]] 2 2, [[3]] 3 3 3}
\end{Highlighting}
\end{Shaded}

\subsection{\texorpdfstring{\textbf{6. \texttt{tapply()}}}{6. tapply()}}\label{tapply}

Aplica uma função a subconjuntos de um vetor, definidos por um \textbf{fator} ou grupos.

\begin{itemize}
\tightlist
\item
  \textbf{Uso:} Para agrupamento.\\
\item
  \textbf{Sintaxe:} \texttt{tapply(X,\ INDEX,\ FUN,\ ...)}

  \begin{itemize}
  \tightlist
  \item
    \texttt{X}: Vetor numérico.
  \item
    \texttt{INDEX}: Fator ou lista de fatores para agrupar.
  \end{itemize}
\end{itemize}

\textbf{Exemplo:}

\begin{Shaded}
\begin{Highlighting}[]
\CommentTok{\# Média por grupo}
\NormalTok{vec }\OtherTok{\textless{}{-}} \FunctionTok{c}\NormalTok{(}\DecValTok{1}\NormalTok{, }\DecValTok{2}\NormalTok{, }\DecValTok{3}\NormalTok{, }\DecValTok{4}\NormalTok{, }\DecValTok{5}\NormalTok{, }\DecValTok{6}\NormalTok{)}
\NormalTok{grp }\OtherTok{\textless{}{-}} \FunctionTok{factor}\NormalTok{(}\FunctionTok{c}\NormalTok{(}\StringTok{"A"}\NormalTok{, }\StringTok{"A"}\NormalTok{, }\StringTok{"B"}\NormalTok{, }\StringTok{"B"}\NormalTok{, }\StringTok{"C"}\NormalTok{, }\StringTok{"C"}\NormalTok{))}
\FunctionTok{tapply}\NormalTok{(vec, grp, mean)}
\CommentTok{\# Resultado: $A 1.5, $B 3.5, $C 5.5}
\end{Highlighting}
\end{Shaded}

\subsection{\texorpdfstring{\textbf{7. \texttt{by()}}}{7. by()}}\label{by}

Semelhante a \texttt{tapply()}, mas retorna resultados organizados por \textbf{subgrupos} e funciona com data frames.

\begin{itemize}
\tightlist
\item
  \textbf{Uso:} Para data frames.\\
\item
  \textbf{Sintaxe:} \texttt{by(data,\ INDICES,\ FUN,\ ...)}
\end{itemize}

\textbf{Exemplo:}

\begin{Shaded}
\begin{Highlighting}[]
\CommentTok{\# Soma dos valores por grupo em um data frame}
\NormalTok{df }\OtherTok{\textless{}{-}} \FunctionTok{data.frame}\NormalTok{(}\AttributeTok{value =} \DecValTok{1}\SpecialCharTok{:}\DecValTok{6}\NormalTok{, }\AttributeTok{group =} \FunctionTok{c}\NormalTok{(}\StringTok{"A"}\NormalTok{, }\StringTok{"A"}\NormalTok{, }\StringTok{"B"}\NormalTok{, }\StringTok{"B"}\NormalTok{, }\StringTok{"C"}\NormalTok{, }\StringTok{"C"}\NormalTok{))}
\FunctionTok{by}\NormalTok{(df}\SpecialCharTok{$}\NormalTok{value, df}\SpecialCharTok{$}\NormalTok{group, sum)}
\CommentTok{\# Resultado:}
\CommentTok{\# A: 3}
\CommentTok{\# B: 7}
\CommentTok{\# C: 11}
\end{Highlighting}
\end{Shaded}

\subsection*{Conclusão}\label{conclusuxe3o}
\addcontentsline{toc}{subsection}{Conclusão}

As funções da família \texttt{apply()} tornam o R extremamente poderoso e eficiente para manipulação de dados. Use:
- \textbf{\texttt{apply()}} para matrizes/arrays.
- \textbf{\texttt{lapply()}} e \textbf{\texttt{sapply()}} para listas/vetores.
- \textbf{\texttt{mapply()}} para múltiplos vetores.
- \textbf{\texttt{tapply()}} e \textbf{\texttt{by()}} para agrupamentos.

\begin{center}\rule{0.5\linewidth}{0.5pt}\end{center}

\section{\texorpdfstring{Pipe (\texttt{\%\textgreater{}\%}) do dplyr}{Pipe (\%\textgreater\%) do dplyr}}\label{pipe-do-dplyr}

\subsection{\texorpdfstring{O que é o \texttt{\%\textgreater{}\%}?}{O que é o \%\textgreater\%?}}\label{o-que-uxe9-o}

O operador \texttt{\%\textgreater{}\%} é chamado de pipe.
Ele permite encadear funções de maneira a facilitar a leitura e compreensão do código.
Ao invés de ficar atribuindo valores intermediários ou aninhando várias funções, você simplesmente passa o resultado de uma função como entrada para a próxima.

\subsection{Exemplos de Uso}\label{exemplos-de-uso}

Vamos usar o conjunto de dados mtcars para entender como o pipe facilita a manipulação de dados:

Sem \texttt{\%\textgreater{}\%} (Sem Pipe)
Imagine que queremos fazer as seguintes operações no dataset mtcars:

Filtrar apenas os carros que têm mais de 20 milhas por galão (mpg \textgreater{} 20).
Selecionar apenas as colunas mpg, cyl, e hp.
Sem o operador \texttt{\%\textgreater{}\%}, o código ficaria assim:

\begin{Shaded}
\begin{Highlighting}[]
\FunctionTok{library}\NormalTok{(dplyr)}

\CommentTok{\# Sem pipe}
\NormalTok{filtered\_data }\OtherTok{\textless{}{-}} \FunctionTok{filter}\NormalTok{(mtcars, mpg }\SpecialCharTok{\textgreater{}} \DecValTok{20}\NormalTok{)}
\NormalTok{selected\_data }\OtherTok{\textless{}{-}} \FunctionTok{select}\NormalTok{(filtered\_data, mpg, cyl, hp)}
\end{Highlighting}
\end{Shaded}

Com \texttt{\%\textgreater{}\%} (Com Pipe)
Podemos simplificar isso usando o pipe:

\begin{Shaded}
\begin{Highlighting}[]
\FunctionTok{library}\NormalTok{(dplyr)}

\CommentTok{\# Usando pipe}
\NormalTok{selected\_data }\OtherTok{\textless{}{-}}\NormalTok{ mtcars }\SpecialCharTok{\%\textgreater{}\%}
  \FunctionTok{filter}\NormalTok{(mpg }\SpecialCharTok{\textgreater{}} \DecValTok{20}\NormalTok{) }\SpecialCharTok{\%\textgreater{}\%}
  \FunctionTok{select}\NormalTok{(mpg, cyl, hp)}
\end{Highlighting}
\end{Shaded}

\begin{table}
\centering
\begin{tabular}{l|r|r|r}
\hline
  & mpg & cyl & hp\\
\hline
Mazda RX4 & 21.0 & 6 & 110\\
\hline
Mazda RX4 Wag & 21.0 & 6 & 110\\
\hline
Datsun 710 & 22.8 & 4 & 93\\
\hline
Hornet 4 Drive & 21.4 & 6 & 110\\
\hline
Merc 240D & 24.4 & 4 & 62\\
\hline
Merc 230 & 22.8 & 4 & 95\\
\hline
\end{tabular}
\end{table}

Note como o código com \texttt{\%\textgreater{}\%} é mais legível e direto. Ele segue uma lógica sequencial que permite facilmente entender o que está sendo feito em cada etapa.

\begin{center}\rule{0.5\linewidth}{0.5pt}\end{center}

\section{data.table}\label{data.table}

\subsection{\texorpdfstring{O que é o \texttt{:=}?}{O que é o :=?}}\label{o-que-uxe9-o-1}

O operador \texttt{:=} é chamado de operador de atribuição por referência.
Ele pertence ao pacote data.table e é usado para adicionar ou modificar colunas em um data.table.
O \texttt{:=} é mais eficiente do que usar \texttt{\textless{}-} em um data.table, pois faz as modificações ``por referência'', o que significa que não cria cópias desnecessárias dos dados, sendo muito mais rápido e eficiente em termos de memória.

\subsection{Exemplos de Uso}\label{exemplos-de-uso-1}

Vamos usar o pacote data.table para entender como o \texttt{:=}
funciona.

Primeiro, vamos criar um data.table com alguns dados fictícios:

\begin{Shaded}
\begin{Highlighting}[]
\FunctionTok{library}\NormalTok{(data.table)}

\CommentTok{\# Criação de um data.table}
\NormalTok{DT }\OtherTok{\textless{}{-}} \FunctionTok{data.table}\NormalTok{(}
  \AttributeTok{id =} \DecValTok{1}\SpecialCharTok{:}\DecValTok{5}\NormalTok{,}
  \AttributeTok{valor =} \FunctionTok{c}\NormalTok{(}\DecValTok{10}\NormalTok{, }\DecValTok{15}\NormalTok{, }\DecValTok{20}\NormalTok{, }\DecValTok{25}\NormalTok{, }\DecValTok{30}\NormalTok{)}
\NormalTok{)}
\end{Highlighting}
\end{Shaded}

Adicionando uma Nova Coluna com \texttt{:=}

Podemos adicionar uma nova coluna chamada valor\_2 que seja o dobro do valor existente na coluna valor:

\begin{Shaded}
\begin{Highlighting}[]
\NormalTok{DT[, valor\_2 }\SpecialCharTok{:=}\NormalTok{ valor }\SpecialCharTok{*} \DecValTok{2}\NormalTok{]}

\FunctionTok{print}\NormalTok{(DT)}
\end{Highlighting}
\end{Shaded}

\begin{verbatim}
##    id valor valor_2
## 1:  1    10      20
## 2:  2    15      30
## 3:  3    20      40
## 4:  4    25      50
## 5:  5    30      60
\end{verbatim}

Modificando uma Coluna Existente com \texttt{:=}

Podemos também modificar uma coluna existente. Vamos modificar a coluna valor\_2 para ser o triplo do valor da coluna valor:

\begin{Shaded}
\begin{Highlighting}[]
\NormalTok{DT[, valor\_2 }\SpecialCharTok{:=}\NormalTok{ valor }\SpecialCharTok{*} \DecValTok{3}\NormalTok{]}

\FunctionTok{print}\NormalTok{(DT)}
\end{Highlighting}
\end{Shaded}

\begin{verbatim}
##    id valor valor_2
## 1:  1    10      30
## 2:  2    15      45
## 3:  3    20      60
## 4:  4    25      75
## 5:  5    30      90
\end{verbatim}

\chapter{Dendrometria}\label{dendrometria}

\section{Diâmetros}\label{diuxe2metros}

\subsection{Histogramas de distribuição diamétrica}\label{histogramas-de-distribuiuxe7uxe3o-diamuxe9trica}

Este tipo de abordagem é útil para se verificar o grau de aproximação da Normalidade dos dados pela ``formato'' das classes através da frequência de indivíduos por classe diamétrica.

Os engenheiros florestais geralmente estabelecem classes diamétricas em intervalos fixos de 1,5 ou 2,0 cm para poderem comparar entre si as inúmeras parcelas; bem como para acompanhar a mudança de classe das árvores ao longo do tempo.

\subsubsection{Criando dados para demonstração}\label{criando-dados-para-demonstrauxe7uxe3o}

\begin{Shaded}
\begin{Highlighting}[]
\CommentTok{\# Definir número de árvores por parcela}
\NormalTok{n\_arvores }\OtherTok{\textless{}{-}} \DecValTok{50}

\CommentTok{\# Gerar dados de 3 parcelas com diâmetros distribuídos normalmente}
\FunctionTok{set.seed}\NormalTok{(}\DecValTok{123}\NormalTok{)  }\CommentTok{\# Para garantir reprodutibilidade}

\NormalTok{parcela1 }\OtherTok{\textless{}{-}} \FunctionTok{rnorm}\NormalTok{(n\_arvores, }\AttributeTok{mean =} \DecValTok{25}\NormalTok{, }\AttributeTok{sd =} \DecValTok{5}\NormalTok{)  }\CommentTok{\# Parcela 1 com média 25 cm e desvio padrão 5 cm}
\NormalTok{parcela2 }\OtherTok{\textless{}{-}} \FunctionTok{rnorm}\NormalTok{(n\_arvores, }\AttributeTok{mean =} \DecValTok{30}\NormalTok{, }\AttributeTok{sd =} \DecValTok{7}\NormalTok{)  }\CommentTok{\# Parcela 2 com média 30 cm e desvio padrão 7 cm}
\NormalTok{parcela3 }\OtherTok{\textless{}{-}} \FunctionTok{rnorm}\NormalTok{(n\_arvores, }\AttributeTok{mean =} \DecValTok{35}\NormalTok{, }\AttributeTok{sd =} \DecValTok{6}\NormalTok{)  }\CommentTok{\# Parcela 3 com média 35 cm e desvio padrão 6 cm}

\CommentTok{\# Criar um data frame com os dados das parcelas}
\NormalTok{dados\_inventario }\OtherTok{\textless{}{-}} \FunctionTok{data.frame}\NormalTok{(}
  \AttributeTok{Parcela =} \FunctionTok{rep}\NormalTok{(}\FunctionTok{c}\NormalTok{(}\StringTok{"1"}\NormalTok{, }\StringTok{"2"}\NormalTok{, }\StringTok{"3"}\NormalTok{), }\AttributeTok{each =}\NormalTok{ n\_arvores),}
  \AttributeTok{Diametro =} \FunctionTok{c}\NormalTok{(parcela1, parcela2, parcela3)}
\NormalTok{)}
\end{Highlighting}
\end{Shaded}

\subsubsection{Construindo os histogramas}\label{construindo-os-histogramas}

\begin{Shaded}
\begin{Highlighting}[]
\CommentTok{\# Definir as classes diamétricas com intervalo de 2 cm}
\NormalTok{intervalo }\OtherTok{\textless{}{-}} \DecValTok{2}
\NormalTok{min\_diametro }\OtherTok{\textless{}{-}} \FunctionTok{floor}\NormalTok{(}\FunctionTok{min}\NormalTok{(dados\_inventario}\SpecialCharTok{$}\NormalTok{Diametro))  }\CommentTok{\# Valor mínimo de diâmetro arredondado para baixo}
\NormalTok{max\_diametro }\OtherTok{\textless{}{-}} \FunctionTok{ceiling}\NormalTok{(}\FunctionTok{max}\NormalTok{(dados\_inventario}\SpecialCharTok{$}\NormalTok{Diametro))  }\CommentTok{\# Valor máximo de diâmetro arredondado para cima}
\NormalTok{intervalos }\OtherTok{\textless{}{-}} \FunctionTok{seq}\NormalTok{(min\_diametro }\SpecialCharTok{{-}}\NormalTok{ intervalo, max\_diametro }\SpecialCharTok{+}\NormalTok{ intervalo, }\AttributeTok{by =}\NormalTok{ intervalo)  }\CommentTok{\# Definir as classes com intervalo de 2 cm}

\CommentTok{\# Criar layout para os gráficos}
\FunctionTok{par}\NormalTok{(}\AttributeTok{mfrow =} \FunctionTok{c}\NormalTok{(}\DecValTok{1}\NormalTok{, }\DecValTok{3}\NormalTok{))  }\CommentTok{\# Define 3 gráficos em uma linha}

\CommentTok{\# Definir lista de parcelas}
\NormalTok{parcelas }\OtherTok{\textless{}{-}} \FunctionTok{unique}\NormalTok{(dados\_inventario}\SpecialCharTok{$}\NormalTok{Parcela)}

\CommentTok{\# Loop para plotar histogramas de cada parcela}
\ControlFlowTok{for}\NormalTok{ (parcela }\ControlFlowTok{in}\NormalTok{ parcelas) \{}
  \CommentTok{\# Selecionar os dados da parcela atual}
\NormalTok{  dados\_parcela }\OtherTok{\textless{}{-}}\NormalTok{ dados\_inventario[dados\_inventario}\SpecialCharTok{$}\NormalTok{Parcela }\SpecialCharTok{==}\NormalTok{ parcela, }\StringTok{"Diametro"}\NormalTok{]}
  \CommentTok{\# Plotar o histograma}
  \FunctionTok{hist}\NormalTok{(dados\_parcela,}
       \AttributeTok{breaks =}\NormalTok{ intervalos,}
       \AttributeTok{main =} \FunctionTok{paste}\NormalTok{(}\StringTok{"Histograma {-} parcela "}\NormalTok{, parcela),}
       \AttributeTok{xlab =} \StringTok{"Diâmetro (cm)"}\NormalTok{, }
       \AttributeTok{ylab =} \StringTok{"Frequência"}\NormalTok{,}
       \AttributeTok{ylim =} \FunctionTok{c}\NormalTok{(}\DecValTok{0}\NormalTok{, }\DecValTok{10}\NormalTok{),}
       \AttributeTok{col =} \StringTok{"skyblue"}\NormalTok{,}
       \AttributeTok{border =} \StringTok{"black"}\NormalTok{)}
\NormalTok{\}}
\end{Highlighting}
\end{Shaded}

\includegraphics{Entre_Florestas_e_Dados_files/figure-latex/unnamed-chunk-32-1.pdf}

\section{Exercício}\label{exercuxedcio}

Faça o mesmo gráfico com intervalos de 1,5 cm e mude a cor das barras para vermelho.

Ver solução

Solução do problema

\begin{Shaded}
\begin{Highlighting}[]
\FunctionTok{par}\NormalTok{(}\AttributeTok{mfrow =} \FunctionTok{c}\NormalTok{(}\DecValTok{1}\NormalTok{, }\DecValTok{3}\NormalTok{))  }\CommentTok{\# Define 3 gráficos em uma linha}

\CommentTok{\# Definir as classes diamétricas com intervalo de 2 cm}
\NormalTok{intervalo }\OtherTok{\textless{}{-}} \FloatTok{1.5}
\NormalTok{intervalos }\OtherTok{\textless{}{-}} \FunctionTok{seq}\NormalTok{(min\_diametro }\SpecialCharTok{{-}}\NormalTok{ intervalo, max\_diametro }\SpecialCharTok{+}\NormalTok{ intervalo, }\AttributeTok{by =}\NormalTok{ intervalo)}

\CommentTok{\# Loop para plotar histogramas de cada parcela}
\ControlFlowTok{for}\NormalTok{ (parcela }\ControlFlowTok{in}\NormalTok{ parcelas) \{}
  \CommentTok{\# Selecionar os dados da parcela atual}
\NormalTok{  dados\_parcela }\OtherTok{\textless{}{-}}\NormalTok{ dados\_inventario[dados\_inventario}\SpecialCharTok{$}\NormalTok{Parcela }\SpecialCharTok{==}\NormalTok{ parcela, }\StringTok{"Diametro"}\NormalTok{]}
  \CommentTok{\# Plotar o histograma}
  \FunctionTok{hist}\NormalTok{(dados\_parcela,}
      \AttributeTok{breaks =}\NormalTok{ intervalos,}
      \AttributeTok{main =} \FunctionTok{paste}\NormalTok{(}\StringTok{"Histograma {-} parcela "}\NormalTok{, parcela),}
      \AttributeTok{xlab =} \StringTok{"Diâmetro (cm)"}\NormalTok{, }
      \AttributeTok{ylab =} \StringTok{"Frequência"}\NormalTok{,}
      \AttributeTok{ylim =} \FunctionTok{c}\NormalTok{(}\DecValTok{0}\NormalTok{, }\DecValTok{10}\NormalTok{),}
      \AttributeTok{col =} \StringTok{"red"}\NormalTok{,}
      \AttributeTok{border =} \StringTok{"black"}\NormalTok{)}
\NormalTok{\}}
\end{Highlighting}
\end{Shaded}

\includegraphics{Entre_Florestas_e_Dados_files/figure-latex/unnamed-chunk-33-1.pdf}

\chapter{Inventário florestal}\label{inventuxe1rio-florestal}

\section{Fitossociologia (Teoria)}\label{fitossociologia-teoria}

Os engenheiros florestais realizam estudos de fitossociologia para compreender a composição, a estrutura e as interações das comunidades vegetais em um determinado ecossistema. Esses estudos permitem identificar espécies dominantes, associadas e raras, além de avaliar a biodiversidade e as dinâmicas ecológicas do ambiente. Essas informações são essenciais para planejar o manejo sustentável, recuperar áreas degradadas e conservar ecossistemas naturais.

Os engenheiros florestais realizam estudos de fitossociologia para determinar tipologias vegetais e fitofisionomias, aspectos fundamentais em processos como a Autorização para Exploração Vegetal (AUTEX) e os Programas de Recuperação Ambiental (PRA).

\subsection{Estrutura horizontal}\label{estrutura-horizontal}

\subsubsection{Densidade absoluta}\label{densidade-absoluta}

Refere-se ao número de indivíduos de uma espécie \(i\) (\(n_i\)) por unidade de área (\(A\)), expressando a abundância total da espécie na comunidade.

\[
DA_i = \frac{n_i}{A}
\]

\subsubsection{Densidade relativa}\label{densidade-relativa}

É a proporção da densidade absoluta de uma espécie (\(DA_i\)) em relação à soma das densidades absolutas de todas as espécies, expressa em porcentagem. Indica a importância relativa da espécie em termos de abundância.

\[
DR_i = \frac{DA_i}{\sum_{i=1}^s (DA_i)}
\]

\subsubsection{Dominância absoluta}\label{dominuxe2ncia-absoluta}

Calcula a área basal total (\(G_i\)) de todos os indivíduos de uma espécie por unidade de área (\(A\)). Mede o espaço físico ocupado por uma espécie no ambiente.

\[
DoA_i = \frac{\sum_{j=1}^{n_i} g_j}{A} = \frac{G_i}{A}
\]

\subsubsection{Dominância relativa}\label{dominuxe2ncia-relativa}

Representa a proporção da dominância absoluta de uma espécie (\(DoA_i\)) em relação à dominância total (\(G_T\)) de todas as espécies, expressa em porcentagem. Indica a importância relativa em termos de área ocupada.

\[
DoR_i = \frac{DoA_i}{\sum_{i=1}^{S} (DoA_i)}*100=\frac{G_i}{G_T}*100
\]

\subsubsection{Frequência Absoluta}\label{frequuxeancia-absoluta}

É a relação entre o número de unidades amostrais onde a espécie está presente (\(U_i\)) e o total de unidades amostrais (\(U_T\)), expressa em porcentagem. Mede a distribuição da espécie no espaço amostrado.

\[
FA_i = \frac{U_i}{U_T}*100
\]

\subsubsection{Frequência Relativa}\label{frequuxeancia-relativa}

Corresponde à frequência absoluta de uma espécie (\(FA_i\)) em relação à soma das frequências absolutas de todas as espécies, expressa em porcentagem. Avalia a importância espacial relativa da espécie.

\[
FR_i = \frac{FA_i}{\sum_{i=1}^{S}(FA_i)}*100
\]

\subsubsection{Valor de Cobertura}\label{valor-de-cobertura}

É a soma da densidade relativa (\(DR_i\)), da dominância relativa (\(DoR_i\)) e, em alguns casos, da frequência relativa (\(FR_i\)), dependendo da metodologia. Reflete a contribuição total de uma espécie na comunidade.
\[
VC_i = DR_i + DoR_i + FR_i
\]

\subsubsection{Porcentagem de Cobertura (Horizontal)}\label{porcentagem-de-cobertura-horizontal}

É a média ponderada de densidade relativa, dominância relativa e, opcionalmente, frequência relativa. A interpretação pode variar entre o cálculo bidimensional (\(DR + DoR\)) ou tridimensional (\(DR + DoR + FR\)), dependendo do objetivo.
\[
PC_i = \frac{DR_i + DoR_i + FR_i}{3}
\]

\subsection{Diversidade}\label{diversidade}

\subsubsection{Índice de Shannom}\label{uxedndice-de-shannom}

Quantifica a diversidade da comunidade, considerando tanto a abundância (\(p_i\)) quanto a equitabilidade das espécies. Valores maiores indicam maior diversidade e distribuição equitativa entre as espécies.

\[
H' = -\sum_{i=1}^S p_i \ln(p_i)
\]

\subsection{Agregação}\label{agregauxe7uxe3o}

\subsubsection{Índice de Morisita}\label{uxedndice-de-morisita}

Mede o padrão de distribuição espacial das espécies. Valores \(I_\delta > 1\) indicam agregação, \(I_\delta = 1\) distribuição ao acaso, e \(I_\delta < 1\) distribuição uniforme.

\[
I_\delta = \frac{n \sum_{i=1}^S n_i (n_i - 1)}{N (N - 1)}
\]

\subsection{Estrutura vertical}\label{estrutura-vertical}

\subsubsection{Posição Sociológica Absoluta}\label{posiuxe7uxe3o-socioluxf3gica-absoluta}

Representa a distribuição vertical de uma espécie na comunidade, considerando a contribuição proporcional das classes de altura.
\[
PSA_i = \sum_{j=1}^{J}\left(\frac{N_j}{N} \cdot N_{ij}\right)
\]

\subsubsection{Posição Sociológica Relativa}\label{posiuxe7uxe3o-socioluxf3gica-relativa}

É a proporção da posição sociológica absoluta de uma espécie (\(PSA_i\)) em relação à soma das posições sociológicas absolutas de todas as espécies, expressa em porcentagem. Indica a importância relativa da espécie na estrutura vertical.
\[
PSR_i = \frac{PSA_i}{\sum_{i=1}^{S} PSA_i} \cdot 100
\]

\subsubsection{Valor de Importância Absoluta (Horizontal + Vertical)}\label{valor-de-importuxe2ncia-absoluta-horizontal-vertical}

Combina os índices horizontais (densidade relativa, dominância relativa e frequência relativa) com a posição sociológica absoluta para refletir a contribuição global de uma espécie na comunidade.
\[
VI_a = DR_i + DoR_i + FR_i + PSA_i
\]

\subsubsection{Valor de Importância Relativa (Horizontal + Vertical)}\label{valor-de-importuxe2ncia-relativa-horizontal-vertical}

É a média ponderada dos índices horizontais e verticais. Indica a importância relativa de uma espécie considerando a estrutura horizontal e vertical.
\[
VI_r = \frac{DR_i + DoR_i + FR_i + PSA_i}{4}
\]

\subsection{Regeneração}\label{regenerauxe7uxe3o}

Pode-se usar os mesmos índices anteriores de estrutura horizontal, vertical, etc. O resultado final será o Indicador de Regeneração Natural que poderá ser usado para calcular o \textbf{Valor de Importância Ampliado}:
\[
VIA_r = \frac{DR_i + DoR_i + FR_i + PSA_i + RN_i}{5}
\]

\section{Fitossociologia (Aplicação)}\label{fitossociologia-aplicauxe7uxe3o}

De acordo com o estudo realizado por Verly et al.~(2020), a caracterização florística defina para a Reserva Legal do IFMT foi de um fragmento Cerradão em estado de conservação adequada.
De acordo com \citep{verly2020}, a caracterização florística\ldots{}

Dados exemplo:
\href{data/Dados_Reserva_Exemplo.xlsx}{Baixar dados}

\begin{landscape}\begin{table}
\centering
\resizebox{\ifdim\width>\linewidth\linewidth\else\width\fi}{!}{
\begin{tabular}{>{\raggedleft\arraybackslash}p{3cm}>{\raggedleft\arraybackslash}p{5cm}rllllll}
\toprule
Parc & Subparc & Id & Nome Científico & Família & H T & HC & CAP & DAP\\
\midrule
\cellcolor{gray!10}{1} & \cellcolor{gray!10}{1} & \cellcolor{gray!10}{1} & \cellcolor{gray!10}{Myracrodruon urundeuva Allemão} & \cellcolor{gray!10}{Anacardiaceae R.Br.} & \cellcolor{gray!10}{11.5} & \cellcolor{gray!10}{6} & \cellcolor{gray!10}{84.5} & \cellcolor{gray!10}{26.897185382530314}\\
1 & 1 & 2 & SP & - & - & - & - & -\\
\cellcolor{gray!10}{1} & \cellcolor{gray!10}{1} & \cellcolor{gray!10}{3} & \cellcolor{gray!10}{Tabebuia roseoalba (Ridl.) Sandwith} & \cellcolor{gray!10}{Bignoniaceae Juss.} & \cellcolor{gray!10}{10.5} & \cellcolor{gray!10}{5} & \cellcolor{gray!10}{54} & \cellcolor{gray!10}{17.188733853924695}\\
1 & 1 & 4 & Platypodium elegans Vogel & Fabaceae Lindl. & 11.5 & 8.5 & 57.6 & 18.334649444186343\\
\cellcolor{gray!10}{1} & \cellcolor{gray!10}{1} & \cellcolor{gray!10}{5} & \cellcolor{gray!10}{Callisthene fasciculata Mart.} & \cellcolor{gray!10}{Vochysiaceae A.St.-Hil.} & \cellcolor{gray!10}{8.5} & \cellcolor{gray!10}{4.5} & \cellcolor{gray!10}{37.700000000000003} & \cellcolor{gray!10}{12.000282709128909}\\
\addlinespace
1 & 1 & 6 & Magonia pubescens A.St.-Hil. & Sapindaceae Juss. & 11.5 & 8 & 94.7 & 30.143946221604978\\
\cellcolor{gray!10}{1} & \cellcolor{gray!10}{1} & \cellcolor{gray!10}{7} & \cellcolor{gray!10}{Morta} & \cellcolor{gray!10}{-} & \cellcolor{gray!10}{-} & \cellcolor{gray!10}{-} & \cellcolor{gray!10}{-} & \cellcolor{gray!10}{-}\\
1 & 1 & 8 & Pseudobombax tomentosum (Mart. \& Zucc.) A.Robyns & Malvaceae Juss. & 11 & 6 & 108.5 & 34.53662265094129\\
\cellcolor{gray!10}{1} & \cellcolor{gray!10}{1} & \cellcolor{gray!10}{9} & \cellcolor{gray!10}{Platypodium elegans Vogel} & \cellcolor{gray!10}{Fabaceae Lindl.} & \cellcolor{gray!10}{10.5} & \cellcolor{gray!10}{6.5} & \cellcolor{gray!10}{56} & \cellcolor{gray!10}{17.82535362629228}\\
1 & 1 & 10 & Aspidosperma cylindrocarpon Müll.Arg. & Apocynaceae Juss. & 8 & 6 & 36.4 & 11.58647985708998\\
\addlinespace
\cellcolor{gray!10}{1} & \cellcolor{gray!10}{1} & \cellcolor{gray!10}{11} & \cellcolor{gray!10}{Callisthene fasciculata Mart.} & \cellcolor{gray!10}{Vochysiaceae A.St.-Hil.} & \cellcolor{gray!10}{8.5} & \cellcolor{gray!10}{6} & \cellcolor{gray!10}{37.799999999999997} & \cellcolor{gray!10}{12.032113697747286}\\
1 & 1 & 12 & Tabebuia roseoalba (Ridl.) Sandwith & Bignoniaceae Juss. & 8 & 6 & 41.5 & 13.209860276627314\\
\cellcolor{gray!10}{1} & \cellcolor{gray!10}{1} & \cellcolor{gray!10}{13} & \cellcolor{gray!10}{Callisthene fasciculata Mart.} & \cellcolor{gray!10}{Vochysiaceae A.St.-Hil.} & \cellcolor{gray!10}{10} & \cellcolor{gray!10}{7} & \cellcolor{gray!10}{44} & \cellcolor{gray!10}{14.00563499208679}\\
1 & 1 & 14 & Aspidosperma cylindrocarpon Müll.Arg. & Apocynaceae Juss. & 7.5 & 4 & 31.6 & 10.058592403407786\\
\cellcolor{gray!10}{1} & \cellcolor{gray!10}{1} & \cellcolor{gray!10}{15} & \cellcolor{gray!10}{Aspidosperma subincanum Mart.} & \cellcolor{gray!10}{Apocynaceae Juss.} & \cellcolor{gray!10}{12} & \cellcolor{gray!10}{4} & \cellcolor{gray!10}{79} & \cellcolor{gray!10}{25.146481008519466}\\
\addlinespace
1 & 1 & 16 & Aspidosperma cylindrocarpon Müll.Arg. & Apocynaceae Juss. & 12 & 5.5 & 55.6 & 17.698029671818762\\
\cellcolor{gray!10}{1} & \cellcolor{gray!10}{1} & \cellcolor{gray!10}{17} & \cellcolor{gray!10}{Platypodium elegans Vogel} & \cellcolor{gray!10}{Fabaceae Lindl.} & \cellcolor{gray!10}{11} & \cellcolor{gray!10}{8.5} & \cellcolor{gray!10}{32.5} & \cellcolor{gray!10}{10.345071300973197}\\
1 & 1 & 18 & Anadenanthera colubrina (Vell.) Brenan & Fabaceae Lindl. & 13 & 9.5 & 58.8 & 18.716621307606893\\
\cellcolor{gray!10}{1} & \cellcolor{gray!10}{1} & \cellcolor{gray!10}{19} & \cellcolor{gray!10}{Aspidosperma cylindrocarpon Müll.Arg.} & \cellcolor{gray!10}{Apocynaceae Juss.} & \cellcolor{gray!10}{15} & \cellcolor{gray!10}{9} & \cellcolor{gray!10}{106} & \cellcolor{gray!10}{33.74084793548181}\\
1 & 1 & 20 & Platypodium elegans Vogel & Fabaceae Lindl. & 7.5 & 2.5 & 32 & 10.185916357881302\\
\addlinespace
\cellcolor{gray!10}{1} & \cellcolor{gray!10}{1} & \cellcolor{gray!10}{21} & \cellcolor{gray!10}{Astronium fraxinifolium Schott} & \cellcolor{gray!10}{Anacardiaceae R.Br.} & \cellcolor{gray!10}{9.5} & \cellcolor{gray!10}{7.5} & \cellcolor{gray!10}{37.299999999999997} & \cellcolor{gray!10}{11.872958754655391}\\
1 & 1 & 22 & Simarouba Aubl. & Simaroubaceae DC. & 9.5 & 7 & 43.5 & 13.846480048994895\\
\cellcolor{gray!10}{1} & \cellcolor{gray!10}{1} & \cellcolor{gray!10}{23} & \cellcolor{gray!10}{Morta} & \cellcolor{gray!10}{-} & \cellcolor{gray!10}{-} & \cellcolor{gray!10}{-} & \cellcolor{gray!10}{-} & \cellcolor{gray!10}{-}\\
1 & 1 & 24 & Astronium fraxinifolium Schott & Anacardiaceae R.Br. & 10 & 9 & 42.6 & 13.560001151429484\\
\cellcolor{gray!10}{1} & \cellcolor{gray!10}{1} & \cellcolor{gray!10}{25} & \cellcolor{gray!10}{Platypodium elegans Vogel} & \cellcolor{gray!10}{Fabaceae Lindl.} & \cellcolor{gray!10}{11.5} & \cellcolor{gray!10}{5.5} & \cellcolor{gray!10}{50} & \cellcolor{gray!10}{15.915494309189533}\\
\addlinespace
1 & 1 & 26 & Callisthene fasciculata Mart. & Vochysiaceae A.St.-Hil. & 11 & 5.5 & 88.7 & 28.234086904502234\\
\cellcolor{gray!10}{1} & \cellcolor{gray!10}{1} & \cellcolor{gray!10}{27} & \cellcolor{gray!10}{Handroanthus impetiginosus (Mart. ex DC.) Mattos} & \cellcolor{gray!10}{Bignoniaceae Juss.} & \cellcolor{gray!10}{11} & \cellcolor{gray!10}{6} & \cellcolor{gray!10}{70.400000000000006} & \cellcolor{gray!10}{22.409015987338865}\\
1 & 1 & 28 & Platypodium elegans Vogel & Fabaceae Lindl. & 11.5 & 8 & 52.5 & 16.71126902464901\\
\cellcolor{gray!10}{1} & \cellcolor{gray!10}{1} & \cellcolor{gray!10}{29} & \cellcolor{gray!10}{Machaerium acutifolium Vogel} & \cellcolor{gray!10}{Fabaceae Lindl.} & \cellcolor{gray!10}{10.5} & \cellcolor{gray!10}{6.5} & \cellcolor{gray!10}{38.4} & \cellcolor{gray!10}{12.223099629457561}\\
1 & 1 & 30 & Platypodium elegans Vogel & Fabaceae Lindl. & 10 & 7 & 31.5 & 10.026761414789407\\
\addlinespace
\cellcolor{gray!10}{1} & \cellcolor{gray!10}{1} & \cellcolor{gray!10}{31} & \cellcolor{gray!10}{Handroanthus impetiginosus (Mart. ex DC.) Mattos} & \cellcolor{gray!10}{Bignoniaceae Juss.} & \cellcolor{gray!10}{11} & \cellcolor{gray!10}{8} & \cellcolor{gray!10}{77.5} & \cellcolor{gray!10}{24.669016179243776}\\
1 & 1 & 32 & Senegalia polyphylla (DC.) Britton \& Rose & Fabaceae Lindl. & 11.5 & 9 & 33.5 & 10.663381187156988\\
\cellcolor{gray!10}{1} & \cellcolor{gray!10}{1} & \cellcolor{gray!10}{33} & \cellcolor{gray!10}{Aspidosperma cylindrocarpon Müll.Arg.} & \cellcolor{gray!10}{Apocynaceae Juss.} & \cellcolor{gray!10}{9} & \cellcolor{gray!10}{5.5} & \cellcolor{gray!10}{38.700000000000003} & \cellcolor{gray!10}{12.318592595312701}\\
1 & 1 & 34 & Platypodium elegans Vogel & Fabaceae Lindl. & 9 & 7.5 & 34.5 & 10.981691073340778\\
\cellcolor{gray!10}{1} & \cellcolor{gray!10}{1} & \cellcolor{gray!10}{35} & \cellcolor{gray!10}{Callisthene fasciculata Mart.} & \cellcolor{gray!10}{Vochysiaceae A.St.-Hil.} & \cellcolor{gray!10}{9.5} & \cellcolor{gray!10}{6} & \cellcolor{gray!10}{45.4} & \cellcolor{gray!10}{14.451268832744097}\\
\addlinespace
1 & 1 & 36 & Qualea Aubl. & Vochysiaceae A.St.-Hil. & 10.5 & 6 & 66.3 & 21.103945453985322\\
\cellcolor{gray!10}{1} & \cellcolor{gray!10}{1} & \cellcolor{gray!10}{37} & \cellcolor{gray!10}{Qualea Aubl.} & \cellcolor{gray!10}{Vochysiaceae A.St.-Hil.} & \cellcolor{gray!10}{10} & \cellcolor{gray!10}{6} & \cellcolor{gray!10}{55.4} & \cellcolor{gray!10}{17.634367694582004}\\
1 & 1 & 38 & Callisthene fasciculata Mart. & Vochysiaceae A.St.-Hil. & 10.5 & 4.5 & 55.5 & 17.666198683200381\\
\cellcolor{gray!10}{1} & \cellcolor{gray!10}{1} & \cellcolor{gray!10}{39} & \cellcolor{gray!10}{Callisthene fasciculata Mart.} & \cellcolor{gray!10}{Vochysiaceae A.St.-Hil.} & \cellcolor{gray!10}{12} & \cellcolor{gray!10}{7} & \cellcolor{gray!10}{79} & \cellcolor{gray!10}{25.146481008519466}\\
1 & 1 & 40 & Myracrodruon urundeuva Allemão & Anacardiaceae R.Br. & 13 & 6.5 & 106.7 & 33.963664855810464\\
\addlinespace
\cellcolor{gray!10}{1} & \cellcolor{gray!10}{1} & \cellcolor{gray!10}{41} & \cellcolor{gray!10}{Dipteryx alata Vogel} & \cellcolor{gray!10}{Fabaceae Lindl.} & \cellcolor{gray!10}{13.5} & \cellcolor{gray!10}{3.5} & \cellcolor{gray!10}{96.6} & \cellcolor{gray!10}{30.748735005354177}\\
1 & 1 & 42 & Platypodium elegans Vogel & Fabaceae Lindl. & 12 & 8.5 & 92.5 & 29.443664472000638\\
\cellcolor{gray!10}{1} & \cellcolor{gray!10}{1} & \cellcolor{gray!10}{43} & \cellcolor{gray!10}{Callisthene fasciculata Mart.} & \cellcolor{gray!10}{Vochysiaceae A.St.-Hil.} & \cellcolor{gray!10}{11} & \cellcolor{gray!10}{7} & \cellcolor{gray!10}{62} & \cellcolor{gray!10}{19.735212943395023}\\
1 & 1 & 44 & Callisthene fasciculata Mart. & Vochysiaceae A.St.-Hil. & 10 & 7 & 42.5 & 13.528170162811104\\
\cellcolor{gray!10}{1} & \cellcolor{gray!10}{1} & \cellcolor{gray!10}{45} & \cellcolor{gray!10}{Aspidosperma subincanum Mart.} & \cellcolor{gray!10}{Apocynaceae Juss.} & \cellcolor{gray!10}{13.5} & \cellcolor{gray!10}{8.5} & \cellcolor{gray!10}{70.099999999999994} & \cellcolor{gray!10}{22.313523021483725}\\
\addlinespace
1 & 1 & 46 & Callisthene fasciculata Mart. & Vochysiaceae A.St.-Hil. & 11.5 & 7 & 65.5 & 20.84929754503829\\
\cellcolor{gray!10}{1} & \cellcolor{gray!10}{1} & \cellcolor{gray!10}{47} & \cellcolor{gray!10}{Anadenanthera colubrina (Vell.) Brenan} & \cellcolor{gray!10}{Fabaceae Lindl.} & \cellcolor{gray!10}{13.5} & \cellcolor{gray!10}{10} & \cellcolor{gray!10}{62.6} & \cellcolor{gray!10}{19.926198875105296}\\
1 & 2 & 91 & Qualea Aubl. & Vochysiaceae A.St.-Hil. & 10.5 & 7.5 & 61.1 & 19.44873404582961\\
\cellcolor{gray!10}{1} & \cellcolor{gray!10}{2} & \cellcolor{gray!10}{90} & \cellcolor{gray!10}{Platypodium elegans Vogel} & \cellcolor{gray!10}{Fabaceae Lindl.} & \cellcolor{gray!10}{11} & \cellcolor{gray!10}{7.5} & \cellcolor{gray!10}{37.5} & \cellcolor{gray!10}{11.93662073189215}\\
1 & 2 & 89 & Dipteryx alata Vogel & Fabaceae Lindl. & 12.5 & 8 & 73.400000000000006 & 23.363945645890237\\
\addlinespace
\cellcolor{gray!10}{1} & \cellcolor{gray!10}{2} & \cellcolor{gray!10}{88} & \cellcolor{gray!10}{Platypodium elegans Vogel} & \cellcolor{gray!10}{Fabaceae Lindl.} & \cellcolor{gray!10}{9} & \cellcolor{gray!10}{7} & \cellcolor{gray!10}{32.799999999999997} & \cellcolor{gray!10}{10.440564266828334}\\
1 & 2 & 87 & Qualea Aubl. & Vochysiaceae A.St.-Hil. & 10 & 7.5 & 40 & 12.732395447351628\\
\cellcolor{gray!10}{1} & \cellcolor{gray!10}{2} & \cellcolor{gray!10}{86} & \cellcolor{gray!10}{Dipteryx alata Vogel} & \cellcolor{gray!10}{Fabaceae Lindl.} & \cellcolor{gray!10}{13} & \cellcolor{gray!10}{6} & \cellcolor{gray!10}{118.1} & \cellcolor{gray!10}{37.592397558305677}\\
1 & 2 & 85 & Aspidosperma cylindrocarpon Müll.Arg. & Apocynaceae Juss. & 7 & 4.5 & 38.1 & 12.127606663602425\\
\cellcolor{gray!10}{1} & \cellcolor{gray!10}{2} & \cellcolor{gray!10}{84} & \cellcolor{gray!10}{Morta} & \cellcolor{gray!10}{-} & \cellcolor{gray!10}{-} & \cellcolor{gray!10}{-} & \cellcolor{gray!10}{-} & \cellcolor{gray!10}{-}\\
\addlinespace
1 & 2 & 83 & Simarouba Aubl. & Simaroubaceae DC. & 8.5 & 7 & 31.4 & 9.9949304261710274\\
\cellcolor{gray!10}{1} & \cellcolor{gray!10}{2} & \cellcolor{gray!10}{82} & \cellcolor{gray!10}{Senegalia polyphylla (DC.) Britton \& Rose} & \cellcolor{gray!10}{Fabaceae Lindl.} & \cellcolor{gray!10}{10.5} & \cellcolor{gray!10}{6.5} & \cellcolor{gray!10}{36.700000000000003} & \cellcolor{gray!10}{11.681972822945118}\\
1 & 2 & 81 & Guapira Aubl. & Nyctaginaceae Juss. & 10 & 7 & 44.6 & 14.196620923797065\\
\cellcolor{gray!10}{1} & \cellcolor{gray!10}{2} & \cellcolor{gray!10}{80} & \cellcolor{gray!10}{Callisthene fasciculata Mart.} & \cellcolor{gray!10}{Vochysiaceae A.St.-Hil.} & \cellcolor{gray!10}{11.5} & \cellcolor{gray!10}{7} & \cellcolor{gray!10}{66.099999999999994} & \cellcolor{gray!10}{21.040283476748563}\\
1 & 2 & 79 & Peltogyne confertiflora (Mart. ex Hayne) Benth. & Fabaceae Lindl. & 10 & 7.5 & 46.2 & 14.70591674169113\\
\addlinespace
\cellcolor{gray!10}{1} & \cellcolor{gray!10}{2} & \cellcolor{gray!10}{78} & \cellcolor{gray!10}{Aspidosperma cylindrocarpon Müll.Arg.} & \cellcolor{gray!10}{Apocynaceae Juss.} & \cellcolor{gray!10}{11} & \cellcolor{gray!10}{7} & \cellcolor{gray!10}{44.3} & \cellcolor{gray!10}{14.101127957941927}\\
1 & 2 & 77 & Magonia pubescens A.St.-Hil. & Sapindaceae Juss. & 10 & 4.5 & 64.099999999999994 & 20.403663704380982\\
\cellcolor{gray!10}{1} & \cellcolor{gray!10}{2} & \cellcolor{gray!10}{76} & \cellcolor{gray!10}{Aspidosperma subincanum Mart.} & \cellcolor{gray!10}{Apocynaceae Juss.} & \cellcolor{gray!10}{10.5} & \cellcolor{gray!10}{8.5} & \cellcolor{gray!10}{46.7} & \cellcolor{gray!10}{14.865071684783025}\\
1 & 2 & 75 & Morfoespécie 3 & NA & 10 & 5 & 32.700000000000003 & 10.408733278209956\\
\cellcolor{gray!10}{1} & \cellcolor{gray!10}{2} & \cellcolor{gray!10}{74} & \cellcolor{gray!10}{Aspidosperma cylindrocarpon Müll.Arg.} & \cellcolor{gray!10}{Apocynaceae Juss.} & \cellcolor{gray!10}{10} & \cellcolor{gray!10}{7} & \cellcolor{gray!10}{37.9} & \cellcolor{gray!10}{12.063944686365666}\\
\addlinespace
1 & 2 & 73 & Myrciaria tenella (DC.) O.Berg & Myrtaceae Juss. & 8.5 & 4 & 33.6 & 10.695212175775367\\
\cellcolor{gray!10}{1} & \cellcolor{gray!10}{2} & \cellcolor{gray!10}{72} & \cellcolor{gray!10}{Roupala montana Aubl.} & \cellcolor{gray!10}{Proteaceae Juss.} & \cellcolor{gray!10}{7} & \cellcolor{gray!10}{4.5} & \cellcolor{gray!10}{32.4} & \cellcolor{gray!10}{10.313240312354818}\\
1 & 2 & 71 & Platypodium elegans Vogel & Fabaceae Lindl. & 7.5 & 6 & 34 & 10.822536130248883\\
\cellcolor{gray!10}{1} & \cellcolor{gray!10}{2} & \cellcolor{gray!10}{70} & \cellcolor{gray!10}{Qualea Aubl.} & \cellcolor{gray!10}{Vochysiaceae A.St.-Hil.} & \cellcolor{gray!10}{10} & \cellcolor{gray!10}{4.5} & \cellcolor{gray!10}{54.8} & \cellcolor{gray!10}{17.443381762871727}\\
1 & 2 & 69 & Callisthene fasciculata Mart. & Vochysiaceae A.St.-Hil. & 10 & 5.5 & 64.900000000000006 & 20.658311613328017\\
\addlinespace
\cellcolor{gray!10}{1} & \cellcolor{gray!10}{2} & \cellcolor{gray!10}{68} & \cellcolor{gray!10}{Platypodium elegans Vogel} & \cellcolor{gray!10}{Fabaceae Lindl.} & \cellcolor{gray!10}{8} & \cellcolor{gray!10}{4} & \cellcolor{gray!10}{35.9} & \cellcolor{gray!10}{11.427324913998085}\\
1 & 2 & 67 & Myracrodruon urundeuva Allemão & Anacardiaceae R.Br. & 12.5 & 6 & 101.1 & 32.181129493181238\\
\cellcolor{gray!10}{1} & \cellcolor{gray!10}{2} & \cellcolor{gray!10}{66} & \cellcolor{gray!10}{Aspidosperma cylindrocarpon Müll.Arg.} & \cellcolor{gray!10}{Apocynaceae Juss.} & \cellcolor{gray!10}{10.5} & \cellcolor{gray!10}{6.5} & \cellcolor{gray!10}{50} & \cellcolor{gray!10}{15.915494309189533}\\
1 & 2 & 65 & Astronium fraxinifolium Schott & Anacardiaceae R.Br. & 12.5 & 6 & 54.6 & 17.379719785634972\\
\cellcolor{gray!10}{1} & \cellcolor{gray!10}{2} & \cellcolor{gray!10}{64} & \cellcolor{gray!10}{Myracrodruon urundeuva Allemão} & \cellcolor{gray!10}{Anacardiaceae R.Br.} & \cellcolor{gray!10}{8} & \cellcolor{gray!10}{4} & \cellcolor{gray!10}{53.7} & \cellcolor{gray!10}{17.093240888069559}\\
\addlinespace
1 & 2 & 63 & Platypodium elegans Vogel & Fabaceae Lindl. & 11 & 7 & 48.6 & 15.469860468532227\\
\cellcolor{gray!10}{1} & \cellcolor{gray!10}{2} & \cellcolor{gray!10}{62} & \cellcolor{gray!10}{Diptychandra aurantiaca Tul.} & \cellcolor{gray!10}{Fabaceae Lindl.} & \cellcolor{gray!10}{12} & \cellcolor{gray!10}{6} & \cellcolor{gray!10}{60.5} & \cellcolor{gray!10}{19.257748114119337}\\
1 & 2 & 61 & Platypodium elegans Vogel & Fabaceae Lindl. & 10 & 6 & 37.5 & 11.93662073189215\\
\cellcolor{gray!10}{1} & \cellcolor{gray!10}{2} & \cellcolor{gray!10}{60} & \cellcolor{gray!10}{Anadenanthera colubrina (Vell.) Brenan} & \cellcolor{gray!10}{Fabaceae Lindl.} & \cellcolor{gray!10}{11} & \cellcolor{gray!10}{7.5} & \cellcolor{gray!10}{56.8} & \cellcolor{gray!10}{18.080001535239308}\\
1 & 2 & 59 & Platypodium elegans Vogel & Fabaceae Lindl. & 11 & 7 & 56 & 17.82535362629228\\
\addlinespace
\cellcolor{gray!10}{1} & \cellcolor{gray!10}{2} & \cellcolor{gray!10}{58} & \cellcolor{gray!10}{Astronium fraxinifolium Schott} & \cellcolor{gray!10}{Anacardiaceae R.Br.} & \cellcolor{gray!10}{15} & \cellcolor{gray!10}{9.5} & \cellcolor{gray!10}{110} & \cellcolor{gray!10}{35.014087480216972}\\
1 & 2 & 57 & Eugenia dysenterica (Mart.) DC. & Myrtaceae Juss. & 11 & 8 & 80 & 25.464790894703256\\
\cellcolor{gray!10}{1} & \cellcolor{gray!10}{2} & \cellcolor{gray!10}{56} & \cellcolor{gray!10}{Platypodium elegans Vogel} & \cellcolor{gray!10}{Fabaceae Lindl.} & \cellcolor{gray!10}{10.5} & \cellcolor{gray!10}{7.5} & \cellcolor{gray!10}{33.5} & \cellcolor{gray!10}{10.663381187156988}\\
1 & 2 & 55 & Tabebuia roseoalba (Ridl.) Sandwith & Bignoniaceae Juss. & 9.5 & 7.5 & 38.5 & 12.254930618075941\\
\cellcolor{gray!10}{1} & \cellcolor{gray!10}{2} & \cellcolor{gray!10}{54} & \cellcolor{gray!10}{Aspidosperma cylindrocarpon Müll.Arg.} & \cellcolor{gray!10}{Apocynaceae Juss.} & \cellcolor{gray!10}{9.5} & \cellcolor{gray!10}{7.5} & \cellcolor{gray!10}{33.9} & \cellcolor{gray!10}{10.790705141630504}\\
\addlinespace
1 & 2 & 53 & Platypodium elegans Vogel & Fabaceae Lindl. & 10.5 & 8.5 & 36 & 11.459155902616464\\
\cellcolor{gray!10}{1} & \cellcolor{gray!10}{2} & \cellcolor{gray!10}{52} & \cellcolor{gray!10}{Morta} & \cellcolor{gray!10}{-} & \cellcolor{gray!10}{-} & \cellcolor{gray!10}{-} & \cellcolor{gray!10}{-} & \cellcolor{gray!10}{-}\\
1 & 2 & 51 & Aspidosperma cylindrocarpon Müll.Arg. & Apocynaceae Juss. & 7 & 2.5 & 32.9 & 10.472395255446713\\
\cellcolor{gray!10}{1} & \cellcolor{gray!10}{2} & \cellcolor{gray!10}{50} & \cellcolor{gray!10}{Astronium fraxinifolium Schott} & \cellcolor{gray!10}{Anacardiaceae R.Br.} & \cellcolor{gray!10}{11} & \cellcolor{gray!10}{3} & \cellcolor{gray!10}{82.5} & \cellcolor{gray!10}{26.260565610162732}\\
1 & 2 & 49 & Qualea Aubl. & Vochysiaceae A.St.-Hil. & 9.5 & 7 & 54.4 & 17.316057808398213\\
\addlinespace
\cellcolor{gray!10}{1} & \cellcolor{gray!10}{2} & \cellcolor{gray!10}{48} & \cellcolor{gray!10}{Simarouba Aubl.} & \cellcolor{gray!10}{Simaroubaceae DC.} & \cellcolor{gray!10}{8.5} & \cellcolor{gray!10}{7} & \cellcolor{gray!10}{34.1} & \cellcolor{gray!10}{10.854367118867263}\\
1 & 3 & 92 & Guapira Aubl. & Nyctaginaceae Juss. & 7 & 5 & 37 & 11.777465788800255\\
\cellcolor{gray!10}{1} & \cellcolor{gray!10}{3} & \cellcolor{gray!10}{93} & \cellcolor{gray!10}{Vatairea macrocarpa (Benth.) Ducke} & \cellcolor{gray!10}{Fabaceae Lindl.} & \cellcolor{gray!10}{12} & \cellcolor{gray!10}{7} & \cellcolor{gray!10}{72.5} & \cellcolor{gray!10}{23.077466748324824}\\
1 & 3 & 94 & PP & - & - & - & - & -\\
\cellcolor{gray!10}{1} & \cellcolor{gray!10}{3} & \cellcolor{gray!10}{95} & \cellcolor{gray!10}{PP} & \cellcolor{gray!10}{-} & \cellcolor{gray!10}{-} & \cellcolor{gray!10}{-} & \cellcolor{gray!10}{-} & \cellcolor{gray!10}{-}\\
\addlinespace
1 & 3 & 96 & PP & - & - & - & - & -\\
\cellcolor{gray!10}{1} & \cellcolor{gray!10}{3} & \cellcolor{gray!10}{97} & \cellcolor{gray!10}{Platypodium elegans Vogel} & \cellcolor{gray!10}{Fabaceae Lindl.} & \cellcolor{gray!10}{11.5} & \cellcolor{gray!10}{7.5} & \cellcolor{gray!10}{54} & \cellcolor{gray!10}{17.188733853924695}\\
1 & 3 & 98 & Myracrodruon urundeuva Allemão & Anacardiaceae R.Br. & 9 & 4.5 & 36.5 & 11.618310845708359\\
\cellcolor{gray!10}{1} & \cellcolor{gray!10}{3} & \cellcolor{gray!10}{99} & \cellcolor{gray!10}{PP} & \cellcolor{gray!10}{-} & \cellcolor{gray!10}{-} & \cellcolor{gray!10}{-} & \cellcolor{gray!10}{-} & \cellcolor{gray!10}{-}\\
1 & 3 & 100 & Platypodium elegans Vogel & Fabaceae Lindl. & 9 & 7 & 38.5 & 12.254930618075941\\
\addlinespace
\cellcolor{gray!10}{1} & \cellcolor{gray!10}{3} & \cellcolor{gray!10}{101} & \cellcolor{gray!10}{Platypodium elegans Vogel} & \cellcolor{gray!10}{Fabaceae Lindl.} & \cellcolor{gray!10}{11} & \cellcolor{gray!10}{8} & \cellcolor{gray!10}{50.5} & \cellcolor{gray!10}{16.074649252281429}\\
1 & 3 & 102 & Curatella americana L. & Dilleniaceae  Salisb. & 13 & 5 & 108.6 & 34.568453639559664\\
\cellcolor{gray!10}{1} & \cellcolor{gray!10}{3} & \cellcolor{gray!10}{103} & \cellcolor{gray!10}{Astronium fraxinifolium Schott} & \cellcolor{gray!10}{Anacardiaceae R.Br.} & \cellcolor{gray!10}{13} & \cellcolor{gray!10}{7} & \cellcolor{gray!10}{59} & \cellcolor{gray!10}{18.780283284843652}\\
1 & 3 & 104 & Protium heptaphyllum (Aubl.) Marchand & Burseraceae Kunth & 11 & 5.5 & 39.299999999999997 & 12.509578527022972\\
\cellcolor{gray!10}{1} & \cellcolor{gray!10}{3} & \cellcolor{gray!10}{105} & \cellcolor{gray!10}{Tabebuia roseoalba (Ridl.) Sandwith} & \cellcolor{gray!10}{Bignoniaceae Juss.} & \cellcolor{gray!10}{12} & \cellcolor{gray!10}{7.5} & \cellcolor{gray!10}{40} & \cellcolor{gray!10}{12.732395447351628}\\
\addlinespace
1 & 3 & 106 & Tabebuia aurea (Silva Manso) Benth. \& Hook.f. ex S.Moore & Bignoniaceae Juss. & 10 & 4 & 61 & 19.416903057211233\\
\cellcolor{gray!10}{1} & \cellcolor{gray!10}{3} & \cellcolor{gray!10}{107} & \cellcolor{gray!10}{Aspidosperma cylindrocarpon Müll.Arg.} & \cellcolor{gray!10}{Apocynaceae Juss.} & \cellcolor{gray!10}{10} & \cellcolor{gray!10}{4} & \cellcolor{gray!10}{51.5} & \cellcolor{gray!10}{16.392959138465219}\\
1 & 3 & 108 & Callisthene fasciculata Mart. & Vochysiaceae A.St.-Hil. & 11 & 4.5 & 75.5 & 24.032396406876195\\
\cellcolor{gray!10}{1} & \cellcolor{gray!10}{3} & \cellcolor{gray!10}{109} & \cellcolor{gray!10}{Vatairea macrocarpa (Benth.) Ducke} & \cellcolor{gray!10}{Fabaceae Lindl.} & \cellcolor{gray!10}{11.5} & \cellcolor{gray!10}{8.5} & \cellcolor{gray!10}{50} & \cellcolor{gray!10}{15.915494309189533}\\
1 & 3 & 110 & Aspidosperma cylindrocarpon Müll.Arg. & Apocynaceae Juss. & 9 & 4 & 31.4 & 9.9949304261710274\\
\addlinespace
\cellcolor{gray!10}{1} & \cellcolor{gray!10}{3} & \cellcolor{gray!10}{111} & \cellcolor{gray!10}{Qualea Aubl.} & \cellcolor{gray!10}{Vochysiaceae A.St.-Hil.} & \cellcolor{gray!10}{11} & \cellcolor{gray!10}{9} & \cellcolor{gray!10}{41.4} & \cellcolor{gray!10}{13.178029288008934}\\
1 & 3 & 112 & Dipteryx alata Vogel & Fabaceae Lindl. & 11 & 7 & 67.5 & 21.485917317405871\\
\cellcolor{gray!10}{1} & \cellcolor{gray!10}{3} & \cellcolor{gray!10}{113} & \cellcolor{gray!10}{Aspidosperma cylindrocarpon Müll.Arg.} & \cellcolor{gray!10}{Apocynaceae Juss.} & \cellcolor{gray!10}{10.5} & \cellcolor{gray!10}{8} & \cellcolor{gray!10}{36} & \cellcolor{gray!10}{11.459155902616464}\\
1 & 3 & 114 & Aspidosperma cylindrocarpon Müll.Arg. & Apocynaceae Juss. & 11.5 & 8 & 55.2 & 17.570705717345248\\
\cellcolor{gray!10}{1} & \cellcolor{gray!10}{3} & \cellcolor{gray!10}{115} & \cellcolor{gray!10}{PP} & \cellcolor{gray!10}{-} & \cellcolor{gray!10}{-} & \cellcolor{gray!10}{-} & \cellcolor{gray!10}{-} & \cellcolor{gray!10}{-}\\
\addlinespace
1 & 3 & 116 & Eugenia dysenterica (Mart.) DC. & Myrtaceae Juss. & 8 & 6 & 35.4 & 11.26816997090619\\
\cellcolor{gray!10}{1} & \cellcolor{gray!10}{3} & \cellcolor{gray!10}{117} & \cellcolor{gray!10}{Bowdichia virgilioides Kunth} & \cellcolor{gray!10}{Fabaceae Lindl.} & \cellcolor{gray!10}{12} & \cellcolor{gray!10}{9} & \cellcolor{gray!10}{45.2} & \cellcolor{gray!10}{14.38760685550734}\\
1 & 3 & 118 & Eugenia dysenterica (Mart.) DC. & Myrtaceae Juss. & 7 & 4 & 31.5 & 10.026761414789407\\
\cellcolor{gray!10}{1} & \cellcolor{gray!10}{3} & \cellcolor{gray!10}{119} & \cellcolor{gray!10}{Morta} & \cellcolor{gray!10}{-} & \cellcolor{gray!10}{-} & \cellcolor{gray!10}{-} & \cellcolor{gray!10}{-} & \cellcolor{gray!10}{-}\\
1 & 3 & 120 & PP & - & - & - & - & -\\
\addlinespace
\cellcolor{gray!10}{1} & \cellcolor{gray!10}{3} & \cellcolor{gray!10}{121} & \cellcolor{gray!10}{Tabebuia roseoalba (Ridl.) Sandwith} & \cellcolor{gray!10}{Bignoniaceae Juss.} & \cellcolor{gray!10}{8} & \cellcolor{gray!10}{3} & \cellcolor{gray!10}{38.5} & \cellcolor{gray!10}{12.254930618075941}\\
1 & 3 & 122 & Morta & - & - & - & NA & -\\
\cellcolor{gray!10}{1} & \cellcolor{gray!10}{3} & \cellcolor{gray!10}{123} & \cellcolor{gray!10}{Magonia pubescens A.St.-Hil.} & \cellcolor{gray!10}{Sapindaceae Juss.} & \cellcolor{gray!10}{13} & \cellcolor{gray!10}{2} & \cellcolor{gray!10}{55.5} & \cellcolor{gray!10}{17.666198683200381}\\
1 & 3 & 124 & Magonia pubescens A.St.-Hil. & Sapindaceae Juss. & 12.5 & 8 & 48 & 15.278874536821952\\
\cellcolor{gray!10}{1} & \cellcolor{gray!10}{3} & \cellcolor{gray!10}{125} & \cellcolor{gray!10}{Pseudobombax tomentosum (Mart. \& Zucc.) A.Robyns} & \cellcolor{gray!10}{Malvaceae Juss.} & \cellcolor{gray!10}{11} & \cellcolor{gray!10}{3.5} & \cellcolor{gray!10}{123} & \cellcolor{gray!10}{39.152116000606256}\\
\addlinespace
1 & 3 & 126 & Callisthene fasciculata Mart. & Vochysiaceae A.St.-Hil. & 10.5 & 7.5 & 34.9 & 11.109015027814294\\
\cellcolor{gray!10}{1} & \cellcolor{gray!10}{3} & \cellcolor{gray!10}{127} & \cellcolor{gray!10}{Aspidosperma cylindrocarpon Müll.Arg.} & \cellcolor{gray!10}{Apocynaceae Juss.} & \cellcolor{gray!10}{10} & \cellcolor{gray!10}{3} & \cellcolor{gray!10}{53} & \cellcolor{gray!10}{16.870423967740905}\\
1 & 3 & 128 & Aspidosperma cylindrocarpon Müll.Arg. & Apocynaceae Juss. & 8 & 3.5 & 31.5 & 10.026761414789407\\
\cellcolor{gray!10}{1} & \cellcolor{gray!10}{3} & \cellcolor{gray!10}{129} & \cellcolor{gray!10}{Acrocomia aculeata (Jacq.) Lodd. ex Mart} & \cellcolor{gray!10}{Arecaceae Schultz Sch.} & \cellcolor{gray!10}{13} & \cellcolor{gray!10}{11} & \cellcolor{gray!10}{43} & \cellcolor{gray!10}{13.687325105903}\\
1 & 3 & 130 & Vatairea macrocarpa (Benth.) Ducke & Fabaceae Lindl. & 11 & 9 & 46 & 14.642254764454371\\
\addlinespace
\cellcolor{gray!10}{1} & \cellcolor{gray!10}{3} & \cellcolor{gray!10}{131} & \cellcolor{gray!10}{Magonia pubescens A.St.-Hil.} & \cellcolor{gray!10}{Sapindaceae Juss.} & \cellcolor{gray!10}{11} & \cellcolor{gray!10}{7} & \cellcolor{gray!10}{47.5} & \cellcolor{gray!10}{15.119719593730057}\\
1 & 3 & 132 & Morta & - & - & - & - & -\\
\cellcolor{gray!10}{1} & \cellcolor{gray!10}{3} & \cellcolor{gray!10}{133} & \cellcolor{gray!10}{Qualea Aubl.} & \cellcolor{gray!10}{Vochysiaceae A.St.-Hil.} & \cellcolor{gray!10}{11} & \cellcolor{gray!10}{5} & \cellcolor{gray!10}{48.5} & \cellcolor{gray!10}{15.438029479913848}\\
1 & 3 & 134 & Acrocomia aculeata (Jacq.) Lodd. ex Mart & Arecaceae Schultz Sch. & 11.5 & 10 & 50 & 15.915494309189533\\
\cellcolor{gray!10}{1} & \cellcolor{gray!10}{3} & \cellcolor{gray!10}{135} & \cellcolor{gray!10}{Aspidosperma cylindrocarpon Müll.Arg.} & \cellcolor{gray!10}{Apocynaceae Juss.} & \cellcolor{gray!10}{11} & \cellcolor{gray!10}{6.5} & \cellcolor{gray!10}{32.5} & \cellcolor{gray!10}{10.345071300973197}\\
\addlinespace
1 & 3 & 136 & Aspidosperma subincanum Mart. & Apocynaceae Juss. & 12.5 & 6 & 49.3 & 15.692677388860879\\
\cellcolor{gray!10}{1} & \cellcolor{gray!10}{3} & \cellcolor{gray!10}{137} & \cellcolor{gray!10}{Tabebuia roseoalba (Ridl.) Sandwith} & \cellcolor{gray!10}{Bignoniaceae Juss.} & \cellcolor{gray!10}{11.5} & \cellcolor{gray!10}{5} & \cellcolor{gray!10}{45.7} & \cellcolor{gray!10}{14.546761798599235}\\
1 & 3 & 138 & Qualea Aubl. & Vochysiaceae A.St.-Hil. & 10.5 & 6 & 44.6 & 14.196620923797065\\
\cellcolor{gray!10}{1} & \cellcolor{gray!10}{3} & \cellcolor{gray!10}{139} & \cellcolor{gray!10}{Eugenia dysenterica (Mart.) DC.} & \cellcolor{gray!10}{Myrtaceae Juss.} & \cellcolor{gray!10}{8.5} & \cellcolor{gray!10}{6} & \cellcolor{gray!10}{58.9} & \cellcolor{gray!10}{18.74845229622527}\\
1 & 3 & 140 & Simarouba Aubl. & Simaroubaceae DC. & 17 & 12 & 59.8 & 19.034931193790683\\
\addlinespace
\cellcolor{gray!10}{1} & \cellcolor{gray!10}{3} & \cellcolor{gray!10}{141} & \cellcolor{gray!10}{Magonia pubescens A.St.-Hil.} & \cellcolor{gray!10}{Sapindaceae Juss.} & \cellcolor{gray!10}{13} & \cellcolor{gray!10}{7} & \cellcolor{gray!10}{65.5} & \cellcolor{gray!10}{20.84929754503829}\\
1 & 3 & 142 & Platypodium elegans Vogel & Fabaceae Lindl. & 11.5 & 9 & 46.7 & 14.865071684783025\\
\cellcolor{gray!10}{1} & \cellcolor{gray!10}{3} & \cellcolor{gray!10}{143} & \cellcolor{gray!10}{Tabebuia roseoalba (Ridl.) Sandwith} & \cellcolor{gray!10}{Bignoniaceae Juss.} & \cellcolor{gray!10}{10.5} & \cellcolor{gray!10}{6.5} & \cellcolor{gray!10}{61.4} & \cellcolor{gray!10}{19.544227011684747}\\
1 & 3 & 144 & Hymenaea stigonocarpa Mart. ex Hayne & Fabaceae Lindl. & 13 & 7 & 97 & 30.876058959827695\\
\cellcolor{gray!10}{1} & \cellcolor{gray!10}{3} & \cellcolor{gray!10}{145} & \cellcolor{gray!10}{Protium heptaphyllum (Aubl.) Marchand} & \cellcolor{gray!10}{Burseraceae Kunth} & \cellcolor{gray!10}{6} & \cellcolor{gray!10}{5.5} & \cellcolor{gray!10}{32.9} & \cellcolor{gray!10}{10.472395255446713}\\
\addlinespace
1 & 3 & 146 & Tabebuia roseoalba (Ridl.) Sandwith & Bignoniaceae Juss. & 8.5 & 5 & 35.799999999999997 & 11.395493925379705\\
\cellcolor{gray!10}{1} & \cellcolor{gray!10}{3} & \cellcolor{gray!10}{147} & \cellcolor{gray!10}{Magonia pubescens A.St.-Hil.} & \cellcolor{gray!10}{Sapindaceae Juss.} & \cellcolor{gray!10}{12.5} & \cellcolor{gray!10}{10} & \cellcolor{gray!10}{63.2} & \cellcolor{gray!10}{20.117184806815573}\\
1 & 3 & 148 & Callisthene fasciculata Mart. & Vochysiaceae A.St.-Hil. & 13.5 & 10 & 105.1 & 33.4543690379164\\
\cellcolor{gray!10}{1} & \cellcolor{gray!10}{4} & \cellcolor{gray!10}{149} & \cellcolor{gray!10}{Pseudobombax tomentosum (Mart. \& Zucc.) A.Robyns} & \cellcolor{gray!10}{Malvaceae Juss.} & \cellcolor{gray!10}{10} & \cellcolor{gray!10}{4.5} & \cellcolor{gray!10}{84.8} & \cellcolor{gray!10}{26.99267834838545}\\
1 & 4 & 150 & Callisthene fasciculata Mart. & Vochysiaceae A.St.-Hil. & 11.5 & 8 & 76.599999999999994 & 24.382537281678363\\
\addlinespace
\cellcolor{gray!10}{1} & \cellcolor{gray!10}{4} & \cellcolor{gray!10}{151} & \cellcolor{gray!10}{Myracrodruon urundeuva Allemão} & \cellcolor{gray!10}{Anacardiaceae R.Br.} & \cellcolor{gray!10}{12} & \cellcolor{gray!10}{10} & \cellcolor{gray!10}{69.099999999999994} & \cellcolor{gray!10}{21.995213135299934}\\
1 & 4 & 152 & Astronium fraxinifolium Schott & Anacardiaceae R.Br. & 10 & 7.5 & 41.6 & 13.241691265245693\\
\cellcolor{gray!10}{1} & \cellcolor{gray!10}{4} & \cellcolor{gray!10}{153} & \cellcolor{gray!10}{Callisthene fasciculata Mart.} & \cellcolor{gray!10}{Vochysiaceae A.St.-Hil.} & \cellcolor{gray!10}{7.5} & \cellcolor{gray!10}{4} & \cellcolor{gray!10}{93.3} & \cellcolor{gray!10}{29.698312380947669}\\
1 & 4 & 154 & Dipteryx alata Vogel & Fabaceae Lindl. & 10.5 & 6 & 53.5 & 17.0295789108328\\
\cellcolor{gray!10}{1} & \cellcolor{gray!10}{4} & \cellcolor{gray!10}{155} & \cellcolor{gray!10}{Dipteryx alata Vogel} & \cellcolor{gray!10}{Fabaceae Lindl.} & \cellcolor{gray!10}{12.5} & \cellcolor{gray!10}{8.5} & \cellcolor{gray!10}{57} & \cellcolor{gray!10}{18.143663512476071}\\
\addlinespace
1 & 4 & 156 & Myracrodruon urundeuva Allemão & Anacardiaceae R.Br. & 13.5 & 7.5 & 76.2 & 24.255213327204849\\
\cellcolor{gray!10}{1} & \cellcolor{gray!10}{4} & \cellcolor{gray!10}{157} & \cellcolor{gray!10}{Diptychandra aurantiaca Tul.} & \cellcolor{gray!10}{Fabaceae Lindl.} & \cellcolor{gray!10}{13.5} & \cellcolor{gray!10}{10} & \cellcolor{gray!10}{74.5} & \cellcolor{gray!10}{23.714086520692405}\\
1 & 4 & 158 & Callisthene fasciculata Mart. & Vochysiaceae A.St.-Hil. & 13.5 & 5 & 96 & 30.557749073643905\\
\cellcolor{gray!10}{1} & \cellcolor{gray!10}{4} & \cellcolor{gray!10}{159} & \cellcolor{gray!10}{Eugenia dysenterica (Mart.) DC.} & \cellcolor{gray!10}{Myrtaceae Juss.} & \cellcolor{gray!10}{10.5} & \cellcolor{gray!10}{7.5} & \cellcolor{gray!10}{62.6} & \cellcolor{gray!10}{19.926198875105296}\\
1 & 4 & 160 & Roupala montana Aubl. & Proteaceae Juss. & 8.5 & 7 & 32 & 10.185916357881302\\
\addlinespace
\cellcolor{gray!10}{1} & \cellcolor{gray!10}{4} & \cellcolor{gray!10}{161} & \cellcolor{gray!10}{Bowdichia virgilioides Kunth} & \cellcolor{gray!10}{Fabaceae Lindl.} & \cellcolor{gray!10}{14} & \cellcolor{gray!10}{9} & \cellcolor{gray!10}{71.099999999999994} & \cellcolor{gray!10}{22.631832907667516}\\
1 & 4 & 162 & Physocalymma scaberrimum Pohl & Lythraceae J.St.-Hil. & 11 & 7.5 & 54.9 & 17.475212751490108\\
\cellcolor{gray!10}{1} & \cellcolor{gray!10}{4} & \cellcolor{gray!10}{163} & \cellcolor{gray!10}{Myracrodruon urundeuva Allemão} & \cellcolor{gray!10}{Anacardiaceae R.Br.} & \cellcolor{gray!10}{13.5} & \cellcolor{gray!10}{7.5} & \cellcolor{gray!10}{89.3} & \cellcolor{gray!10}{28.425072836212507}\\
1 & 4 & 164 & Callisthene fasciculata Mart. & Vochysiaceae A.St.-Hil. & 10.5 & 7 & 52.8 & 16.806761990504146\\
\cellcolor{gray!10}{1} & \cellcolor{gray!10}{4} & \cellcolor{gray!10}{165} & \cellcolor{gray!10}{Callisthene fasciculata Mart.} & \cellcolor{gray!10}{Vochysiaceae A.St.-Hil.} & \cellcolor{gray!10}{10} & \cellcolor{gray!10}{6.5} & \cellcolor{gray!10}{57.7} & \cellcolor{gray!10}{18.366480432804725}\\
\addlinespace
1 & 4 & 166 & Aspidosperma cylindrocarpon Müll.Arg. & Apocynaceae Juss. & 15 & 9 & 107 & 34.0591578216656\\
\cellcolor{gray!10}{1} & \cellcolor{gray!10}{4} & \cellcolor{gray!10}{167} & \cellcolor{gray!10}{Dipteryx alata Vogel} & \cellcolor{gray!10}{Fabaceae Lindl.} & \cellcolor{gray!10}{11} & \cellcolor{gray!10}{8} & \cellcolor{gray!10}{78.2} & \cellcolor{gray!10}{24.891833099572434}\\
1 & 4 & 168 & Dipteryx alata Vogel & Fabaceae Lindl. & 10.5 & 7.5 & 71.2 & 22.663663896285897\\
\cellcolor{gray!10}{1} & \cellcolor{gray!10}{4} & \cellcolor{gray!10}{169} & \cellcolor{gray!10}{Vatairea macrocarpa (Benth.) Ducke} & \cellcolor{gray!10}{Fabaceae Lindl.} & \cellcolor{gray!10}{10.5} & \cellcolor{gray!10}{8} & \cellcolor{gray!10}{39.200000000000003} & \cellcolor{gray!10}{12.477747538404596}\\
1 & 4 & 170 & PS & - & - & - & - & -\\
\addlinespace
\cellcolor{gray!10}{1} & \cellcolor{gray!10}{4} & \cellcolor{gray!10}{171} & \cellcolor{gray!10}{Handroanthus chrysotrichus (Mart. ex DC.) Mattos} & \cellcolor{gray!10}{Bignoniaceae Juss.} & \cellcolor{gray!10}{9.5} & \cellcolor{gray!10}{6.5} & \cellcolor{gray!10}{50.2} & \cellcolor{gray!10}{15.979156286426294}\\
1 & 4 & 172 & Repetida & - & - & - & - & -\\
\cellcolor{gray!10}{1} & \cellcolor{gray!10}{4} & \cellcolor{gray!10}{173} & \cellcolor{gray!10}{Dipteryx alata Vogel} & \cellcolor{gray!10}{Fabaceae Lindl.} & \cellcolor{gray!10}{11.5} & \cellcolor{gray!10}{9} & \cellcolor{gray!10}{59.8} & \cellcolor{gray!10}{19.034931193790683}\\
1 & 4 & 174 & Myracrodruon urundeuva Allemão & Anacardiaceae R.Br. & 12 & 8.5 & 62.7 & 19.958029863723677\\
\cellcolor{gray!10}{1} & \cellcolor{gray!10}{4} & \cellcolor{gray!10}{175} & \cellcolor{gray!10}{Pseudobombax tomentosum (Mart. \& Zucc.) A.Robyns} & \cellcolor{gray!10}{Malvaceae Juss.} & \cellcolor{gray!10}{7.5} & \cellcolor{gray!10}{6} & \cellcolor{gray!10}{62.6} & \cellcolor{gray!10}{19.926198875105296}\\
\addlinespace
1 & 4 & 176 & Myracrodruon urundeuva Allemão & Anacardiaceae R.Br. & 10 & 7 & 56.9 & 18.111832523857689\\
\cellcolor{gray!10}{1} & \cellcolor{gray!10}{4} & \cellcolor{gray!10}{177} & \cellcolor{gray!10}{Diptychandra aurantiaca Tul.} & \cellcolor{gray!10}{Fabaceae Lindl.} & \cellcolor{gray!10}{8} & \cellcolor{gray!10}{6} & \cellcolor{gray!10}{39.4} & \cellcolor{gray!10}{12.541409515641352}\\
1 & 4 & 178 & Callisthene fasciculata Mart. & Vochysiaceae A.St.-Hil. & 7 & 4 & 71.3 & 22.695494884904274\\
\cellcolor{gray!10}{1} & \cellcolor{gray!10}{4} & \cellcolor{gray!10}{179} & \cellcolor{gray!10}{Aspidosperma cylindrocarpon Müll.Arg.} & \cellcolor{gray!10}{Apocynaceae Juss.} & \cellcolor{gray!10}{7.5} & \cellcolor{gray!10}{4.5} & \cellcolor{gray!10}{34.9} & \cellcolor{gray!10}{11.109015027814294}\\
1 & 4 & 180 & Anadenanthera colubrina (Vell.) Brenan & Fabaceae Lindl. & 12 & 9 & 85.4 & 27.183664280095726\\
\addlinespace
\cellcolor{gray!10}{2} & \cellcolor{gray!10}{1} & \cellcolor{gray!10}{201} & \cellcolor{gray!10}{Protium heptaphyllum (Aubl.) Marchand} & \cellcolor{gray!10}{Burseraceae Kunth} & \cellcolor{gray!10}{9.5} & \cellcolor{gray!10}{6} & \cellcolor{gray!10}{37.700000000000003} & \cellcolor{gray!10}{12.000282709128909}\\
2 & 1 & 202 & Protium heptaphyllum (Aubl.) Marchand & Burseraceae Kunth & 9.5 & 5.5 & 34.4 & 10.949860084722399\\
\cellcolor{gray!10}{2} & \cellcolor{gray!10}{1} & \cellcolor{gray!10}{203} & \cellcolor{gray!10}{Roupala montana Aubl.} & \cellcolor{gray!10}{Proteaceae Juss.} & \cellcolor{gray!10}{9} & \cellcolor{gray!10}{7.5} & \cellcolor{gray!10}{32.5} & \cellcolor{gray!10}{10.345071300973197}\\
2 & 1 & 204 & Protium heptaphyllum (Aubl.) Marchand & Burseraceae Kunth & 8 & 6 & 32 & 10.185916357881302\\
\cellcolor{gray!10}{2} & \cellcolor{gray!10}{1} & \cellcolor{gray!10}{205} & \cellcolor{gray!10}{Protium heptaphyllum (Aubl.) Marchand} & \cellcolor{gray!10}{Burseraceae Kunth} & \cellcolor{gray!10}{9} & \cellcolor{gray!10}{6} & \cellcolor{gray!10}{40.700000000000003} & \cellcolor{gray!10}{12.955212367680282}\\
\addlinespace
2 & 1 & 206 & Roupala montana Aubl. & Proteaceae Juss. & 11 & 7.5 & 44.5 & 14.164789935178685\\
\cellcolor{gray!10}{2} & \cellcolor{gray!10}{1} & \cellcolor{gray!10}{207} & \cellcolor{gray!10}{Tabebuia aurea (Silva Manso) Benth. \& Hook.f. ex S.Moore} & \cellcolor{gray!10}{Bignoniaceae Juss.} & \cellcolor{gray!10}{8.5} & \cellcolor{gray!10}{7} & \cellcolor{gray!10}{52.8} & \cellcolor{gray!10}{16.806761990504146}\\
2 & 1 & 208 & Diptychandra aurantiaca Tul. & Fabaceae Lindl. & 14 & 8 & 87 & 27.69296009798979\\
\cellcolor{gray!10}{2} & \cellcolor{gray!10}{1} & \cellcolor{gray!10}{209} & \cellcolor{gray!10}{Diptychandra aurantiaca Tul.} & \cellcolor{gray!10}{Fabaceae Lindl.} & \cellcolor{gray!10}{11} & \cellcolor{gray!10}{4} & \cellcolor{gray!10}{37.4} & \cellcolor{gray!10}{11.904789743273771}\\
2 & 1 & 210 & Anadenanthera colubrina (Vell.) Brenan & Fabaceae Lindl. & 15.5 & 4.5 & 81 & 25.783100780887047\\
\addlinespace
\cellcolor{gray!10}{2} & \cellcolor{gray!10}{1} & \cellcolor{gray!10}{211} & \cellcolor{gray!10}{Diptychandra aurantiaca Tul.} & \cellcolor{gray!10}{Fabaceae Lindl.} & \cellcolor{gray!10}{13.5} & \cellcolor{gray!10}{5} & \cellcolor{gray!10}{110.7} & \cellcolor{gray!10}{35.236904400545633}\\
2 & 1 & 212 & Anadenanthera colubrina (Vell.) Brenan & Fabaceae Lindl. & 16 & 9 & 46.7 & 14.865071684783025\\
\cellcolor{gray!10}{2} & \cellcolor{gray!10}{1} & \cellcolor{gray!10}{213} & \cellcolor{gray!10}{Platypodium elegans Vogel} & \cellcolor{gray!10}{Fabaceae Lindl.} & \cellcolor{gray!10}{10.5} & \cellcolor{gray!10}{4} & \cellcolor{gray!10}{31.4} & \cellcolor{gray!10}{9.9949304261710274}\\
2 & 1 & 214 & Morfoespécie 7 & NA & 13 & 9.5 & 40.4 & 12.859719401825144\\
\cellcolor{gray!10}{2} & \cellcolor{gray!10}{1} & \cellcolor{gray!10}{215} & \cellcolor{gray!10}{Anadenanthera colubrina (Vell.) Brenan} & \cellcolor{gray!10}{Fabaceae Lindl.} & \cellcolor{gray!10}{16} & \cellcolor{gray!10}{10.5} & \cellcolor{gray!10}{74.8} & \cellcolor{gray!10}{23.809579486547541}\\
\addlinespace
2 & 1 & 216 & Diptychandra aurantiaca Tul. & Fabaceae Lindl. & 13 & 7 & 68.7 & 21.86788918082642\\
\cellcolor{gray!10}{2} & \cellcolor{gray!10}{1} & \cellcolor{gray!10}{217} & \cellcolor{gray!10}{Anadenanthera colubrina (Vell.) Brenan} & \cellcolor{gray!10}{Fabaceae Lindl.} & \cellcolor{gray!10}{18} & \cellcolor{gray!10}{12} & \cellcolor{gray!10}{90.3} & \cellcolor{gray!10}{28.743382722396298}\\
2 & 1 & 218 & Terminalia argentea Mart. & Combretaceae R.Br. & 15.5 & 9 & 87.2 & 27.756622075226549\\
\cellcolor{gray!10}{2} & \cellcolor{gray!10}{1} & \cellcolor{gray!10}{219} & \cellcolor{gray!10}{Aspidosperma cylindrocarpon Müll.Arg.} & \cellcolor{gray!10}{Apocynaceae Juss.} & \cellcolor{gray!10}{14} & \cellcolor{gray!10}{6} & \cellcolor{gray!10}{124.8} & \cellcolor{gray!10}{39.725073795737075}\\
2 & 1 & 220 & Handroanthus impetiginosus (Mart. ex DC.) Mattos & Bignoniaceae Juss. & 13 & 9 & 39.200000000000003 & 12.477747538404596\\
\addlinespace
\cellcolor{gray!10}{2} & \cellcolor{gray!10}{1} & \cellcolor{gray!10}{221} & \cellcolor{gray!10}{Callisthene fasciculata Mart.} & \cellcolor{gray!10}{Vochysiaceae A.St.-Hil.} & \cellcolor{gray!10}{10.5} & \cellcolor{gray!10}{7.5} & \cellcolor{gray!10}{34.200000000000003} & \cellcolor{gray!10}{10.886198107485642}\\
2 & 1 & 222 & Platypodium elegans Vogel & Fabaceae Lindl. & 11.5 & 7 & 55.8 & 17.761691649055518\\
\cellcolor{gray!10}{2} & \cellcolor{gray!10}{1} & \cellcolor{gray!10}{223} & \cellcolor{gray!10}{Acrocomia aculeata (Jacq.) Lodd. ex Mart} & \cellcolor{gray!10}{Arecaceae Schultz Sch.} & \cellcolor{gray!10}{13.5} & \cellcolor{gray!10}{12} & \cellcolor{gray!10}{42} & \cellcolor{gray!10}{13.369015219719209}\\
2 & 1 & 224 & Platypodium elegans Vogel & Fabaceae Lindl. & 14 & 3 & 77 & 24.509861236151881\\
\cellcolor{gray!10}{2} & \cellcolor{gray!10}{1} & \cellcolor{gray!10}{225} & \cellcolor{gray!10}{Pseudobombax tomentosum (Mart. \& Zucc.) A.Robyns} & \cellcolor{gray!10}{Malvaceae Juss.} & \cellcolor{gray!10}{10} & \cellcolor{gray!10}{3} & \cellcolor{gray!10}{98} & \cellcolor{gray!10}{31.194368846011486}\\
\addlinespace
2 & 1 & 226 & Platypodium elegans Vogel & Fabaceae Lindl. & 10 & 7 & 42 & 13.369015219719209\\
\cellcolor{gray!10}{2} & \cellcolor{gray!10}{1} & \cellcolor{gray!10}{227} & \cellcolor{gray!10}{Pseudobombax tomentosum (Mart. \& Zucc.) A.Robyns} & \cellcolor{gray!10}{Malvaceae Juss.} & \cellcolor{gray!10}{8} & \cellcolor{gray!10}{2} & \cellcolor{gray!10}{75.8} & \cellcolor{gray!10}{24.127889372731332}\\
2 & 1 & 228 & Diptychandra aurantiaca Tul. & Fabaceae Lindl. & 14 & 9 & 59.4 & 18.907607239317166\\
\cellcolor{gray!10}{2} & \cellcolor{gray!10}{1} & \cellcolor{gray!10}{229} & \cellcolor{gray!10}{Aspidosperma cylindrocarpon Müll.Arg.} & \cellcolor{gray!10}{Apocynaceae Juss.} & \cellcolor{gray!10}{10} & \cellcolor{gray!10}{7.5} & \cellcolor{gray!10}{34.9} & \cellcolor{gray!10}{11.109015027814294}\\
2 & 1 & 230 & Guapira Aubl. & Nyctaginaceae Juss. & 8.5 & 2 & 55 & 17.507043740108486\\
\addlinespace
\cellcolor{gray!10}{2} & \cellcolor{gray!10}{1} & \cellcolor{gray!10}{231} & \cellcolor{gray!10}{Tabebuia roseoalba (Ridl.) Sandwith} & \cellcolor{gray!10}{Bignoniaceae Juss.} & \cellcolor{gray!10}{11.5} & \cellcolor{gray!10}{6} & \cellcolor{gray!10}{50.7} & \cellcolor{gray!10}{16.138311229518187}\\
2 & 1 & 232 & Diptychandra aurantiaca Tul. & Fabaceae Lindl. & 12 & 7 & 40.299999999999997 & 12.827888413206765\\
\cellcolor{gray!10}{2} & \cellcolor{gray!10}{1} & \cellcolor{gray!10}{233} & \cellcolor{gray!10}{Tabebuia roseoalba (Ridl.) Sandwith} & \cellcolor{gray!10}{Bignoniaceae Juss.} & \cellcolor{gray!10}{9.5} & \cellcolor{gray!10}{4} & \cellcolor{gray!10}{32.9} & \cellcolor{gray!10}{10.472395255446713}\\
2 & 1 & 234 & Callisthene fasciculata Mart. & Vochysiaceae A.St.-Hil. & 12 & 5 & 77.7 & 24.732678156480539\\
\cellcolor{gray!10}{2} & \cellcolor{gray!10}{1} & \cellcolor{gray!10}{235} & \cellcolor{gray!10}{Aspidosperma subincanum Mart.} & \cellcolor{gray!10}{Apocynaceae Juss.} & \cellcolor{gray!10}{11} & \cellcolor{gray!10}{6.5} & \cellcolor{gray!10}{37.299999999999997} & \cellcolor{gray!10}{11.872958754655391}\\
\addlinespace
2 & 1 & 236 & Pseudobombax tomentosum (Mart. \& Zucc.) A.Robyns & Malvaceae Juss. & 11 & 3.5 & 134.1 & 42.685355737246326\\
\cellcolor{gray!10}{2} & \cellcolor{gray!10}{1} & \cellcolor{gray!10}{237} & \cellcolor{gray!10}{Tabebuia roseoalba (Ridl.) Sandwith} & \cellcolor{gray!10}{Bignoniaceae Juss.} & \cellcolor{gray!10}{10} & \cellcolor{gray!10}{4} & \cellcolor{gray!10}{35} & \cellcolor{gray!10}{11.140846016432674}\\
2 & 1 & 238 & Qualea Aubl. & Vochysiaceae A.St.-Hil. & 8.5 & 4 & 36.299999999999997 & 11.554648868471601\\
\cellcolor{gray!10}{2} & \cellcolor{gray!10}{1} & \cellcolor{gray!10}{239} & \cellcolor{gray!10}{Eugenia dysenterica (Mart.) DC.} & \cellcolor{gray!10}{Myrtaceae Juss.} & \cellcolor{gray!10}{7.5} & \cellcolor{gray!10}{2.5} & \cellcolor{gray!10}{59.8} & \cellcolor{gray!10}{19.034931193790683}\\
2 & 1 & 240 & Astronium fraxinifolium Schott & Anacardiaceae R.Br. & 11 & 6 & 36.200000000000003 & 11.522817879853223\\
\addlinespace
\cellcolor{gray!10}{2} & \cellcolor{gray!10}{1} & \cellcolor{gray!10}{241} & \cellcolor{gray!10}{Magonia pubescens A.St.-Hil.} & \cellcolor{gray!10}{Sapindaceae Juss.} & \cellcolor{gray!10}{12} & \cellcolor{gray!10}{8.5} & \cellcolor{gray!10}{54.8} & \cellcolor{gray!10}{17.443381762871727}\\
2 & 1 & 242 & Agonandra brasiliensis Miers ex Benth. \& Hook.f. & Opiliaceae Valeton & 9.5 & 6 & 53.2 & 16.934085944977664\\
\cellcolor{gray!10}{2} & \cellcolor{gray!10}{1} & \cellcolor{gray!10}{243} & \cellcolor{gray!10}{Bowdichia virgilioides Kunth} & \cellcolor{gray!10}{Fabaceae Lindl.} & \cellcolor{gray!10}{15.5} & \cellcolor{gray!10}{8} & \cellcolor{gray!10}{83.7} & \cellcolor{gray!10}{26.642537473583282}\\
2 & 1 & 244 & Myracrodruon urundeuva Allemão & Anacardiaceae R.Br. & 12.2 & 5.5 & 96.6 & 30.748735005354177\\
\cellcolor{gray!10}{2} & \cellcolor{gray!10}{1} & \cellcolor{gray!10}{245} & \cellcolor{gray!10}{Protium heptaphyllum (Aubl.) Marchand} & \cellcolor{gray!10}{Burseraceae Kunth} & \cellcolor{gray!10}{11} & \cellcolor{gray!10}{7.5} & \cellcolor{gray!10}{31.9} & \cellcolor{gray!10}{10.154085369262923}\\
\addlinespace
2 & 1 & 246 & Callisthene fasciculata Mart. & Vochysiaceae A.St.-Hil. & 11.5 & 7 & 45.1 & 14.35577586688896\\
\cellcolor{gray!10}{2} & \cellcolor{gray!10}{1} & \cellcolor{gray!10}{247} & \cellcolor{gray!10}{Protium heptaphyllum (Aubl.) Marchand} & \cellcolor{gray!10}{Burseraceae Kunth} & \cellcolor{gray!10}{12} & \cellcolor{gray!10}{5} & \cellcolor{gray!10}{46.8} & \cellcolor{gray!10}{14.896902673401403}\\
2 & 1 & 248 & Protium heptaphyllum (Aubl.) Marchand & Burseraceae Kunth & 12 & 6 & 37.299999999999997 & 11.872958754655391\\
\cellcolor{gray!10}{2} & \cellcolor{gray!10}{1} & \cellcolor{gray!10}{249} & \cellcolor{gray!10}{Aspidosperma subincanum Mart.} & \cellcolor{gray!10}{Apocynaceae Juss.} & \cellcolor{gray!10}{15} & \cellcolor{gray!10}{4} & \cellcolor{gray!10}{111.2} & \cellcolor{gray!10}{35.396059343637525}\\
2 & 1 & 250 & Protium heptaphyllum (Aubl.) Marchand & Burseraceae Kunth & 14.5 & 8 & 56 & 17.82535362629228\\
\addlinespace
\cellcolor{gray!10}{2} & \cellcolor{gray!10}{1} & \cellcolor{gray!10}{251} & \cellcolor{gray!10}{Platypodium elegans Vogel} & \cellcolor{gray!10}{Fabaceae Lindl.} & \cellcolor{gray!10}{10} & \cellcolor{gray!10}{8} & \cellcolor{gray!10}{31.5} & \cellcolor{gray!10}{10.026761414789407}\\
2 & 2 & 252 & Callisthene fasciculata Mart. & Vochysiaceae A.St.-Hil. & 10 & 4 & 36.5 & 11.618310845708359\\
\cellcolor{gray!10}{2} & \cellcolor{gray!10}{2} & \cellcolor{gray!10}{253} & \cellcolor{gray!10}{Protium heptaphyllum (Aubl.) Marchand} & \cellcolor{gray!10}{Burseraceae Kunth} & \cellcolor{gray!10}{10} & \cellcolor{gray!10}{4} & \cellcolor{gray!10}{63.4} & \cellcolor{gray!10}{20.180846784052328}\\
2 & 2 & 254 & Myracrodruon urundeuva Allemão & Anacardiaceae R.Br. & 10 & 6 & 39.5 & 12.573240504259733\\
\cellcolor{gray!10}{2} & \cellcolor{gray!10}{2} & \cellcolor{gray!10}{255} & \cellcolor{gray!10}{Dipteryx alata Vogel} & \cellcolor{gray!10}{Fabaceae Lindl.} & \cellcolor{gray!10}{11.5} & \cellcolor{gray!10}{7} & \cellcolor{gray!10}{75.099999999999994} & \cellcolor{gray!10}{23.905072452402678}\\
\addlinespace
2 & 2 & 256 & Handroanthus impetiginosus (Mart. ex DC.) Mattos & Bignoniaceae Juss. & 13.5 & 5 & 85.1 & 27.088171314240586\\
\cellcolor{gray!10}{2} & \cellcolor{gray!10}{2} & \cellcolor{gray!10}{257} & \cellcolor{gray!10}{Vatairea macrocarpa (Benth.) Ducke} & \cellcolor{gray!10}{Fabaceae Lindl.} & \cellcolor{gray!10}{10.5} & \cellcolor{gray!10}{8} & \cellcolor{gray!10}{39.200000000000003} & \cellcolor{gray!10}{12.477747538404596}\\
2 & 2 & 258 & Anadenanthera colubrina (Vell.) Brenan & Fabaceae Lindl. & 11 & 9 & 47.5 & 15.119719593730057\\
\cellcolor{gray!10}{2} & \cellcolor{gray!10}{2} & \cellcolor{gray!10}{259} & \cellcolor{gray!10}{Pseudobombax tomentosum (Mart. \& Zucc.) A.Robyns} & \cellcolor{gray!10}{Malvaceae Juss.} & \cellcolor{gray!10}{8} & \cellcolor{gray!10}{3} & \cellcolor{gray!10}{70.099999999999994} & \cellcolor{gray!10}{22.313523021483725}\\
2 & 2 & 260 & Aspidosperma subincanum Mart. & Apocynaceae Juss. & 9.5 & 6.5 & 35.5 & 11.300000959524569\\
\addlinespace
\cellcolor{gray!10}{2} & \cellcolor{gray!10}{2} & \cellcolor{gray!10}{261} & \cellcolor{gray!10}{Aspidosperma cylindrocarpon Müll.Arg.} & \cellcolor{gray!10}{Apocynaceae Juss.} & \cellcolor{gray!10}{9.5} & \cellcolor{gray!10}{6.5} & \cellcolor{gray!10}{40.9} & \cellcolor{gray!10}{13.018874344917039}\\
2 & 2 & 262 & Curatella americana L. & Dilleniaceae  Salisb. & 9.5 & 5.5 & 52.6 & 16.743100013267391\\
\cellcolor{gray!10}{2} & \cellcolor{gray!10}{2} & \cellcolor{gray!10}{263} & \cellcolor{gray!10}{Tabebuia aurea (Silva Manso) Benth. \& Hook.f. ex S.Moore} & \cellcolor{gray!10}{Bignoniaceae Juss.} & \cellcolor{gray!10}{10} & \cellcolor{gray!10}{8} & \cellcolor{gray!10}{64} & \cellcolor{gray!10}{20.371832715762604}\\
2 & 2 & 264 & Protium heptaphyllum (Aubl.) Marchand & Burseraceae Kunth & 10 & 3.5 & 81 & 25.783100780887047\\
\cellcolor{gray!10}{2} & \cellcolor{gray!10}{2} & \cellcolor{gray!10}{265} & \cellcolor{gray!10}{Protium heptaphyllum (Aubl.) Marchand} & \cellcolor{gray!10}{Burseraceae Kunth} & \cellcolor{gray!10}{10} & \cellcolor{gray!10}{5} & \cellcolor{gray!10}{68.2} & \cellcolor{gray!10}{21.708734237734525}\\
\addlinespace
2 & 2 & 266 & Hymenaea stigonocarpa Mart. ex Hayne & Fabaceae Lindl. & 9.5 & 5.5 & 41.7 & 13.273522253864073\\
\cellcolor{gray!10}{2} & \cellcolor{gray!10}{2} & \cellcolor{gray!10}{267} & \cellcolor{gray!10}{Aspidosperma subincanum Mart.} & \cellcolor{gray!10}{Apocynaceae Juss.} & \cellcolor{gray!10}{8} & \cellcolor{gray!10}{6} & \cellcolor{gray!10}{33.299999999999997} & \cellcolor{gray!10}{10.599719209920229}\\
2 & 2 & 268 & Callisthene fasciculata Mart. & Vochysiaceae A.St.-Hil. & 8 & 5 & 45.4 & 14.451268832744097\\
\cellcolor{gray!10}{2} & \cellcolor{gray!10}{2} & \cellcolor{gray!10}{269} & \cellcolor{gray!10}{Roupala montana Aubl.} & \cellcolor{gray!10}{Proteaceae Juss.} & \cellcolor{gray!10}{9} & \cellcolor{gray!10}{6} & \cellcolor{gray!10}{44.4} & \cellcolor{gray!10}{14.132958946560306}\\
2 & 2 & 270 & Acrocomia aculeata (Jacq.) Lodd. ex Mart & Arecaceae Schultz Sch. & 7 & 5 & 39.6 & 12.605071492878112\\
\addlinespace
\cellcolor{gray!10}{2} & \cellcolor{gray!10}{2} & \cellcolor{gray!10}{271} & \cellcolor{gray!10}{Aspidosperma cylindrocarpon Müll.Arg.} & \cellcolor{gray!10}{Apocynaceae Juss.} & \cellcolor{gray!10}{8} & \cellcolor{gray!10}{6} & \cellcolor{gray!10}{31.6} & \cellcolor{gray!10}{10.058592403407786}\\
2 & 2 & 272 & Aspidosperma cylindrocarpon Müll.Arg. & Apocynaceae Juss. & 7.5 & 5.5 & 36.799999999999997 & 11.713803811563496\\
\cellcolor{gray!10}{2} & \cellcolor{gray!10}{2} & \cellcolor{gray!10}{273} & \cellcolor{gray!10}{Vatairea macrocarpa (Benth.) Ducke} & \cellcolor{gray!10}{Fabaceae Lindl.} & \cellcolor{gray!10}{8.5} & \cellcolor{gray!10}{6} & \cellcolor{gray!10}{37.6} & \cellcolor{gray!10}{11.968451720510529}\\
2 & 2 & 274 & Anadenanthera colubrina (Vell.) Brenan & Fabaceae Lindl. & 9 & 5.5 & 45.7 & 14.546761798599235\\
\cellcolor{gray!10}{2} & \cellcolor{gray!10}{2} & \cellcolor{gray!10}{275} & \cellcolor{gray!10}{Vatairea macrocarpa (Benth.) Ducke} & \cellcolor{gray!10}{Fabaceae Lindl.} & \cellcolor{gray!10}{8.5} & \cellcolor{gray!10}{5} & \cellcolor{gray!10}{35.1} & \cellcolor{gray!10}{11.172677005051053}\\
\addlinespace
2 & 2 & 276 & Aspidosperma subincanum Mart. & Apocynaceae Juss. & 8.5 & 5.5 & 37.200000000000003 & 11.841127766037014\\
\cellcolor{gray!10}{2} & \cellcolor{gray!10}{2} & \cellcolor{gray!10}{277} & \cellcolor{gray!10}{Guapira Aubl.} & \cellcolor{gray!10}{Nyctaginaceae Juss.} & \cellcolor{gray!10}{9.5} & \cellcolor{gray!10}{5.5} & \cellcolor{gray!10}{35.4} & \cellcolor{gray!10}{11.26816997090619}\\
2 & 2 & 278 & Pseudobombax tomentosum (Mart. \& Zucc.) A.Robyns & Malvaceae Juss. & 6.5 & 2 & 33.299999999999997 & 10.599719209920229\\
\cellcolor{gray!10}{2} & \cellcolor{gray!10}{2} & \cellcolor{gray!10}{279} & \cellcolor{gray!10}{Guettarda viburnoides Cham. \& Schltdl.} & \cellcolor{gray!10}{Rubiaceae Juss.} & \cellcolor{gray!10}{8.5} & \cellcolor{gray!10}{6.5} & \cellcolor{gray!10}{35.200000000000003} & \cellcolor{gray!10}{11.204507993669433}\\
2 & 2 & 280 & Terminalia argentea Mart. & Combretaceae R.Br. & 9 & 6.5 & 36.6 & 11.650141834326739\\
\addlinespace
\cellcolor{gray!10}{2} & \cellcolor{gray!10}{2} & \cellcolor{gray!10}{281} & \cellcolor{gray!10}{Vatairea macrocarpa (Benth.) Ducke} & \cellcolor{gray!10}{Fabaceae Lindl.} & \cellcolor{gray!10}{10} & \cellcolor{gray!10}{6} & \cellcolor{gray!10}{34.4} & \cellcolor{gray!10}{10.949860084722399}\\
2 & 2 & 282 & Aspidosperma cylindrocarpon Müll.Arg. & Apocynaceae Juss. & 9.5 & 6 & 43 & 13.687325105903\\
\cellcolor{gray!10}{2} & \cellcolor{gray!10}{2} & \cellcolor{gray!10}{283} & \cellcolor{gray!10}{Guettarda viburnoides Cham. \& Schltdl.} & \cellcolor{gray!10}{Rubiaceae Juss.} & \cellcolor{gray!10}{8.5} & \cellcolor{gray!10}{5.5} & \cellcolor{gray!10}{34.9} & \cellcolor{gray!10}{11.109015027814294}\\
2 & 2 & 284 & Callisthene fasciculata Mart. & Vochysiaceae A.St.-Hil. & 8.5 & 3 & 42.8 & 13.623663128666241\\
\cellcolor{gray!10}{2} & \cellcolor{gray!10}{2} & \cellcolor{gray!10}{285} & \cellcolor{gray!10}{Tabebuia roseoalba (Ridl.) Sandwith} & \cellcolor{gray!10}{Bignoniaceae Juss.} & \cellcolor{gray!10}{8} & \cellcolor{gray!10}{4.5} & \cellcolor{gray!10}{37.799999999999997} & \cellcolor{gray!10}{12.032113697747286}\\
\addlinespace
2 & 2 & 286 & Protium heptaphyllum (Aubl.) Marchand & Burseraceae Kunth & 9.5 & 4 & 68.3 & 21.740565226352903\\
\cellcolor{gray!10}{2} & \cellcolor{gray!10}{2} & \cellcolor{gray!10}{287} & \cellcolor{gray!10}{Astronium fraxinifolium Schott} & \cellcolor{gray!10}{Anacardiaceae R.Br.} & \cellcolor{gray!10}{10.5} & \cellcolor{gray!10}{7} & \cellcolor{gray!10}{45.7} & \cellcolor{gray!10}{14.546761798599235}\\
2 & 2 & 288 & Aspidosperma cylindrocarpon Müll.Arg. & Apocynaceae Juss. & 11 & 7 & 61.7 & 19.639719977539887\\
\cellcolor{gray!10}{2} & \cellcolor{gray!10}{2} & \cellcolor{gray!10}{289} & \cellcolor{gray!10}{Aspidosperma cylindrocarpon Müll.Arg.} & \cellcolor{gray!10}{Apocynaceae Juss.} & \cellcolor{gray!10}{11} & \cellcolor{gray!10}{8.5} & \cellcolor{gray!10}{44} & \cellcolor{gray!10}{14.00563499208679}\\
2 & 2 & 290 & Aspidosperma subincanum Mart. & Apocynaceae Juss. & 8.5 & 4.5 & 34.200000000000003 & 10.886198107485642\\
\addlinespace
\cellcolor{gray!10}{2} & \cellcolor{gray!10}{2} & \cellcolor{gray!10}{291} & \cellcolor{gray!10}{Dipteryx alata Vogel} & \cellcolor{gray!10}{Fabaceae Lindl.} & \cellcolor{gray!10}{8.5} & \cellcolor{gray!10}{4.5} & \cellcolor{gray!10}{38.799999999999997} & \cellcolor{gray!10}{12.350423583931077}\\
2 & 2 & 292 & Morfoespécie 8 & NA & 10 & 6 & 39.9 & 12.700564458733249\\
\cellcolor{gray!10}{2} & \cellcolor{gray!10}{2} & \cellcolor{gray!10}{293} & \cellcolor{gray!10}{Protium heptaphyllum (Aubl.) Marchand} & \cellcolor{gray!10}{Burseraceae Kunth} & \cellcolor{gray!10}{10} & \cellcolor{gray!10}{5} & \cellcolor{gray!10}{62.4} & \cellcolor{gray!10}{19.862536897868537}\\
2 & 2 & 294 & Callisthene fasciculata Mart. & Vochysiaceae A.St.-Hil. & 10 & 3.5 & 63.2 & 20.117184806815573\\
\cellcolor{gray!10}{2} & \cellcolor{gray!10}{2} & \cellcolor{gray!10}{295} & \cellcolor{gray!10}{Platypodium elegans Vogel} & \cellcolor{gray!10}{Fabaceae Lindl.} & \cellcolor{gray!10}{11} & \cellcolor{gray!10}{7} & \cellcolor{gray!10}{34} & \cellcolor{gray!10}{10.822536130248883}\\
\addlinespace
2 & 2 & 296 & Callisthene fasciculata Mart. & Vochysiaceae A.St.-Hil. & 10.5 & 4 & 63.2 & 20.117184806815573\\
\cellcolor{gray!10}{2} & \cellcolor{gray!10}{2} & \cellcolor{gray!10}{297} & \cellcolor{gray!10}{Handroanthus impetiginosus (Mart. ex DC.) Mattos} & \cellcolor{gray!10}{Bignoniaceae Juss.} & \cellcolor{gray!10}{12} & \cellcolor{gray!10}{6} & \cellcolor{gray!10}{59.8} & \cellcolor{gray!10}{19.034931193790683}\\
2 & 2 & 298 & Astronium fraxinifolium Schott & Anacardiaceae R.Br. & 10 & 6.5 & 38.799999999999997 & 12.350423583931077\\
\cellcolor{gray!10}{2} & \cellcolor{gray!10}{2} & \cellcolor{gray!10}{299} & \cellcolor{gray!10}{Vatairea macrocarpa (Benth.) Ducke} & \cellcolor{gray!10}{Fabaceae Lindl.} & \cellcolor{gray!10}{11.5} & \cellcolor{gray!10}{7.5} & \cellcolor{gray!10}{43.7} & \cellcolor{gray!10}{13.910142026231654}\\
2 & 2 & 300 & Aspidosperma cylindrocarpon Müll.Arg. & Apocynaceae Juss. & 10 & 6 & 34.200000000000003 & 10.886198107485642\\
\addlinespace
\cellcolor{gray!10}{2} & \cellcolor{gray!10}{2} & \cellcolor{gray!10}{301} & \cellcolor{gray!10}{Myrciaria tenella (DC.) O.Berg} & \cellcolor{gray!10}{Myrtaceae Juss.} & \cellcolor{gray!10}{10.5} & \cellcolor{gray!10}{4} & \cellcolor{gray!10}{31.5} & \cellcolor{gray!10}{10.026761414789407}\\
2 & 2 & 302 & Tabebuia roseoalba (Ridl.) Sandwith & Bignoniaceae Juss. & 9.5 & 3 & 31.4 & 9.9949304261710274\\
\cellcolor{gray!10}{2} & \cellcolor{gray!10}{2} & \cellcolor{gray!10}{303} & \cellcolor{gray!10}{Aspidosperma subincanum Mart.} & \cellcolor{gray!10}{Apocynaceae Juss.} & \cellcolor{gray!10}{10.5} & \cellcolor{gray!10}{7} & \cellcolor{gray!10}{43.7} & \cellcolor{gray!10}{13.910142026231654}\\
2 & 2 & 304 & Aspidosperma cylindrocarpon Müll.Arg. & Apocynaceae Juss. & 10 & 7 & 34.700000000000003 & 11.045353050577537\\
\cellcolor{gray!10}{2} & \cellcolor{gray!10}{2} & \cellcolor{gray!10}{305} & \cellcolor{gray!10}{Aspidosperma cylindrocarpon Müll.Arg.} & \cellcolor{gray!10}{Apocynaceae Juss.} & \cellcolor{gray!10}{10} & \cellcolor{gray!10}{6} & \cellcolor{gray!10}{42.1} & \cellcolor{gray!10}{13.400846208337589}\\
\addlinespace
2 & 2 & 306 & Myrciaria tenella (DC.) O.Berg & Myrtaceae Juss. & 11 & 5 & 42 & 13.369015219719209\\
\cellcolor{gray!10}{2} & \cellcolor{gray!10}{2} & \cellcolor{gray!10}{307} & \cellcolor{gray!10}{Astronium fraxinifolium Schott} & \cellcolor{gray!10}{Anacardiaceae R.Br.} & \cellcolor{gray!10}{12} & \cellcolor{gray!10}{8} & \cellcolor{gray!10}{63.3} & \cellcolor{gray!10}{20.14901579543395}\\
2 & 2 & 448 & Lafoensia pacari A.St.-Hil. & Lythraceae J.St.-Hil. & 7.5 & 3.5 & 39 & 12.414085561167836\\
\cellcolor{gray!10}{2} & \cellcolor{gray!10}{2} & \cellcolor{gray!10}{308} & \cellcolor{gray!10}{Guapira Aubl.} & \cellcolor{gray!10}{Nyctaginaceae Juss.} & \cellcolor{gray!10}{7.5} & \cellcolor{gray!10}{3} & \cellcolor{gray!10}{40} & \cellcolor{gray!10}{12.732395447351628}\\
2 & 3 & 310 & Anadenanthera colubrina (Vell.) Brenan & Fabaceae Lindl. & 10 & 8 & 35.799999999999997 & 11.395493925379705\\
\addlinespace
\cellcolor{gray!10}{2} & \cellcolor{gray!10}{3} & \cellcolor{gray!10}{311} & \cellcolor{gray!10}{Myracrodruon urundeuva Allemão} & \cellcolor{gray!10}{Anacardiaceae R.Br.} & \cellcolor{gray!10}{9.5} & \cellcolor{gray!10}{5} & \cellcolor{gray!10}{44} & \cellcolor{gray!10}{14.00563499208679}\\
2 & 3 & 312 & Rhamnidium elaeocarpum Reissek & Rhamnaceae Juss. & 10 & 6 & 38 & 12.095775674984045\\
\cellcolor{gray!10}{2} & \cellcolor{gray!10}{3} & \cellcolor{gray!10}{313} & \cellcolor{gray!10}{Platypodium elegans Vogel} & \cellcolor{gray!10}{Fabaceae Lindl.} & \cellcolor{gray!10}{11.5} & \cellcolor{gray!10}{7} & \cellcolor{gray!10}{46.1} & \cellcolor{gray!10}{14.674085753072751}\\
2 & 3 & 314 & Astronium fraxinifolium Schott & Anacardiaceae R.Br. & 13 & 8.5 & 65.5 & 20.84929754503829\\
\cellcolor{gray!10}{2} & \cellcolor{gray!10}{3} & \cellcolor{gray!10}{315} & \cellcolor{gray!10}{Aspidosperma cylindrocarpon Müll.Arg.} & \cellcolor{gray!10}{Apocynaceae Juss.} & \cellcolor{gray!10}{10.5} & \cellcolor{gray!10}{7} & \cellcolor{gray!10}{34.1} & \cellcolor{gray!10}{10.854367118867263}\\
\addlinespace
2 & 3 & 316 & Tabebuia roseoalba (Ridl.) Sandwith & Bignoniaceae Juss. & 12 & 8 & 51.7 & 16.456621115701978\\
\cellcolor{gray!10}{2} & \cellcolor{gray!10}{3} & \cellcolor{gray!10}{317} & \cellcolor{gray!10}{Pseudobombax tomentosum (Mart. \& Zucc.) A.Robyns} & \cellcolor{gray!10}{Malvaceae Juss.} & \cellcolor{gray!10}{7.5} & \cellcolor{gray!10}{3} & \cellcolor{gray!10}{65.900000000000006} & \cellcolor{gray!10}{20.976621499511808}\\
2 & 3 & 318 & Aspidosperma cylindrocarpon Müll.Arg. & Apocynaceae Juss. & 10.5 & 5 & 49.9 & 15.883663320571154\\
\cellcolor{gray!10}{2} & \cellcolor{gray!10}{3} & \cellcolor{gray!10}{319} & \cellcolor{gray!10}{Aspidosperma cylindrocarpon Müll.Arg.} & \cellcolor{gray!10}{Apocynaceae Juss.} & \cellcolor{gray!10}{13} & \cellcolor{gray!10}{6} & \cellcolor{gray!10}{67.900000000000006} & \cellcolor{gray!10}{21.613241271879389}\\
2 & 3 & 320 & Aspidosperma cylindrocarpon Müll.Arg. & Apocynaceae Juss. & 9 & 6 & 43.1 & 13.719156094521379\\
\addlinespace
\cellcolor{gray!10}{2} & \cellcolor{gray!10}{3} & \cellcolor{gray!10}{321} & \cellcolor{gray!10}{Pseudobombax tomentosum (Mart. \& Zucc.) A.Robyns} & \cellcolor{gray!10}{Malvaceae Juss.} & \cellcolor{gray!10}{11} & \cellcolor{gray!10}{2} & \cellcolor{gray!10}{88.4} & \cellcolor{gray!10}{28.138593938647098}\\
2 & 3 & 322 & Terminalia argentea Mart. & Combretaceae R.Br. & 13 & 8 & 63.5 & 20.212677772670709\\
\cellcolor{gray!10}{2} & \cellcolor{gray!10}{3} & \cellcolor{gray!10}{323} & \cellcolor{gray!10}{Guapira Aubl.} & \cellcolor{gray!10}{Nyctaginaceae Juss.} & \cellcolor{gray!10}{7.5} & \cellcolor{gray!10}{4} & \cellcolor{gray!10}{36.4} & \cellcolor{gray!10}{11.58647985708998}\\
2 & 3 & 324 & Myracrodruon urundeuva Allemão & Anacardiaceae R.Br. & 10 & 6 & 43.5 & 13.846480048994895\\
\cellcolor{gray!10}{2} & \cellcolor{gray!10}{3} & \cellcolor{gray!10}{325} & \cellcolor{gray!10}{Protium heptaphyllum (Aubl.) Marchand} & \cellcolor{gray!10}{Burseraceae Kunth} & \cellcolor{gray!10}{8} & \cellcolor{gray!10}{6} & \cellcolor{gray!10}{36.299999999999997} & \cellcolor{gray!10}{11.554648868471601}\\
\addlinespace
2 & 3 & 326 & Pseudobombax tomentosum (Mart. \& Zucc.) A.Robyns & Malvaceae Juss. & 10 & 2 & 87.5 & 27.852115041081685\\
\cellcolor{gray!10}{2} & \cellcolor{gray!10}{3} & \cellcolor{gray!10}{327} & \cellcolor{gray!10}{Aspidosperma cylindrocarpon Müll.Arg.} & \cellcolor{gray!10}{Apocynaceae Juss.} & \cellcolor{gray!10}{10} & \cellcolor{gray!10}{7} & \cellcolor{gray!10}{41} & \cellcolor{gray!10}{13.050705333535419}\\
2 & 3 & 328 & Anadenanthera colubrina (Vell.) Brenan & Fabaceae Lindl. & 13.5 & 9 & 63.8 & 20.308170738525845\\
\cellcolor{gray!10}{2} & \cellcolor{gray!10}{3} & \cellcolor{gray!10}{329} & \cellcolor{gray!10}{Guettarda viburnoides Cham. \& Schltdl.} & \cellcolor{gray!10}{Rubiaceae Juss.} & \cellcolor{gray!10}{8.5} & \cellcolor{gray!10}{5} & \cellcolor{gray!10}{32.5} & \cellcolor{gray!10}{10.345071300973197}\\
2 & 3 & 330 & Protium heptaphyllum (Aubl.) Marchand & Burseraceae Kunth & 8 & 5 & 40.1 & 12.764226435970008\\
\addlinespace
\cellcolor{gray!10}{2} & \cellcolor{gray!10}{3} & \cellcolor{gray!10}{331} & \cellcolor{gray!10}{Repitida} & \cellcolor{gray!10}{-} & \cellcolor{gray!10}{-} & \cellcolor{gray!10}{-} & \cellcolor{gray!10}{-} & \cellcolor{gray!10}{-}\\
2 & 3 & 332 & Vatairea macrocarpa (Benth.) Ducke & Fabaceae Lindl. & 11 & 7.5 & 58.1 & 18.493804387278239\\
\cellcolor{gray!10}{2} & \cellcolor{gray!10}{3} & \cellcolor{gray!10}{333} & \cellcolor{gray!10}{Protium heptaphyllum (Aubl.) Marchand} & \cellcolor{gray!10}{Burseraceae Kunth} & \cellcolor{gray!10}{10.5} & \cellcolor{gray!10}{6} & \cellcolor{gray!10}{58.2} & \cellcolor{gray!10}{18.52563537589662}\\
2 & 3 & 334 & Callisthene fasciculata Mart. & Vochysiaceae A.St.-Hil. & 9.5 & 5 & 44.9 & 14.292113889652201\\
\cellcolor{gray!10}{2} & \cellcolor{gray!10}{3} & \cellcolor{gray!10}{335} & \cellcolor{gray!10}{Guapira Aubl.} & \cellcolor{gray!10}{Nyctaginaceae Juss.} & \cellcolor{gray!10}{8} & \cellcolor{gray!10}{5} & \cellcolor{gray!10}{33.200000000000003} & \cellcolor{gray!10}{10.567888221301851}\\
\addlinespace
2 & 3 & 336 & Aspidosperma cylindrocarpon Müll.Arg. & Apocynaceae Juss. & 12 & 5 & 44.1 & 14.03746598070517\\
\cellcolor{gray!10}{2} & \cellcolor{gray!10}{3} & \cellcolor{gray!10}{337} & \cellcolor{gray!10}{Terminalia argentea Mart.} & \cellcolor{gray!10}{Combretaceae R.Br.} & \cellcolor{gray!10}{11.5} & \cellcolor{gray!10}{7.5} & \cellcolor{gray!10}{54} & \cellcolor{gray!10}{17.188733853924695}\\
2 & 3 & 338 & Dilodendron bipinnatum Radlk. & Sapindaceae Juss. & 10 & 7 & 60.6 & 19.289579102737715\\
\cellcolor{gray!10}{2} & \cellcolor{gray!10}{3} & \cellcolor{gray!10}{339} & \cellcolor{gray!10}{Aspidosperma cylindrocarpon Müll.Arg.} & \cellcolor{gray!10}{Apocynaceae Juss.} & \cellcolor{gray!10}{10} & \cellcolor{gray!10}{6} & \cellcolor{gray!10}{46.4} & \cellcolor{gray!10}{14.769578718927887}\\
2 & 3 & 340 & Callisthene fasciculata Mart. & Vochysiaceae A.St.-Hil. & 11.5 & 5 & 62.3 & 19.83070590925016\\
\addlinespace
\cellcolor{gray!10}{2} & \cellcolor{gray!10}{3} & \cellcolor{gray!10}{341} & \cellcolor{gray!10}{Aspidosperma subincanum Mart.} & \cellcolor{gray!10}{Apocynaceae Juss.} & \cellcolor{gray!10}{7} & \cellcolor{gray!10}{5} & \cellcolor{gray!10}{53.4} & \cellcolor{gray!10}{16.997747922214423}\\
2 & 3 & 342 & Aspidosperma subincanum Mart. & Apocynaceae Juss. & 11 & 5 & 58.1 & 18.493804387278239\\
\cellcolor{gray!10}{2} & \cellcolor{gray!10}{3} & \cellcolor{gray!10}{343} & \cellcolor{gray!10}{Protium heptaphyllum (Aubl.) Marchand} & \cellcolor{gray!10}{Burseraceae Kunth} & \cellcolor{gray!10}{10} & \cellcolor{gray!10}{7} & \cellcolor{gray!10}{34.6} & \cellcolor{gray!10}{11.013522061959158}\\
2 & 3 & 344 & Pseudobombax tomentosum (Mart. \& Zucc.) A.Robyns & Malvaceae Juss. & 9 & 7 & 53.8 & 17.125071876687937\\
\cellcolor{gray!10}{2} & \cellcolor{gray!10}{3} & \cellcolor{gray!10}{345} & \cellcolor{gray!10}{Callisthene fasciculata Mart.} & \cellcolor{gray!10}{Vochysiaceae A.St.-Hil.} & \cellcolor{gray!10}{10} & \cellcolor{gray!10}{4} & \cellcolor{gray!10}{37.9} & \cellcolor{gray!10}{12.063944686365666}\\
\addlinespace
2 & 3 & 346 & Callisthene fasciculata Mart. & Vochysiaceae A.St.-Hil. & 12.5 & 9 & 56.4 & 17.952677580765794\\
\cellcolor{gray!10}{2} & \cellcolor{gray!10}{3} & \cellcolor{gray!10}{347} & \cellcolor{gray!10}{Aspidosperma cylindrocarpon Müll.Arg.} & \cellcolor{gray!10}{Apocynaceae Juss.} & \cellcolor{gray!10}{12.5} & \cellcolor{gray!10}{8} & \cellcolor{gray!10}{44.8} & \cellcolor{gray!10}{14.260282901033822}\\
2 & 3 & 348 & Pseudobombax tomentosum (Mart. \& Zucc.) A.Robyns & Malvaceae Juss. & 11.5 & 1.5 & 107.5 & 34.218312764757499\\
\cellcolor{gray!10}{2} & \cellcolor{gray!10}{3} & \cellcolor{gray!10}{349} & \cellcolor{gray!10}{Aspidosperma cylindrocarpon Müll.Arg.} & \cellcolor{gray!10}{Apocynaceae Juss.} & \cellcolor{gray!10}{12.5} & \cellcolor{gray!10}{6.5} & \cellcolor{gray!10}{56.3} & \cellcolor{gray!10}{17.920846592147413}\\
2 & 3 & 350 & Vatairea macrocarpa (Benth.) Ducke & Fabaceae Lindl. & 10.5 & 6 & 50.3 & 16.01098727504467\\
\addlinespace
\cellcolor{gray!10}{2} & \cellcolor{gray!10}{3} & \cellcolor{gray!10}{351} & \cellcolor{gray!10}{Handroanthus impetiginosus (Mart. ex DC.) Mattos} & \cellcolor{gray!10}{Bignoniaceae Juss.} & \cellcolor{gray!10}{12.5} & \cellcolor{gray!10}{6.5} & \cellcolor{gray!10}{77.3} & \cellcolor{gray!10}{24.605354202007018}\\
2 & 3 & 352 & Aspidosperma cylindrocarpon Müll.Arg. & Apocynaceae Juss. & 12 & 6 & 47 & 14.960564650638162\\
\cellcolor{gray!10}{2} & \cellcolor{gray!10}{3} & \cellcolor{gray!10}{353} & \cellcolor{gray!10}{Aspidosperma cylindrocarpon Müll.Arg.} & \cellcolor{gray!10}{Apocynaceae Juss.} & \cellcolor{gray!10}{10.5} & \cellcolor{gray!10}{7} & \cellcolor{gray!10}{50.1} & \cellcolor{gray!10}{15.947325297807915}\\
2 & 3 & 354 & Myracrodruon urundeuva Allemão & Anacardiaceae R.Br. & 11 & 7.5 & 43.9 & 13.973804003468411\\
\cellcolor{gray!10}{2} & \cellcolor{gray!10}{3} & \cellcolor{gray!10}{355} & \cellcolor{gray!10}{Astronium fraxinifolium Schott} & \cellcolor{gray!10}{Anacardiaceae R.Br.} & \cellcolor{gray!10}{13} & \cellcolor{gray!10}{7.5} & \cellcolor{gray!10}{76} & \cellcolor{gray!10}{24.191551349968091}\\
\addlinespace
2 & 3 & 356 & Maytenus Molina & Celastraceae R.Br. & 9.5 & 5 & 60.2 & 19.162255148264201\\
\cellcolor{gray!10}{2} & \cellcolor{gray!10}{3} & \cellcolor{gray!10}{357} & \cellcolor{gray!10}{Callisthene fasciculata Mart.} & \cellcolor{gray!10}{Vochysiaceae A.St.-Hil.} & \cellcolor{gray!10}{9.5} & \cellcolor{gray!10}{5.5} & \cellcolor{gray!10}{53.5} & \cellcolor{gray!10}{17.0295789108328}\\
2 & 3 & 358 & Magonia pubescens A.St.-Hil. & Sapindaceae Juss. & 11 & 7 & 41.7 & 13.273522253864073\\
\cellcolor{gray!10}{2} & \cellcolor{gray!10}{3} & \cellcolor{gray!10}{359} & \cellcolor{gray!10}{Vatairea macrocarpa (Benth.) Ducke} & \cellcolor{gray!10}{Fabaceae Lindl.} & \cellcolor{gray!10}{10.5} & \cellcolor{gray!10}{8} & \cellcolor{gray!10}{39.4} & \cellcolor{gray!10}{12.541409515641352}\\
2 & 3 & 360 & Myracrodruon urundeuva Allemão & Anacardiaceae R.Br. & 11.5 & 6 & 67.2 & 21.390424351550735\\
\addlinespace
\cellcolor{gray!10}{2} & \cellcolor{gray!10}{3} & \cellcolor{gray!10}{361} & \cellcolor{gray!10}{Callisthene fasciculata Mart.} & \cellcolor{gray!10}{Vochysiaceae A.St.-Hil.} & \cellcolor{gray!10}{11} & \cellcolor{gray!10}{5.5} & \cellcolor{gray!10}{58.1} & \cellcolor{gray!10}{18.493804387278239}\\
2 & 3 & 362 & Pouteria torta (Mart.) Radlk. & Sapotaceae Juss. & 9 & 4 & 36.4 & 11.58647985708998\\
\cellcolor{gray!10}{2} & \cellcolor{gray!10}{3} & \cellcolor{gray!10}{363} & \cellcolor{gray!10}{Handroanthus impetiginosus (Mart. ex DC.) Mattos} & \cellcolor{gray!10}{Bignoniaceae Juss.} & \cellcolor{gray!10}{15} & \cellcolor{gray!10}{3.5} & \cellcolor{gray!10}{98.6} & \cellcolor{gray!10}{31.385354777721759}\\
2 & 3 & 364 & Aspidosperma cylindrocarpon Müll.Arg. & Apocynaceae Juss. & 10.5 & 7.5 & 43.8 & 13.941973014850031\\
\cellcolor{gray!10}{2} & \cellcolor{gray!10}{3} & \cellcolor{gray!10}{365} & \cellcolor{gray!10}{Tabebuia roseoalba (Ridl.) Sandwith} & \cellcolor{gray!10}{Bignoniaceae Juss.} & \cellcolor{gray!10}{10} & \cellcolor{gray!10}{6} & \cellcolor{gray!10}{52.5} & \cellcolor{gray!10}{16.71126902464901}\\
\addlinespace
2 & 3 & 366 & Callisthene fasciculata Mart. & Vochysiaceae A.St.-Hil. & 10.5 & 5 & 41.3 & 13.146198299390555\\
\cellcolor{gray!10}{2} & \cellcolor{gray!10}{3} & \cellcolor{gray!10}{367} & \cellcolor{gray!10}{Aspidosperma cylindrocarpon Müll.Arg.} & \cellcolor{gray!10}{Apocynaceae Juss.} & \cellcolor{gray!10}{9.5} & \cellcolor{gray!10}{6.5} & \cellcolor{gray!10}{32.299999999999997} & \cellcolor{gray!10}{10.281409323736439}\\
2 & 3 & 368 & Anadenanthera colubrina (Vell.) Brenan & Fabaceae Lindl. & 11 & 6 & 65.5 & 20.84929754503829\\
\cellcolor{gray!10}{2} & \cellcolor{gray!10}{3} & \cellcolor{gray!10}{369} & \cellcolor{gray!10}{Handroanthus impetiginosus (Mart. ex DC.) Mattos} & \cellcolor{gray!10}{Bignoniaceae Juss.} & \cellcolor{gray!10}{12} & \cellcolor{gray!10}{8.5} & \cellcolor{gray!10}{40.5} & \cellcolor{gray!10}{12.891550390443523}\\
2 & 3 & 370 & Guapira Aubl. & Nyctaginaceae Juss. & 10.5 & 4 & 42.7 & 13.591832140047863\\
\addlinespace
\cellcolor{gray!10}{2} & \cellcolor{gray!10}{3} & \cellcolor{gray!10}{371} & \cellcolor{gray!10}{Rhamnidium elaeocarpum Reissek} & \cellcolor{gray!10}{Rhamnaceae Juss.} & \cellcolor{gray!10}{10.5} & \cellcolor{gray!10}{6.5} & \cellcolor{gray!10}{35} & \cellcolor{gray!10}{11.140846016432674}\\
2 & 3 & 372 & Myracrodruon urundeuva Allemão & Anacardiaceae R.Br. & 11 & 6 & 43.4 & 13.814649060376516\\
\cellcolor{gray!10}{2} & \cellcolor{gray!10}{3} & \cellcolor{gray!10}{373} & \cellcolor{gray!10}{Aspidosperma cylindrocarpon Müll.Arg.} & \cellcolor{gray!10}{Apocynaceae Juss.} & \cellcolor{gray!10}{10.5} & \cellcolor{gray!10}{7} & \cellcolor{gray!10}{41.6} & \cellcolor{gray!10}{13.241691265245693}\\
2 & 3 & 374 & Anadenanthera colubrina (Vell.) Brenan & Fabaceae Lindl. & 12 & 7.5 & 75.7 & 24.096058384112954\\
\cellcolor{gray!10}{2} & \cellcolor{gray!10}{3} & \cellcolor{gray!10}{375} & \cellcolor{gray!10}{Vatairea macrocarpa (Benth.) Ducke} & \cellcolor{gray!10}{Fabaceae Lindl.} & \cellcolor{gray!10}{8.5} & \cellcolor{gray!10}{6} & \cellcolor{gray!10}{34.700000000000003} & \cellcolor{gray!10}{11.045353050577537}\\
\addlinespace
2 & 3 & 376 & Aspidosperma cylindrocarpon Müll.Arg. & Apocynaceae Juss. & 11.5 & 7.5 & 49.2 & 15.660846400242503\\
\cellcolor{gray!10}{2} & \cellcolor{gray!10}{3} & \cellcolor{gray!10}{377} & \cellcolor{gray!10}{Dilodendron bipinnatum Radlk.} & \cellcolor{gray!10}{Sapindaceae Juss.} & \cellcolor{gray!10}{9} & \cellcolor{gray!10}{7} & \cellcolor{gray!10}{32.9} & \cellcolor{gray!10}{10.472395255446713}\\
2 & 3 & 378 & Handroanthus impetiginosus (Mart. ex DC.) Mattos & Bignoniaceae Juss. & 10.5 & 6 & 42.1 & 13.400846208337589\\
\cellcolor{gray!10}{2} & \cellcolor{gray!10}{3} & \cellcolor{gray!10}{379} & \cellcolor{gray!10}{Guettarda viburnoides Cham. \& Schltdl.} & \cellcolor{gray!10}{Rubiaceae Juss.} & \cellcolor{gray!10}{10} & \cellcolor{gray!10}{6.5} & \cellcolor{gray!10}{46.1} & \cellcolor{gray!10}{14.674085753072751}\\
2 & 3 & 380 & Aspidosperma cylindrocarpon Müll.Arg. & Apocynaceae Juss. & 9.5 & 7 & 39 & 12.414085561167836\\
\addlinespace
\cellcolor{gray!10}{2} & \cellcolor{gray!10}{4} & \cellcolor{gray!10}{381} & \cellcolor{gray!10}{Aspidosperma cylindrocarpon Müll.Arg.} & \cellcolor{gray!10}{Apocynaceae Juss.} & \cellcolor{gray!10}{11} & \cellcolor{gray!10}{6.5} & \cellcolor{gray!10}{56.4} & \cellcolor{gray!10}{17.952677580765794}\\
2 & 4 & 382 & Fabaceae 1 & Fabaceae Lindl. & 11.5 & 9 & 63.9 & 20.340001727144223\\
\cellcolor{gray!10}{2} & \cellcolor{gray!10}{4} & \cellcolor{gray!10}{383} & \cellcolor{gray!10}{Protium heptaphyllum (Aubl.) Marchand} & \cellcolor{gray!10}{Burseraceae Kunth} & \cellcolor{gray!10}{10.5} & \cellcolor{gray!10}{6.5} & \cellcolor{gray!10}{50.2} & \cellcolor{gray!10}{15.979156286426294}\\
2 & 4 & 384 & Callisthene fasciculata Mart. & Vochysiaceae A.St.-Hil. & 9.5 & 6.5 & 38.4 & 12.223099629457561\\
\cellcolor{gray!10}{2} & \cellcolor{gray!10}{4} & \cellcolor{gray!10}{385} & \cellcolor{gray!10}{Anadenanthera colubrina (Vell.) Brenan} & \cellcolor{gray!10}{Fabaceae Lindl.} & \cellcolor{gray!10}{9.5} & \cellcolor{gray!10}{6} & \cellcolor{gray!10}{40.5} & \cellcolor{gray!10}{12.891550390443523}\\
\addlinespace
2 & 4 & 386 & Aspidosperma cylindrocarpon Müll.Arg. & Apocynaceae Juss. & 10.5 & 6 & 45.3 & 14.419437844125717\\
\cellcolor{gray!10}{2} & \cellcolor{gray!10}{4} & \cellcolor{gray!10}{387} & \cellcolor{gray!10}{Tabebuia roseoalba (Ridl.) Sandwith} & \cellcolor{gray!10}{Bignoniaceae Juss.} & \cellcolor{gray!10}{9} & \cellcolor{gray!10}{4.5} & \cellcolor{gray!10}{48} & \cellcolor{gray!10}{15.278874536821952}\\
2 & 4 & 388 & Magonia pubescens A.St.-Hil. & Sapindaceae Juss. & 11 & 8 & 54 & 17.188733853924695\\
\cellcolor{gray!10}{2} & \cellcolor{gray!10}{4} & \cellcolor{gray!10}{389} & \cellcolor{gray!10}{Magonia pubescens A.St.-Hil.} & \cellcolor{gray!10}{Sapindaceae Juss.} & \cellcolor{gray!10}{8.5} & \cellcolor{gray!10}{4} & \cellcolor{gray!10}{35.9} & \cellcolor{gray!10}{11.427324913998085}\\
2 & 4 & 390 & Aspidosperma cylindrocarpon Müll.Arg. & Apocynaceae Juss. & 10 & 8 & 40.5 & 12.891550390443523\\
\addlinespace
\cellcolor{gray!10}{2} & \cellcolor{gray!10}{4} & \cellcolor{gray!10}{391} & \cellcolor{gray!10}{Myracrodruon urundeuva Allemão} & \cellcolor{gray!10}{Anacardiaceae R.Br.} & \cellcolor{gray!10}{11} & \cellcolor{gray!10}{7} & \cellcolor{gray!10}{50.1} & \cellcolor{gray!10}{15.947325297807915}\\
2 & 4 & 392 & Astronium fraxinifolium Schott & Anacardiaceae R.Br. & 12 & 6 & 69.2 & 22.027044123918316\\
\cellcolor{gray!10}{2} & \cellcolor{gray!10}{4} & \cellcolor{gray!10}{393} & \cellcolor{gray!10}{Magonia pubescens A.St.-Hil.} & \cellcolor{gray!10}{Sapindaceae Juss.} & \cellcolor{gray!10}{8.5} & \cellcolor{gray!10}{4} & \cellcolor{gray!10}{41.4} & \cellcolor{gray!10}{13.178029288008934}\\
2 & 4 & 394 & Protium heptaphyllum (Aubl.) Marchand & Burseraceae Kunth & 11 & 5 & 57.6 & 18.334649444186343\\
\cellcolor{gray!10}{2} & \cellcolor{gray!10}{4} & \cellcolor{gray!10}{395} & \cellcolor{gray!10}{Platypodium elegans Vogel} & \cellcolor{gray!10}{Fabaceae Lindl.} & \cellcolor{gray!10}{11} & \cellcolor{gray!10}{7.5} & \cellcolor{gray!10}{33.799999999999997} & \cellcolor{gray!10}{10.758874153012124}\\
\addlinespace
2 & 4 & 396 & Aspidosperma cylindrocarpon Müll.Arg. & Apocynaceae Juss. & 10.5 & 7.5 & 32.5 & 10.345071300973197\\
\cellcolor{gray!10}{2} & \cellcolor{gray!10}{4} & \cellcolor{gray!10}{397} & \cellcolor{gray!10}{Aspidosperma subincanum Mart.} & \cellcolor{gray!10}{Apocynaceae Juss.} & \cellcolor{gray!10}{10.5} & \cellcolor{gray!10}{8.5} & \cellcolor{gray!10}{35.6} & \cellcolor{gray!10}{11.331831948142948}\\
2 & 4 & 398 & Acrocomia aculeata (Jacq.) Lodd. ex Mart & Arecaceae Schultz Sch. & 10.5 & 7.5 & 31.5 & 10.026761414789407\\
\cellcolor{gray!10}{2} & \cellcolor{gray!10}{4} & \cellcolor{gray!10}{399} & \cellcolor{gray!10}{Callisthene fasciculata Mart.} & \cellcolor{gray!10}{Vochysiaceae A.St.-Hil.} & \cellcolor{gray!10}{9.5} & \cellcolor{gray!10}{6.5} & \cellcolor{gray!10}{44.4} & \cellcolor{gray!10}{14.132958946560306}\\
2 & 4 & 400 & Fabaceae 1 & Fabaceae Lindl. & 8 & 5 & 51 & 16.233804195373324\\
\addlinespace
\cellcolor{gray!10}{2} & \cellcolor{gray!10}{4} & \cellcolor{gray!10}{401} & \cellcolor{gray!10}{Aspidosperma subincanum Mart.} & \cellcolor{gray!10}{Apocynaceae Juss.} & \cellcolor{gray!10}{10.5} & \cellcolor{gray!10}{7.5} & \cellcolor{gray!10}{45.8} & \cellcolor{gray!10}{14.578592787217612}\\
2 & 4 & 402 & Platypodium elegans Vogel & Fabaceae Lindl. & 12.5 & 8 & 62.3 & 19.83070590925016\\
\cellcolor{gray!10}{2} & \cellcolor{gray!10}{4} & \cellcolor{gray!10}{403} & \cellcolor{gray!10}{Tabebuia roseoalba (Ridl.) Sandwith} & \cellcolor{gray!10}{Bignoniaceae Juss.} & \cellcolor{gray!10}{9} & \cellcolor{gray!10}{6} & \cellcolor{gray!10}{34} & \cellcolor{gray!10}{10.822536130248883}\\
2 & 4 & 404 & Protium heptaphyllum (Aubl.) Marchand & Burseraceae Kunth & 10.5 & 4 & 44.9 & 14.292113889652201\\
\cellcolor{gray!10}{2} & \cellcolor{gray!10}{4} & \cellcolor{gray!10}{405} & \cellcolor{gray!10}{Tabebuia roseoalba (Ridl.) Sandwith} & \cellcolor{gray!10}{Bignoniaceae Juss.} & \cellcolor{gray!10}{11.5} & \cellcolor{gray!10}{7.5} & \cellcolor{gray!10}{37.9} & \cellcolor{gray!10}{12.063944686365666}\\
\addlinespace
2 & 4 & 406 & Qualea Aubl. & Vochysiaceae A.St.-Hil. & 11.5 & 7.5 & 61.5 & 19.576058000303128\\
\cellcolor{gray!10}{2} & \cellcolor{gray!10}{4} & \cellcolor{gray!10}{407} & \cellcolor{gray!10}{Dilodendron bipinnatum Radlk.} & \cellcolor{gray!10}{Sapindaceae Juss.} & \cellcolor{gray!10}{8.5} & \cellcolor{gray!10}{7.5} & \cellcolor{gray!10}{36.4} & \cellcolor{gray!10}{11.58647985708998}\\
2 & 4 & 408 & Callisthene fasciculata Mart. & Vochysiaceae A.St.-Hil. & 10 & 5 & 76.400000000000006 & 24.318875304441612\\
\cellcolor{gray!10}{2} & \cellcolor{gray!10}{4} & \cellcolor{gray!10}{440} & \cellcolor{gray!10}{Pseudobombax tomentosum (Mart. \& Zucc.) A.Robyns} & \cellcolor{gray!10}{Malvaceae Juss.} & \cellcolor{gray!10}{12.5} & \cellcolor{gray!10}{6.5} & \cellcolor{gray!10}{89.5} & \cellcolor{gray!10}{28.488734813449266}\\
2 & 4 & 409 & Diptychandra aurantiaca Tul. & Fabaceae Lindl. & 13 & 9 & 53 & 16.870423967740905\\
\addlinespace
\cellcolor{gray!10}{2} & \cellcolor{gray!10}{4} & \cellcolor{gray!10}{410} & \cellcolor{gray!10}{Tabebuia roseoalba (Ridl.) Sandwith} & \cellcolor{gray!10}{Bignoniaceae Juss.} & \cellcolor{gray!10}{11} & \cellcolor{gray!10}{6.5} & \cellcolor{gray!10}{56.8} & \cellcolor{gray!10}{18.080001535239308}\\
2 & 4 & 411 & Guazuma ulmifolia Lam. & Malvaceae Juss. & 12.5 & 8.5 & 53.4 & 16.997747922214423\\
\cellcolor{gray!10}{2} & \cellcolor{gray!10}{4} & \cellcolor{gray!10}{445} & \cellcolor{gray!10}{Aspidosperma cylindrocarpon Müll.Arg.} & \cellcolor{gray!10}{Apocynaceae Juss.} & \cellcolor{gray!10}{10} & \cellcolor{gray!10}{6.5} & \cellcolor{gray!10}{35} & \cellcolor{gray!10}{11.140846016432674}\\
2 & 4 & 412 & Callisthene fasciculata Mart. & Vochysiaceae A.St.-Hil. & 12 & 6 & 113 & 35.969017138768351\\
\cellcolor{gray!10}{2} & \cellcolor{gray!10}{4} & \cellcolor{gray!10}{413} & \cellcolor{gray!10}{Anadenanthera colubrina (Vell.) Brenan} & \cellcolor{gray!10}{Fabaceae Lindl.} & \cellcolor{gray!10}{12} & \cellcolor{gray!10}{7} & \cellcolor{gray!10}{67.900000000000006} & \cellcolor{gray!10}{21.613241271879389}\\
\addlinespace
2 & 4 & 414 & Myrciaria tenella (DC.) O.Berg & Myrtaceae Juss. & 9.5 & 5.5 & 34.1 & 10.854367118867263\\
\cellcolor{gray!10}{2} & \cellcolor{gray!10}{4} & \cellcolor{gray!10}{415} & \cellcolor{gray!10}{Tabebuia roseoalba (Ridl.) Sandwith} & \cellcolor{gray!10}{Bignoniaceae Juss.} & \cellcolor{gray!10}{12} & \cellcolor{gray!10}{6} & \cellcolor{gray!10}{55.7} & \cellcolor{gray!10}{17.729860660437144}\\
2 & 4 & 416 & Aspidosperma subincanum Mart. & Apocynaceae Juss. & 10 & 6 & 31.5 & 10.026761414789407\\
\cellcolor{gray!10}{2} & \cellcolor{gray!10}{4} & \cellcolor{gray!10}{417} & \cellcolor{gray!10}{Dilodendron bipinnatum Radlk.} & \cellcolor{gray!10}{Sapindaceae Juss.} & \cellcolor{gray!10}{10} & \cellcolor{gray!10}{7} & \cellcolor{gray!10}{31.5} & \cellcolor{gray!10}{10.026761414789407}\\
2 & 4 & 418 & Aspidosperma cylindrocarpon Müll.Arg. & Apocynaceae Juss. & 10 & 5.5 & 39 & 12.414085561167836\\
\addlinespace
\cellcolor{gray!10}{2} & \cellcolor{gray!10}{4} & \cellcolor{gray!10}{419} & \cellcolor{gray!10}{Aspidosperma cylindrocarpon Müll.Arg.} & \cellcolor{gray!10}{Apocynaceae Juss.} & \cellcolor{gray!10}{10.5} & \cellcolor{gray!10}{7} & \cellcolor{gray!10}{33} & \cellcolor{gray!10}{10.504226244065093}\\
2 & 4 & 420 & Aspidosperma cylindrocarpon Müll.Arg. & Apocynaceae Juss. & 11.5 & 6.5 & 55.8 & 17.761691649055518\\
\cellcolor{gray!10}{2} & \cellcolor{gray!10}{4} & \cellcolor{gray!10}{421} & \cellcolor{gray!10}{Guapira Aubl.} & \cellcolor{gray!10}{Nyctaginaceae Juss.} & \cellcolor{gray!10}{7.5} & \cellcolor{gray!10}{2} & \cellcolor{gray!10}{47.2} & \cellcolor{gray!10}{15.024226627874921}\\
2 & 4 & 422 & Astronium fraxinifolium Schott & Anacardiaceae R.Br. & 10.5 & 9 & 47 & 14.960564650638162\\
\cellcolor{gray!10}{2} & \cellcolor{gray!10}{4} & \cellcolor{gray!10}{423} & \cellcolor{gray!10}{Qualea Aubl.} & \cellcolor{gray!10}{Vochysiaceae A.St.-Hil.} & \cellcolor{gray!10}{10.5} & \cellcolor{gray!10}{8} & \cellcolor{gray!10}{65.2} & \cellcolor{gray!10}{20.753804579183154}\\
\addlinespace
2 & 4 & 424 & Platypodium elegans Vogel & Fabaceae Lindl. & 10.5 & 4.5 & 57.9 & 18.43014241004148\\
\cellcolor{gray!10}{2} & \cellcolor{gray!10}{4} & \cellcolor{gray!10}{425} & \cellcolor{gray!10}{Tabebuia roseoalba (Ridl.) Sandwith} & \cellcolor{gray!10}{Bignoniaceae Juss.} & \cellcolor{gray!10}{9.5} & \cellcolor{gray!10}{5} & \cellcolor{gray!10}{43.2} & \cellcolor{gray!10}{13.750987083139758}\\
2 & 4 & 426 & Handroanthus impetiginosus (Mart. ex DC.) Mattos & Bignoniaceae Juss. & 10 & 7 & 35.299999999999997 & 11.23633898228781\\
\cellcolor{gray!10}{2} & \cellcolor{gray!10}{4} & \cellcolor{gray!10}{427} & \cellcolor{gray!10}{Aspidosperma cylindrocarpon Müll.Arg.} & \cellcolor{gray!10}{Apocynaceae Juss.} & \cellcolor{gray!10}{15} & \cellcolor{gray!10}{5} & \cellcolor{gray!10}{171.4} & \cellcolor{gray!10}{54.558314491901726}\\
2 & 4 & 428 & Platypodium elegans Vogel & Fabaceae Lindl. & 11.5 & 6 & 43.6 & 13.878311037613274\\
\addlinespace
\cellcolor{gray!10}{2} & \cellcolor{gray!10}{4} & \cellcolor{gray!10}{429} & \cellcolor{gray!10}{Diptychandra aurantiaca Tul.} & \cellcolor{gray!10}{Fabaceae Lindl.} & \cellcolor{gray!10}{11.5} & \cellcolor{gray!10}{8} & \cellcolor{gray!10}{56.8} & \cellcolor{gray!10}{18.080001535239308}\\
2 & 4 & 430 & Tabebuia roseoalba (Ridl.) Sandwith & Bignoniaceae Juss. & 10 & 6 & 31.8 & 10.122254380644543\\
\cellcolor{gray!10}{2} & \cellcolor{gray!10}{4} & \cellcolor{gray!10}{431} & \cellcolor{gray!10}{Protium heptaphyllum (Aubl.) Marchand} & \cellcolor{gray!10}{Burseraceae Kunth} & \cellcolor{gray!10}{9.5} & \cellcolor{gray!10}{6} & \cellcolor{gray!10}{32.4} & \cellcolor{gray!10}{10.313240312354818}\\
2 & 4 & 432 & Aspidosperma cylindrocarpon Müll.Arg. & Apocynaceae Juss. & 12.5 & 8 & 58.7 & 18.684790318988515\\
\cellcolor{gray!10}{2} & \cellcolor{gray!10}{4} & \cellcolor{gray!10}{433} & \cellcolor{gray!10}{Anadenanthera colubrina (Vell.) Brenan} & \cellcolor{gray!10}{Fabaceae Lindl.} & \cellcolor{gray!10}{12.5} & \cellcolor{gray!10}{8} & \cellcolor{gray!10}{39.1} & \cellcolor{gray!10}{12.445916549786217}\\
\addlinespace
2 & 4 & 434 & Acrocomia aculeata (Jacq.) Lodd. ex Mart & Arecaceae Schultz Sch. & 13 & 11 & 52.7 & 16.774931001885768\\
\cellcolor{gray!10}{2} & \cellcolor{gray!10}{4} & \cellcolor{gray!10}{435} & \cellcolor{gray!10}{Tabebuia roseoalba (Ridl.) Sandwith} & \cellcolor{gray!10}{Bignoniaceae Juss.} & \cellcolor{gray!10}{10.5} & \cellcolor{gray!10}{7} & \cellcolor{gray!10}{44.1} & \cellcolor{gray!10}{14.03746598070517}\\
2 & 4 & 436 & Aspidosperma cylindrocarpon Müll.Arg. & Apocynaceae Juss. & 12 & 6 & 43.3 & 13.782818071758136\\
\cellcolor{gray!10}{2} & \cellcolor{gray!10}{4} & \cellcolor{gray!10}{437} & \cellcolor{gray!10}{Terminalia argentea Mart.} & \cellcolor{gray!10}{Combretaceae R.Br.} & \cellcolor{gray!10}{12} & \cellcolor{gray!10}{8.5} & \cellcolor{gray!10}{36.299999999999997} & \cellcolor{gray!10}{11.554648868471601}\\
2 & 4 & 438 & Diptychandra aurantiaca Tul. & Fabaceae Lindl. & 10 & 7.5 & 38 & 12.095775674984045\\
\addlinespace
\cellcolor{gray!10}{2} & \cellcolor{gray!10}{4} & \cellcolor{gray!10}{439} & \cellcolor{gray!10}{Callisthene fasciculata Mart.} & \cellcolor{gray!10}{Vochysiaceae A.St.-Hil.} & \cellcolor{gray!10}{9} & \cellcolor{gray!10}{5} & \cellcolor{gray!10}{32} & \cellcolor{gray!10}{10.185916357881302}\\
2 & 4 & 440 & Aspidosperma cylindrocarpon Müll.Arg. & Apocynaceae Juss. & 9 & 6 & 34.5 & 10.981691073340778\\
\cellcolor{gray!10}{2} & \cellcolor{gray!10}{4} & \cellcolor{gray!10}{441} & \cellcolor{gray!10}{Platypodium elegans Vogel} & \cellcolor{gray!10}{Fabaceae Lindl.} & \cellcolor{gray!10}{13} & \cellcolor{gray!10}{7.5} & \cellcolor{gray!10}{64.5} & \cellcolor{gray!10}{20.5309876588545}\\
2 & 4 & 442 & Handroanthus impetiginosus (Mart. ex DC.) Mattos & Bignoniaceae Juss. & 13 & 7.5 & 51.9 & 16.520283092938737\\
\cellcolor{gray!10}{2} & \cellcolor{gray!10}{4} & \cellcolor{gray!10}{443} & \cellcolor{gray!10}{Acrocomia aculeata (Jacq.) Lodd. ex Mart} & \cellcolor{gray!10}{Arecaceae Schultz Sch.} & \cellcolor{gray!10}{13} & \cellcolor{gray!10}{11} & \cellcolor{gray!10}{42} & \cellcolor{gray!10}{13.369015219719209}\\
\addlinespace
2 & 4 & 444 & Platypodium elegans Vogel & Fabaceae Lindl. & 14 & 11 & 47.5 & 15.119719593730057\\
\cellcolor{gray!10}{2} & \cellcolor{gray!10}{4} & \cellcolor{gray!10}{446} & \cellcolor{gray!10}{Aspidosperma cylindrocarpon Müll.Arg.} & \cellcolor{gray!10}{Apocynaceae Juss.} & \cellcolor{gray!10}{10.5} & \cellcolor{gray!10}{6} & \cellcolor{gray!10}{39.6} & \cellcolor{gray!10}{12.605071492878112}\\
2 & 4 & 447 & Roupala montana Aubl. & Proteaceae Juss. & 13.5 & 7.5 & 73.2 & 23.300283668653478\\
\bottomrule
\end{tabular}}
\end{table}
\end{landscape}

\subsection{Consistência dos dados}\label{consistuxeancia-dos-dados}

Antes de qualquer análise, é fundamental verificar a consistência dos dados coletados. Isso envolve identificar e corrigir possíveis erros ou inconsistências, garantindo a qualidade e a confiabilidade das informações para as análises subsequentes.

\begin{Shaded}
\begin{Highlighting}[]
\NormalTok{items\_excluir }\OtherTok{\textless{}{-}} \FunctionTok{c}\NormalTok{(}\StringTok{"SP"}\NormalTok{, }\StringTok{"Morta"}\NormalTok{, }\StringTok{"PP"}\NormalTok{, }\StringTok{"PS"}\NormalTok{, }\StringTok{"Repetida"}\NormalTok{, }\StringTok{"Repitida"}\NormalTok{, }\StringTok{"Morfoespécie 7"}\NormalTok{, }\StringTok{"Morfoespécie 8"}\NormalTok{)}
\NormalTok{BD\_ }\OtherTok{\textless{}{-}}\NormalTok{ BD[}\SpecialCharTok{!}\NormalTok{BD}\SpecialCharTok{$}\StringTok{\textasciigrave{}}\AttributeTok{Nome Científico}\StringTok{\textasciigrave{}} \SpecialCharTok{\%in\%}\NormalTok{ items\_excluir, ]}

\NormalTok{BD\_}\SpecialCharTok{$}\NormalTok{DAP }\OtherTok{\textless{}{-}} \FunctionTok{as.numeric}\NormalTok{(BD\_}\SpecialCharTok{$}\NormalTok{DAP)}
\NormalTok{BD\_}\SpecialCharTok{$}\StringTok{\textasciigrave{}}\AttributeTok{H T}\StringTok{\textasciigrave{}} \OtherTok{\textless{}{-}} \FunctionTok{as.numeric}\NormalTok{(BD\_}\SpecialCharTok{$}\StringTok{\textasciigrave{}}\AttributeTok{H T}\StringTok{\textasciigrave{}}\NormalTok{)}
\end{Highlighting}
\end{Shaded}

\subsection{Distribuição diamétrica geral}\label{distribuiuxe7uxe3o-diamuxe9trica-geral}

A análise da distribuição dos diâmetros das árvores fornece insights sobre a estrutura etária e o desenvolvimento da floresta. Uma distribuição equilibrada indica uma regeneração contínua, enquanto a predominância de árvores em determinadas classes diamétricas pode sugerir distúrbios ou práticas de manejo específicas.

\begin{Shaded}
\begin{Highlighting}[]
\NormalTok{intervalo }\OtherTok{\textless{}{-}} \DecValTok{5}
\NormalTok{min\_d }\OtherTok{\textless{}{-}} \FunctionTok{floor}\NormalTok{(}\FunctionTok{min}\NormalTok{(BD\_}\SpecialCharTok{$}\NormalTok{DAP, }\AttributeTok{na.rm =} \ConstantTok{TRUE}\NormalTok{))}
\NormalTok{max\_d }\OtherTok{\textless{}{-}} \FunctionTok{ceiling}\NormalTok{(}\FunctionTok{max}\NormalTok{(BD\_}\SpecialCharTok{$}\NormalTok{DAP, }\AttributeTok{na.rm =} \ConstantTok{TRUE}\NormalTok{))}
\NormalTok{intervalos }\OtherTok{\textless{}{-}} \FunctionTok{seq}\NormalTok{(}\DecValTok{0}\NormalTok{, max\_d }\SpecialCharTok{+}\NormalTok{ intervalo, }\AttributeTok{by =}\NormalTok{ intervalo)}
\FunctionTok{hist}\NormalTok{(BD\_}\SpecialCharTok{$}\NormalTok{DAP, }\AttributeTok{breaks =}\NormalTok{ intervalos, }\AttributeTok{main =} \ConstantTok{NULL}\NormalTok{, }\AttributeTok{xlab =} \StringTok{"Diâmetro (cm)"}\NormalTok{, }\AttributeTok{ylab =} \StringTok{"Frequência"}\NormalTok{)}
\end{Highlighting}
\end{Shaded}

\includegraphics{Entre_Florestas_e_Dados_files/figure-latex/unnamed-chunk-36-1.pdf}

\subsection{Distribuição por espécies (boxplots)}\label{distribuiuxe7uxe3o-por-espuxe9cies-boxplots}

Os boxplots permitem visualizar a variação dos diâmetros por espécie, destacando tendências e identificando outliers. Essa análise auxilia na compreensão do crescimento e do desenvolvimento de cada espécie, bem como na identificação de espécies que possam necessitar de atenção especial no manejo.

\subsubsection{Boxplot dos DAPs}\label{boxplot-dos-daps}

\begin{Shaded}
\begin{Highlighting}[]
\FunctionTok{require}\NormalTok{(ggplot2)}
\NormalTok{especies }\OtherTok{\textless{}{-}} \FunctionTok{unique}\NormalTok{(BD\_}\SpecialCharTok{$}\StringTok{\textasciigrave{}}\AttributeTok{Nome Científico}\StringTok{\textasciigrave{}}\NormalTok{)}

\NormalTok{p }\OtherTok{\textless{}{-}} \FunctionTok{ggplot}\NormalTok{(BD\_, }\FunctionTok{aes}\NormalTok{(}\AttributeTok{x =} \FunctionTok{reorder}\NormalTok{(}\StringTok{\textasciigrave{}}\AttributeTok{Nome Científico}\StringTok{\textasciigrave{}}\NormalTok{, DAP, median), }\AttributeTok{y =}\NormalTok{ DAP, }\AttributeTok{fill =} \StringTok{\textasciigrave{}}\AttributeTok{Nome Científico}\StringTok{\textasciigrave{}}\NormalTok{)) }\SpecialCharTok{+}
  \FunctionTok{geom\_boxplot}\NormalTok{(}\AttributeTok{outlier.shape =} \ConstantTok{NA}\NormalTok{) }\SpecialCharTok{+}  \CommentTok{\# Boxplot sem os outliers}
  \FunctionTok{theme\_minimal}\NormalTok{(}\AttributeTok{base\_size =} \DecValTok{10}\NormalTok{) }\SpecialCharTok{+}
  \FunctionTok{theme}\NormalTok{(}
    \AttributeTok{axis.text.x =} \FunctionTok{element\_text}\NormalTok{(}\AttributeTok{angle =} \DecValTok{45}\NormalTok{, }\AttributeTok{hjust =} \DecValTok{1}\NormalTok{, }\AttributeTok{size =} \DecValTok{6}\NormalTok{),}
    \AttributeTok{legend.position =} \StringTok{"none"}  \CommentTok{\# Remover legenda, se preferir}
\NormalTok{  ) }\SpecialCharTok{+}
  \FunctionTok{labs}\NormalTok{(}\AttributeTok{title =} \StringTok{"Distribuição dos DAPs por Espécie"}\NormalTok{, }
       \AttributeTok{x =} \StringTok{"Espécies"}\NormalTok{, }\AttributeTok{y =} \StringTok{"DAP (cm)"}\NormalTok{)}

\FunctionTok{print}\NormalTok{(p)}
\end{Highlighting}
\end{Shaded}

\includegraphics{Entre_Florestas_e_Dados_files/figure-latex/unnamed-chunk-37-1.pdf}

\subsubsection{boxplot das alturas totais}\label{boxplot-das-alturas-totais}

\begin{Shaded}
\begin{Highlighting}[]
\NormalTok{p }\OtherTok{\textless{}{-}} \FunctionTok{ggplot}\NormalTok{(BD\_, }\FunctionTok{aes}\NormalTok{(}\AttributeTok{x =} \FunctionTok{reorder}\NormalTok{(}\StringTok{\textasciigrave{}}\AttributeTok{Nome Científico}\StringTok{\textasciigrave{}}\NormalTok{, }\StringTok{\textasciigrave{}}\AttributeTok{H T}\StringTok{\textasciigrave{}}\NormalTok{, median), }\AttributeTok{y =} \StringTok{\textasciigrave{}}\AttributeTok{H T}\StringTok{\textasciigrave{}}\NormalTok{, }\AttributeTok{fill =} \StringTok{\textasciigrave{}}\AttributeTok{Nome Científico}\StringTok{\textasciigrave{}}\NormalTok{)) }\SpecialCharTok{+}
  \FunctionTok{geom\_boxplot}\NormalTok{(}\AttributeTok{outlier.shape =} \ConstantTok{NA}\NormalTok{) }\SpecialCharTok{+}  \CommentTok{\# Boxplot sem os outliers}
  \FunctionTok{theme\_minimal}\NormalTok{(}\AttributeTok{base\_size =} \DecValTok{10}\NormalTok{) }\SpecialCharTok{+}
  \FunctionTok{theme}\NormalTok{(}
    \AttributeTok{axis.text.x =} \FunctionTok{element\_text}\NormalTok{(}\AttributeTok{angle =} \DecValTok{45}\NormalTok{, }\AttributeTok{hjust =} \DecValTok{1}\NormalTok{, }\AttributeTok{size =} \DecValTok{6}\NormalTok{),}
    \AttributeTok{legend.position =} \StringTok{"none"}  \CommentTok{\# Remover legenda, se preferir}
\NormalTok{  ) }\SpecialCharTok{+}
  \FunctionTok{labs}\NormalTok{(}\AttributeTok{title =} \StringTok{"Distribuição das HTs por Espécie"}\NormalTok{, }
       \AttributeTok{x =} \StringTok{"Espécies"}\NormalTok{, }\AttributeTok{y =} \StringTok{"HT (m)"}\NormalTok{)}

\FunctionTok{print}\NormalTok{(p)}
\end{Highlighting}
\end{Shaded}

\includegraphics{Entre_Florestas_e_Dados_files/figure-latex/unnamed-chunk-38-1.pdf}

\subsection{Parâmetros Fitossociológicos}\label{paruxe2metros-fitossocioluxf3gicos}

A aplicação prática dos parâmetros fitossociológicos, como densidade, dominância e frequência, fornece uma visão detalhada da composição e da estrutura da comunidade vegetal. Esses indicadores são essenciais para o planejamento de ações de manejo, conservação e recuperação de áreas florestais.

\subsubsection{Estrutura horizontal}\label{estrutura-horizontal-1}

\begin{Shaded}
\begin{Highlighting}[]
\NormalTok{EH }\OtherTok{\textless{}{-}} \ControlFlowTok{function}\NormalTok{(species, sample, d, A) \{}
\NormalTok{    DT }\OtherTok{\textless{}{-}} \FunctionTok{data.table}\NormalTok{(}\AttributeTok{species =}\NormalTok{ species, }\AttributeTok{sample =}\NormalTok{ sample, }\AttributeTok{d =}\NormalTok{ d)}
\NormalTok{    DT }\OtherTok{\textless{}{-}}\NormalTok{ DT[, }\StringTok{\textasciigrave{}}\AttributeTok{:=}\StringTok{\textasciigrave{}}\NormalTok{(}\AttributeTok{gi =}\NormalTok{ pi }\SpecialCharTok{*}\NormalTok{ d}\SpecialCharTok{\^{}}\DecValTok{2} \SpecialCharTok{/} \DecValTok{40000}\NormalTok{)]}
\NormalTok{    Ui }\OtherTok{\textless{}{-}} \FunctionTok{unique}\NormalTok{(DT, }\AttributeTok{by =} \FunctionTok{c}\NormalTok{(}\StringTok{"species"}\NormalTok{, }\StringTok{"sample"}\NormalTok{))[, .(}\AttributeTok{Ui =}\NormalTok{ .N), by }\OtherTok{=} \StringTok{"species"}\NormalTok{]}
\NormalTok{    ni }\OtherTok{\textless{}{-}}\NormalTok{ DT[, .(}\AttributeTok{ni =}\NormalTok{ .N, }\AttributeTok{Gi =} \FunctionTok{sum}\NormalTok{(gi)), by }\OtherTok{=} \StringTok{"species"}\NormalTok{]}
\NormalTok{    ni }\OtherTok{\textless{}{-}}\NormalTok{ ni[Ui, on }\OtherTok{=} \StringTok{"species"}\NormalTok{]}
\NormalTok{    ni[, DAi }\SpecialCharTok{:=}\NormalTok{ ni }\SpecialCharTok{/}\NormalTok{ A][, DRi }\SpecialCharTok{:=}\NormalTok{ (DAi }\SpecialCharTok{/} \FunctionTok{sum}\NormalTok{(DAi)) }\SpecialCharTok{*} \DecValTok{100}\NormalTok{]}
\NormalTok{    ni[, DoAi }\SpecialCharTok{:=}\NormalTok{ Gi }\SpecialCharTok{/}\NormalTok{ A][, DoRi }\SpecialCharTok{:=}\NormalTok{ (DoAi }\SpecialCharTok{/} \FunctionTok{sum}\NormalTok{(DoAi)) }\SpecialCharTok{*} \DecValTok{100}\NormalTok{]}
\NormalTok{    ni[, FAi }\SpecialCharTok{:=}\NormalTok{ (Ui }\SpecialCharTok{/} \FunctionTok{length}\NormalTok{(}\FunctionTok{unique}\NormalTok{(DT}\SpecialCharTok{$}\NormalTok{sample))) }\SpecialCharTok{*} \DecValTok{100}\NormalTok{][, FRi }\SpecialCharTok{:=}\NormalTok{ (FAi }\SpecialCharTok{/} \FunctionTok{sum}\NormalTok{(FAi)) }\SpecialCharTok{*} \DecValTok{100}\NormalTok{]}

\NormalTok{    num\_cols }\OtherTok{\textless{}{-}} \FunctionTok{names}\NormalTok{(ni)[}\FunctionTok{sapply}\NormalTok{(ni, is.numeric)]}
\NormalTok{    ni[, (num\_cols) }\SpecialCharTok{:=} \FunctionTok{lapply}\NormalTok{(.SD, round, }\DecValTok{4}\NormalTok{), .SDcols }\OtherTok{=}\NormalTok{ num\_cols]}

    \FunctionTok{setnames}\NormalTok{(ni, }\AttributeTok{old =} \StringTok{"species"}\NormalTok{, }\AttributeTok{new =} \StringTok{"Espécies"}\NormalTok{)}

    \FunctionTok{return}\NormalTok{(ni)}
\NormalTok{\}}

\NormalTok{EH\_result }\OtherTok{\textless{}{-}} \FunctionTok{EH}\NormalTok{(BD\_}\SpecialCharTok{$}\StringTok{\textasciigrave{}}\AttributeTok{Nome Científico}\StringTok{\textasciigrave{}}\NormalTok{, BD\_}\SpecialCharTok{$}\NormalTok{Parc, BD\_}\SpecialCharTok{$}\NormalTok{DAP, }\AttributeTok{A =} \FloatTok{0.8}\NormalTok{)}
\end{Highlighting}
\end{Shaded}

Resultados:

\begin{landscape}\begin{table}
\centering
\resizebox{\ifdim\width>\linewidth\linewidth\else\width\fi}{!}{
\begin{tabular}{>{\raggedright\arraybackslash}p{3cm}>{\raggedleft\arraybackslash}p{5cm}rrrrrrrr}
\toprule
Espécies & ni & Gi & Ui & DAi & DRi & DoAi & DoRi & FAi & FRi\\
\midrule
\cellcolor{gray!10}{Myracrodruon urundeuva Allemão} & \cellcolor{gray!10}{18} & \cellcolor{gray!10}{0.6704} & \cellcolor{gray!10}{2} & \cellcolor{gray!10}{22.50} & \cellcolor{gray!10}{4.4010} & \cellcolor{gray!10}{0.8380} & \cellcolor{gray!10}{6.3106} & \cellcolor{gray!10}{100} & \cellcolor{gray!10}{2.9412}\\
Tabebuia roseoalba (Ridl.) Sandwith & 23 & 0.3590 & 2 & 28.75 & 5.6235 & 0.4487 & 3.3793 & 100 & 2.9412\\
\cellcolor{gray!10}{Platypodium elegans Vogel} & \cellcolor{gray!10}{35} & \cellcolor{gray!10}{0.6419} & \cellcolor{gray!10}{2} & \cellcolor{gray!10}{43.75} & \cellcolor{gray!10}{8.5575} & \cellcolor{gray!10}{0.8024} & \cellcolor{gray!10}{6.0427} & \cellcolor{gray!10}{100} & \cellcolor{gray!10}{2.9412}\\
Callisthene fasciculata Mart. & 41 & 1.2693 & 2 & 51.25 & 10.0244 & 1.5866 & 11.9479 & 100 & 2.9412\\
\cellcolor{gray!10}{Magonia pubescens A.St.-Hil.} & \cellcolor{gray!10}{12} & \cellcolor{gray!10}{0.3156} & \cellcolor{gray!10}{2} & \cellcolor{gray!10}{15.00} & \cellcolor{gray!10}{2.9340} & \cellcolor{gray!10}{0.3945} & \cellcolor{gray!10}{2.9711} & \cellcolor{gray!10}{100} & \cellcolor{gray!10}{2.9412}\\
\addlinespace
Pseudobombax tomentosum (Mart. \& Zucc.) A.Robyns & 15 & 0.9521 & 2 & 18.75 & 3.6675 & 1.1901 & 8.9621 & 100 & 2.9412\\
\cellcolor{gray!10}{Aspidosperma cylindrocarpon Müll.Arg.} & \cellcolor{gray!10}{60} & \cellcolor{gray!10}{1.3732} & \cellcolor{gray!10}{2} & \cellcolor{gray!10}{75.00} & \cellcolor{gray!10}{14.6699} & \cellcolor{gray!10}{1.7165} & \cellcolor{gray!10}{12.9260} & \cellcolor{gray!10}{100} & \cellcolor{gray!10}{2.9412}\\
Aspidosperma subincanum Mart. & 16 & 0.3735 & 2 & 20.00 & 3.9120 & 0.4669 & 3.5162 & 100 & 2.9412\\
\cellcolor{gray!10}{Anadenanthera colubrina (Vell.) Brenan} & \cellcolor{gray!10}{17} & \cellcolor{gray!10}{0.5402} & \cellcolor{gray!10}{2} & \cellcolor{gray!10}{21.25} & \cellcolor{gray!10}{4.1565} & \cellcolor{gray!10}{0.6753} & \cellcolor{gray!10}{5.0851} & \cellcolor{gray!10}{100} & \cellcolor{gray!10}{2.9412}\\
Astronium fraxinifolium Schott & 15 & 0.4479 & 2 & 18.75 & 3.6675 & 0.5598 & 4.2159 & 100 & 2.9412\\
\addlinespace
\cellcolor{gray!10}{Simarouba Aubl.} & \cellcolor{gray!10}{4} & \cellcolor{gray!10}{0.0606} & \cellcolor{gray!10}{1} & \cellcolor{gray!10}{5.00} & \cellcolor{gray!10}{0.9780} & \cellcolor{gray!10}{0.0758} & \cellcolor{gray!10}{0.5706} & \cellcolor{gray!10}{50} & \cellcolor{gray!10}{1.4706}\\
Handroanthus impetiginosus (Mart. ex DC.) Mattos & 11 & 0.3690 & 2 & 13.75 & 2.6895 & 0.4612 & 3.4733 & 100 & 2.9412\\
\cellcolor{gray!10}{Machaerium acutifolium Vogel} & \cellcolor{gray!10}{1} & \cellcolor{gray!10}{0.0117} & \cellcolor{gray!10}{1} & \cellcolor{gray!10}{1.25} & \cellcolor{gray!10}{0.2445} & \cellcolor{gray!10}{0.0147} & \cellcolor{gray!10}{0.1105} & \cellcolor{gray!10}{50} & \cellcolor{gray!10}{1.4706}\\
Senegalia polyphylla (DC.) Britton \& Rose & 2 & 0.0196 & 1 & 2.50 & 0.4890 & 0.0246 & 0.1850 & 50 & 1.4706\\
\cellcolor{gray!10}{Qualea Aubl.} & \cellcolor{gray!10}{12} & \cellcolor{gray!10}{0.2719} & \cellcolor{gray!10}{2} & \cellcolor{gray!10}{15.00} & \cellcolor{gray!10}{2.9340} & \cellcolor{gray!10}{0.3399} & \cellcolor{gray!10}{2.5594} & \cellcolor{gray!10}{100} & \cellcolor{gray!10}{2.9412}\\
\addlinespace
Dipteryx alata Vogel & 11 & 0.4873 & 2 & 13.75 & 2.6895 & 0.6092 & 4.5874 & 100 & 2.9412\\
\cellcolor{gray!10}{Guapira Aubl.} & \cellcolor{gray!10}{9} & \cellcolor{gray!10}{0.1251} & \cellcolor{gray!10}{2} & \cellcolor{gray!10}{11.25} & \cellcolor{gray!10}{2.2005} & \cellcolor{gray!10}{0.1563} & \cellcolor{gray!10}{1.1772} & \cellcolor{gray!10}{100} & \cellcolor{gray!10}{2.9412}\\
Peltogyne confertiflora (Mart. ex Hayne) Benth. & 1 & 0.0170 & 1 & 1.25 & 0.2445 & 0.0212 & 0.1599 & 50 & 1.4706\\
\cellcolor{gray!10}{Morfoespécie 3} & \cellcolor{gray!10}{1} & \cellcolor{gray!10}{0.0085} & \cellcolor{gray!10}{1} & \cellcolor{gray!10}{1.25} & \cellcolor{gray!10}{0.2445} & \cellcolor{gray!10}{0.0106} & \cellcolor{gray!10}{0.0801} & \cellcolor{gray!10}{50} & \cellcolor{gray!10}{1.4706}\\
Myrciaria tenella (DC.) O.Berg & 4 & 0.0402 & 2 & 5.00 & 0.9780 & 0.0502 & 0.3781 & 100 & 2.9412\\
\addlinespace
\cellcolor{gray!10}{Roupala montana Aubl.} & \cellcolor{gray!10}{6} & \cellcolor{gray!10}{0.0990} & \cellcolor{gray!10}{2} & \cellcolor{gray!10}{7.50} & \cellcolor{gray!10}{1.4670} & \cellcolor{gray!10}{0.1237} & \cellcolor{gray!10}{0.9318} & \cellcolor{gray!10}{100} & \cellcolor{gray!10}{2.9412}\\
Diptychandra aurantiaca Tul. & 12 & 0.3926 & 2 & 15.00 & 2.9340 & 0.4908 & 3.6957 & 100 & 2.9412\\
\cellcolor{gray!10}{Eugenia dysenterica (Mart.) DC.} & \cellcolor{gray!10}{1} & \cellcolor{gray!10}{0.0509} & \cellcolor{gray!10}{1} & \cellcolor{gray!10}{1.25} & \cellcolor{gray!10}{0.2445} & \cellcolor{gray!10}{0.0637} & \cellcolor{gray!10}{0.4794} & \cellcolor{gray!10}{50} & \cellcolor{gray!10}{1.4706}\\
Vatairea macrocarpa (Benth.) Ducke & 13 & 0.2176 & 2 & 16.25 & 3.1785 & 0.2720 & 2.0485 & 100 & 2.9412\\
\cellcolor{gray!10}{Curatella americana L.} & \cellcolor{gray!10}{2} & \cellcolor{gray!10}{0.1159} & \cellcolor{gray!10}{2} & \cellcolor{gray!10}{2.50} & \cellcolor{gray!10}{0.4890} & \cellcolor{gray!10}{0.1448} & \cellcolor{gray!10}{1.0907} & \cellcolor{gray!10}{100} & \cellcolor{gray!10}{2.9412}\\
\addlinespace
Protium heptaphyllum (Aubl.) Marchand & 23 & 0.4445 & 2 & 28.75 & 5.6235 & 0.5556 & 4.1837 & 100 & 2.9412\\
\cellcolor{gray!10}{Tabebuia aurea (Silva Manso) Benth. \& Hook.f. ex S.Moore} & \cellcolor{gray!10}{3} & \cellcolor{gray!10}{0.0844} & \cellcolor{gray!10}{2} & \cellcolor{gray!10}{3.75} & \cellcolor{gray!10}{0.7335} & \cellcolor{gray!10}{0.1055} & \cellcolor{gray!10}{0.7944} & \cellcolor{gray!10}{100} & \cellcolor{gray!10}{2.9412}\\
Eugenia dysenterica (Mart.) DC. & 5 & 0.1051 & 2 & 6.25 & 1.2225 & 0.1314 & 0.9895 & 100 & 2.9412\\
\cellcolor{gray!10}{Bowdichia virgilioides Kunth} & \cellcolor{gray!10}{3} & \cellcolor{gray!10}{0.1122} & \cellcolor{gray!10}{2} & \cellcolor{gray!10}{3.75} & \cellcolor{gray!10}{0.7335} & \cellcolor{gray!10}{0.1403} & \cellcolor{gray!10}{1.0565} & \cellcolor{gray!10}{100} & \cellcolor{gray!10}{2.9412}\\
Acrocomia aculeata (Jacq.) Lodd. ex Mart & 7 & 0.1052 & 2 & 8.75 & 1.7115 & 0.1314 & 0.9899 & 100 & 2.9412\\
\addlinespace
\cellcolor{gray!10}{Hymenaea stigonocarpa Mart. ex Hayne} & \cellcolor{gray!10}{2} & \cellcolor{gray!10}{0.0887} & \cellcolor{gray!10}{2} & \cellcolor{gray!10}{2.50} & \cellcolor{gray!10}{0.4890} & \cellcolor{gray!10}{0.1109} & \cellcolor{gray!10}{0.8351} & \cellcolor{gray!10}{100} & \cellcolor{gray!10}{2.9412}\\
Physocalymma scaberrimum Pohl & 1 & 0.0240 & 1 & 1.25 & 0.2445 & 0.0300 & 0.2258 & 50 & 1.4706\\
\cellcolor{gray!10}{Handroanthus chrysotrichus (Mart. ex DC.) Mattos} & \cellcolor{gray!10}{1} & \cellcolor{gray!10}{0.0201} & \cellcolor{gray!10}{1} & \cellcolor{gray!10}{1.25} & \cellcolor{gray!10}{0.2445} & \cellcolor{gray!10}{0.0251} & \cellcolor{gray!10}{0.1888} & \cellcolor{gray!10}{50} & \cellcolor{gray!10}{1.4706}\\
Terminalia argentea Mart. & 5 & 0.1369 & 1 & 6.25 & 1.2225 & 0.1712 & 1.2891 & 50 & 1.4706\\
\cellcolor{gray!10}{Agonandra brasiliensis Miers ex Benth. \& Hook.f.} & \cellcolor{gray!10}{1} & \cellcolor{gray!10}{0.0225} & \cellcolor{gray!10}{1} & \cellcolor{gray!10}{1.25} & \cellcolor{gray!10}{0.2445} & \cellcolor{gray!10}{0.0282} & \cellcolor{gray!10}{0.2120} & \cellcolor{gray!10}{50} & \cellcolor{gray!10}{1.4706}\\
\addlinespace
Guettarda viburnoides Cham. \& Schltdl. & 4 & 0.0449 & 1 & 5.00 & 0.9780 & 0.0561 & 0.4224 & 50 & 1.4706\\
\cellcolor{gray!10}{Lafoensia pacari A.St.-Hil.} & \cellcolor{gray!10}{1} & \cellcolor{gray!10}{0.0121} & \cellcolor{gray!10}{1} & \cellcolor{gray!10}{1.25} & \cellcolor{gray!10}{0.2445} & \cellcolor{gray!10}{0.0151} & \cellcolor{gray!10}{0.1139} & \cellcolor{gray!10}{50} & \cellcolor{gray!10}{1.4706}\\
Rhamnidium elaeocarpum Reissek & 2 & 0.0212 & 1 & 2.50 & 0.4890 & 0.0265 & 0.1999 & 50 & 1.4706\\
\cellcolor{gray!10}{Dilodendron bipinnatum Radlk.} & \cellcolor{gray!10}{4} & \cellcolor{gray!10}{0.0563} & \cellcolor{gray!10}{1} & \cellcolor{gray!10}{5.00} & \cellcolor{gray!10}{0.9780} & \cellcolor{gray!10}{0.0703} & \cellcolor{gray!10}{0.5298} & \cellcolor{gray!10}{50} & \cellcolor{gray!10}{1.4706}\\
Maytenus Molina & 1 & 0.0288 & 1 & 1.25 & 0.2445 & 0.0360 & 0.2715 & 50 & 1.4706\\
\addlinespace
\cellcolor{gray!10}{Pouteria torta (Mart.) Radlk.} & \cellcolor{gray!10}{1} & \cellcolor{gray!10}{0.0105} & \cellcolor{gray!10}{1} & \cellcolor{gray!10}{1.25} & \cellcolor{gray!10}{0.2445} & \cellcolor{gray!10}{0.0132} & \cellcolor{gray!10}{0.0993} & \cellcolor{gray!10}{50} & \cellcolor{gray!10}{1.4706}\\
Fabaceae 1 & 2 & 0.0532 & 1 & 2.50 & 0.4890 & 0.0665 & 0.5007 & 50 & 1.4706\\
\cellcolor{gray!10}{Guazuma ulmifolia Lam.} & \cellcolor{gray!10}{1} & \cellcolor{gray!10}{0.0227} & \cellcolor{gray!10}{1} & \cellcolor{gray!10}{1.25} & \cellcolor{gray!10}{0.2445} & \cellcolor{gray!10}{0.0284} & \cellcolor{gray!10}{0.2136} & \cellcolor{gray!10}{50} & \cellcolor{gray!10}{1.4706}\\
\bottomrule
\end{tabular}}
\end{table}
\end{landscape}

\subsubsection{Índice de Shannon (Diversidade)}\label{uxedndice-de-shannon-diversidade}

Na aplicação, o cálculo do índice de Shannon avalia a diversidade alfa da comunidade vegetal. Um valor alto reflete maior diversidade e distribuição equitativa entre as espécies no local estudado.

\begin{Shaded}
\begin{Highlighting}[]
\NormalTok{shannon }\OtherTok{\textless{}{-}} \ControlFlowTok{function}\NormalTok{(species\_counts) \{}
\NormalTok{  pi }\OtherTok{\textless{}{-}}\NormalTok{ species\_counts }\SpecialCharTok{/} \FunctionTok{sum}\NormalTok{(species\_counts)}
  \SpecialCharTok{{-}}\FunctionTok{sum}\NormalTok{(pi }\SpecialCharTok{*} \FunctionTok{log}\NormalTok{(pi), }\AttributeTok{na.rm =} \ConstantTok{TRUE}\NormalTok{)}
\NormalTok{\}}

\NormalTok{H }\OtherTok{\textless{}{-}} \FunctionTok{shannon}\NormalTok{(EH\_result}\SpecialCharTok{$}\NormalTok{ni)}
\FunctionTok{cat}\NormalTok{(}\StringTok{"Índice de Shannon:"}\NormalTok{, H, }\StringTok{"}\SpecialCharTok{\textbackslash{}n}\StringTok{"}\NormalTok{)}
\end{Highlighting}
\end{Shaded}

\begin{verbatim}
## Índice de Shannon: 3.16804
\end{verbatim}

\subsubsection{Índice de Morisita (Agregação)}\label{uxedndice-de-morisita-agregauxe7uxe3o}

Mede o padrão de distribuição espacial das espécies. Valores \(I_\delta > 1\) indicam agregação.

\begin{Shaded}
\begin{Highlighting}[]
\NormalTok{morisita }\OtherTok{\textless{}{-}} \ControlFlowTok{function}\NormalTok{(species\_counts, parcelas) \{}
  \CommentTok{\# Agregar os indivíduos por parcela}
\NormalTok{  counts }\OtherTok{\textless{}{-}} \FunctionTok{tapply}\NormalTok{(species\_counts, parcelas, sum)}
  
\NormalTok{  N }\OtherTok{\textless{}{-}} \FunctionTok{sum}\NormalTok{(counts)       }\CommentTok{\# Total de indivíduos em todas as parcelas}
\NormalTok{  n }\OtherTok{\textless{}{-}} \FunctionTok{length}\NormalTok{(counts)    }\CommentTok{\# Número de parcelas}
\NormalTok{  numerator }\OtherTok{\textless{}{-}}\NormalTok{ n }\SpecialCharTok{*} \FunctionTok{sum}\NormalTok{(counts }\SpecialCharTok{*}\NormalTok{ (counts }\SpecialCharTok{{-}} \DecValTok{1}\NormalTok{))  }\CommentTok{\# Soma dos xi(xi {-} 1) por parcela}
\NormalTok{  denominator }\OtherTok{\textless{}{-}}\NormalTok{ N }\SpecialCharTok{*}\NormalTok{ (N }\SpecialCharTok{{-}} \DecValTok{1}\NormalTok{)                   }\CommentTok{\# Total de combinações possíveis}
\NormalTok{  I\_M }\OtherTok{\textless{}{-}}\NormalTok{ numerator }\SpecialCharTok{/}\NormalTok{ denominator}
  \FunctionTok{return}\NormalTok{(I\_M)}
\NormalTok{\}}

\CommentTok{\# Exemplo de aplicação:}
\CommentTok{\# Substituir species\_counts e parcelas pelos dados reais}
\NormalTok{species\_counts }\OtherTok{\textless{}{-}}\NormalTok{ BD\_}\SpecialCharTok{$}\NormalTok{DAP[}\SpecialCharTok{!}\FunctionTok{is.na}\NormalTok{(BD\_}\SpecialCharTok{$}\NormalTok{DAP)]  }\CommentTok{\# DAP como proxy de presença/indivíduo}
\NormalTok{parcelas }\OtherTok{\textless{}{-}}\NormalTok{ BD\_}\SpecialCharTok{$}\NormalTok{Parc                       }\CommentTok{\# Número da parcela correspondente}
\NormalTok{I\_delta }\OtherTok{\textless{}{-}} \FunctionTok{morisita}\NormalTok{(species\_counts, parcelas)}
\FunctionTok{cat}\NormalTok{(}\StringTok{"Índice de Morisita corrigido:"}\NormalTok{, I\_delta, }\StringTok{"}\SpecialCharTok{\textbackslash{}n}\StringTok{"}\NormalTok{)}
\end{Highlighting}
\end{Shaded}

\begin{verbatim}
## Índice de Morisita corrigido: 1.022617
\end{verbatim}

\subsubsection{Estrutura vertical e classificação em estratos}\label{estrutura-vertical-e-classificauxe7uxe3o-em-estratos}

A inclusão da dimensão vertical nos parâmetros fitossociológicos agrega uma visão tridimensional da comunidade florestal, indo além da análise convencional baseada apenas na densidade e dominância (horizontal).

\begin{Shaded}
\begin{Highlighting}[]
\NormalTok{BD\_}\SpecialCharTok{$}\StringTok{\textasciigrave{}}\AttributeTok{H T}\StringTok{\textasciigrave{}} \OtherTok{\textless{}{-}} \FunctionTok{as.numeric}\NormalTok{(BD\_}\SpecialCharTok{$}\StringTok{\textasciigrave{}}\AttributeTok{H T}\StringTok{\textasciigrave{}}\NormalTok{)}
\NormalTok{meanH }\OtherTok{\textless{}{-}} \FunctionTok{mean}\NormalTok{(BD\_}\SpecialCharTok{$}\StringTok{\textasciigrave{}}\AttributeTok{H T}\StringTok{\textasciigrave{}}\NormalTok{, }\AttributeTok{na.rm =} \ConstantTok{TRUE}\NormalTok{)}
\NormalTok{sdH }\OtherTok{\textless{}{-}} \FunctionTok{sd}\NormalTok{(BD\_}\SpecialCharTok{$}\StringTok{\textasciigrave{}}\AttributeTok{H T}\StringTok{\textasciigrave{}}\NormalTok{, }\AttributeTok{na.rm =} \ConstantTok{TRUE}\NormalTok{)}

\CommentTok{\# Calcular estrato}
\NormalTok{BD\_}\SpecialCharTok{$}\NormalTok{estrato }\OtherTok{\textless{}{-}} \FunctionTok{case\_when}\NormalTok{(}
\NormalTok{  BD\_}\SpecialCharTok{$}\StringTok{\textasciigrave{}}\AttributeTok{H T}\StringTok{\textasciigrave{}} \SpecialCharTok{\textless{}}\NormalTok{ (meanH }\SpecialCharTok{{-}}\NormalTok{ sdH) }\SpecialCharTok{\textasciitilde{}} \StringTok{"Inferior"}\NormalTok{,}
\NormalTok{  BD\_}\SpecialCharTok{$}\StringTok{\textasciigrave{}}\AttributeTok{H T}\StringTok{\textasciigrave{}} \SpecialCharTok{\textgreater{}=}\NormalTok{ (meanH }\SpecialCharTok{{-}}\NormalTok{ sdH) }\SpecialCharTok{\&}\NormalTok{ BD\_}\SpecialCharTok{$}\StringTok{\textasciigrave{}}\AttributeTok{H T}\StringTok{\textasciigrave{}} \SpecialCharTok{\textless{}=}\NormalTok{ (meanH }\SpecialCharTok{+}\NormalTok{ sdH) }\SpecialCharTok{\textasciitilde{}} \StringTok{"Médio"}\NormalTok{,}
\NormalTok{  BD\_}\SpecialCharTok{$}\StringTok{\textasciigrave{}}\AttributeTok{H T}\StringTok{\textasciigrave{}} \SpecialCharTok{\textgreater{}}\NormalTok{ (meanH }\SpecialCharTok{+}\NormalTok{ sdH) }\SpecialCharTok{\textasciitilde{}} \StringTok{"Superior"}
\NormalTok{)}

\CommentTok{\# Calcular PSAi e PSRi}
\NormalTok{resultados }\OtherTok{\textless{}{-}}\NormalTok{ BD\_ }\SpecialCharTok{\%\textgreater{}\%}
  \FunctionTok{group\_by}\NormalTok{(estrato) }\SpecialCharTok{\%\textgreater{}\%}
  \FunctionTok{mutate}\NormalTok{(}
    \AttributeTok{Nj =} \FunctionTok{n}\NormalTok{(),                     }\CommentTok{\# Total de indivíduos no estrato}
    \AttributeTok{N =} \FunctionTok{nrow}\NormalTok{(BD\_),                }\CommentTok{\# Total de indivíduos na floresta}
    \AttributeTok{Pj =}\NormalTok{ Nj }\SpecialCharTok{/}\NormalTok{ N                   }\CommentTok{\# Peso do estrato no total}
\NormalTok{  ) }\SpecialCharTok{\%\textgreater{}\%}
  \FunctionTok{group\_by}\NormalTok{(}\StringTok{\textasciigrave{}}\AttributeTok{Nome Científico}\StringTok{\textasciigrave{}}\NormalTok{, estrato) }\SpecialCharTok{\%\textgreater{}\%}
  \FunctionTok{summarise}\NormalTok{(}
    \AttributeTok{Nji =} \FunctionTok{n}\NormalTok{(),                    }\CommentTok{\# Número de indivíduos da espécie no estrato}
    \AttributeTok{Nj =} \FunctionTok{first}\NormalTok{(Nj),               }\CommentTok{\# Total de indivíduos no estrato}
    \AttributeTok{Pj =} \FunctionTok{first}\NormalTok{(Pj),               }\CommentTok{\# Peso do estrato}
    \AttributeTok{PSAi\_partial =}\NormalTok{ (Nji }\SpecialCharTok{/}\NormalTok{ Nj) }\SpecialCharTok{*}\NormalTok{ Pj, }\CommentTok{\# Contribuição parcial para PSAi}
    \AttributeTok{.groups =} \StringTok{"drop"}
\NormalTok{  ) }\SpecialCharTok{\%\textgreater{}\%}
  \FunctionTok{group\_by}\NormalTok{(}\StringTok{\textasciigrave{}}\AttributeTok{Nome Científico}\StringTok{\textasciigrave{}}\NormalTok{) }\SpecialCharTok{\%\textgreater{}\%}
  \FunctionTok{summarise}\NormalTok{(}
    \AttributeTok{PSAi =} \FunctionTok{sum}\NormalTok{(PSAi\_partial, }\AttributeTok{na.rm =} \ConstantTok{TRUE}\NormalTok{)        }\CommentTok{\# Soma das contribuições parciais para PSAi}
\NormalTok{  ) }\SpecialCharTok{\%\textgreater{}\%}
  \FunctionTok{ungroup}\NormalTok{() }\SpecialCharTok{\%\textgreater{}\%}
  \FunctionTok{mutate}\NormalTok{(}
    \AttributeTok{PSRi =}\NormalTok{ (PSAi }\SpecialCharTok{/} \FunctionTok{sum}\NormalTok{(PSAi, }\AttributeTok{na.rm =} \ConstantTok{TRUE}\NormalTok{)) }\SpecialCharTok{*} \DecValTok{100} \CommentTok{\# PSRi como porcentagem global}
\NormalTok{  )}

\CommentTok{\# Fusão das tabelas com left\_join}
\NormalTok{EH\_result }\OtherTok{\textless{}{-}}\NormalTok{ EH\_result }\SpecialCharTok{\%\textgreater{}\%}
  \FunctionTok{left\_join}\NormalTok{(resultados, }\AttributeTok{by =} \FunctionTok{c}\NormalTok{(}\StringTok{"Espécies"} \OtherTok{=} \StringTok{"Nome Científico"}\NormalTok{))}

\CommentTok{\# Adicionando o Valor de Importância Ampliado (VIA)}
\NormalTok{required\_cols }\OtherTok{\textless{}{-}} \FunctionTok{c}\NormalTok{(}\StringTok{"DAi"}\NormalTok{, }\StringTok{"DoAi"}\NormalTok{, }\StringTok{"FAi"}\NormalTok{, }\StringTok{"PSAi"}\NormalTok{, }\StringTok{"DRi"}\NormalTok{, }\StringTok{"DoRi"}\NormalTok{, }\StringTok{"FRi"}\NormalTok{, }\StringTok{"PSRi"}\NormalTok{)}
\NormalTok{missing\_cols }\OtherTok{\textless{}{-}}\NormalTok{ required\_cols[}\SpecialCharTok{!}\NormalTok{required\_cols }\SpecialCharTok{\%in\%} \FunctionTok{names}\NormalTok{(EH\_result)]}

\ControlFlowTok{if}\NormalTok{ (}\FunctionTok{length}\NormalTok{(missing\_cols) }\SpecialCharTok{\textgreater{}} \DecValTok{0}\NormalTok{) \{}
  \FunctionTok{stop}\NormalTok{(}\FunctionTok{paste}\NormalTok{(}\StringTok{"Colunas ausentes em \textasciigrave{}EH\_result\textasciigrave{}: "}\NormalTok{, }\FunctionTok{paste}\NormalTok{(missing\_cols, }\AttributeTok{collapse =} \StringTok{", "}\NormalTok{)))}
\NormalTok{\}}

\NormalTok{EH\_result }\OtherTok{\textless{}{-}}\NormalTok{ EH\_result }\SpecialCharTok{\%\textgreater{}\%}
  \FunctionTok{mutate}\NormalTok{(}
    \AttributeTok{VIAa =}\NormalTok{ (DAi }\SpecialCharTok{+}\NormalTok{ DoAi }\SpecialCharTok{+}\NormalTok{ FAi }\SpecialCharTok{+}\NormalTok{ PSAi) }\SpecialCharTok{/} \DecValTok{4}\NormalTok{,  }\CommentTok{\# Cálculo do VIA absoluto}
\NormalTok{  )}

\CommentTok{\# Visualizar os resultados finais com arredondamento}
\NormalTok{resultado\_final }\OtherTok{\textless{}{-}}\NormalTok{ EH\_result }\SpecialCharTok{\%\textgreater{}\%}
  \FunctionTok{arrange}\NormalTok{(}\FunctionTok{desc}\NormalTok{(VIAa)) }\SpecialCharTok{\%\textgreater{}\%}
  \FunctionTok{mutate}\NormalTok{(}\FunctionTok{across}\NormalTok{(}\FunctionTok{where}\NormalTok{(is.numeric), round, }\DecValTok{4}\NormalTok{))}
\end{Highlighting}
\end{Shaded}

\begin{landscape}\begin{table}
\centering
\resizebox{\ifdim\width>\linewidth\linewidth\else\width\fi}{!}{
\begin{tabular}{>{\raggedright\arraybackslash}p{3cm}>{\raggedleft\arraybackslash}p{5cm}rrrrrrrrrrr}
\toprule
Espécies & ni & Gi & Ui & DAi & DRi & DoAi & DoRi & FAi & FRi & PSAi & PSRi & VIAa\\
\midrule
\cellcolor{gray!10}{Aspidosperma cylindrocarpon Müll.Arg.} & \cellcolor{gray!10}{60} & \cellcolor{gray!10}{1.3732} & \cellcolor{gray!10}{2} & \cellcolor{gray!10}{75.00} & \cellcolor{gray!10}{14.6699} & \cellcolor{gray!10}{1.7165} & \cellcolor{gray!10}{12.9260} & \cellcolor{gray!10}{100} & \cellcolor{gray!10}{2.9412} & \cellcolor{gray!10}{0.1467} & \cellcolor{gray!10}{14.6699} & \cellcolor{gray!10}{44.2158}\\
Callisthene fasciculata Mart. & 41 & 1.2693 & 2 & 51.25 & 10.0244 & 1.5866 & 11.9479 & 100 & 2.9412 & 0.1002 & 10.0244 & 38.2342\\
\cellcolor{gray!10}{Platypodium elegans Vogel} & \cellcolor{gray!10}{35} & \cellcolor{gray!10}{0.6419} & \cellcolor{gray!10}{2} & \cellcolor{gray!10}{43.75} & \cellcolor{gray!10}{8.5575} & \cellcolor{gray!10}{0.8024} & \cellcolor{gray!10}{6.0427} & \cellcolor{gray!10}{100} & \cellcolor{gray!10}{2.9412} & \cellcolor{gray!10}{0.0856} & \cellcolor{gray!10}{8.5575} & \cellcolor{gray!10}{36.1595}\\
Protium heptaphyllum (Aubl.) Marchand & 23 & 0.4445 & 2 & 28.75 & 5.6235 & 0.5556 & 4.1837 & 100 & 2.9412 & 0.0562 & 5.6235 & 32.3405\\
\cellcolor{gray!10}{Tabebuia roseoalba (Ridl.) Sandwith} & \cellcolor{gray!10}{23} & \cellcolor{gray!10}{0.3590} & \cellcolor{gray!10}{2} & \cellcolor{gray!10}{28.75} & \cellcolor{gray!10}{5.6235} & \cellcolor{gray!10}{0.4487} & \cellcolor{gray!10}{3.3793} & \cellcolor{gray!10}{100} & \cellcolor{gray!10}{2.9412} & \cellcolor{gray!10}{0.0562} & \cellcolor{gray!10}{5.6235} & \cellcolor{gray!10}{32.3137}\\
\addlinespace
Myracrodruon urundeuva Allemão & 18 & 0.6704 & 2 & 22.50 & 4.4010 & 0.8380 & 6.3106 & 100 & 2.9412 & 0.0440 & 4.4010 & 30.8455\\
\cellcolor{gray!10}{Anadenanthera colubrina (Vell.) Brenan} & \cellcolor{gray!10}{17} & \cellcolor{gray!10}{0.5402} & \cellcolor{gray!10}{2} & \cellcolor{gray!10}{21.25} & \cellcolor{gray!10}{4.1565} & \cellcolor{gray!10}{0.6753} & \cellcolor{gray!10}{5.0851} & \cellcolor{gray!10}{100} & \cellcolor{gray!10}{2.9412} & \cellcolor{gray!10}{0.0416} & \cellcolor{gray!10}{4.1565} & \cellcolor{gray!10}{30.4917}\\
Aspidosperma subincanum Mart. & 16 & 0.3735 & 2 & 20.00 & 3.9120 & 0.4669 & 3.5162 & 100 & 2.9412 & 0.0391 & 3.9120 & 30.1265\\
\cellcolor{gray!10}{Pseudobombax tomentosum (Mart. \& Zucc.) A.Robyns} & \cellcolor{gray!10}{15} & \cellcolor{gray!10}{0.9521} & \cellcolor{gray!10}{2} & \cellcolor{gray!10}{18.75} & \cellcolor{gray!10}{3.6675} & \cellcolor{gray!10}{1.1901} & \cellcolor{gray!10}{8.9621} & \cellcolor{gray!10}{100} & \cellcolor{gray!10}{2.9412} & \cellcolor{gray!10}{0.0367} & \cellcolor{gray!10}{3.6675} & \cellcolor{gray!10}{29.9942}\\
Astronium fraxinifolium Schott & 15 & 0.4479 & 2 & 18.75 & 3.6675 & 0.5598 & 4.2159 & 100 & 2.9412 & 0.0367 & 3.6675 & 29.8366\\
\addlinespace
\cellcolor{gray!10}{Vatairea macrocarpa (Benth.) Ducke} & \cellcolor{gray!10}{13} & \cellcolor{gray!10}{0.2176} & \cellcolor{gray!10}{2} & \cellcolor{gray!10}{16.25} & \cellcolor{gray!10}{3.1785} & \cellcolor{gray!10}{0.2720} & \cellcolor{gray!10}{2.0485} & \cellcolor{gray!10}{100} & \cellcolor{gray!10}{2.9412} & \cellcolor{gray!10}{0.0318} & \cellcolor{gray!10}{3.1785} & \cellcolor{gray!10}{29.1384}\\
Diptychandra aurantiaca Tul. & 12 & 0.3926 & 2 & 15.00 & 2.9340 & 0.4908 & 3.6957 & 100 & 2.9412 & 0.0293 & 2.9340 & 28.8800\\
\cellcolor{gray!10}{Magonia pubescens A.St.-Hil.} & \cellcolor{gray!10}{12} & \cellcolor{gray!10}{0.3156} & \cellcolor{gray!10}{2} & \cellcolor{gray!10}{15.00} & \cellcolor{gray!10}{2.9340} & \cellcolor{gray!10}{0.3945} & \cellcolor{gray!10}{2.9711} & \cellcolor{gray!10}{100} & \cellcolor{gray!10}{2.9412} & \cellcolor{gray!10}{0.0293} & \cellcolor{gray!10}{2.9340} & \cellcolor{gray!10}{28.8560}\\
Qualea Aubl. & 12 & 0.2719 & 2 & 15.00 & 2.9340 & 0.3399 & 2.5594 & 100 & 2.9412 & 0.0293 & 2.9340 & 28.8423\\
\cellcolor{gray!10}{Dipteryx alata Vogel} & \cellcolor{gray!10}{11} & \cellcolor{gray!10}{0.4873} & \cellcolor{gray!10}{2} & \cellcolor{gray!10}{13.75} & \cellcolor{gray!10}{2.6895} & \cellcolor{gray!10}{0.6092} & \cellcolor{gray!10}{4.5874} & \cellcolor{gray!10}{100} & \cellcolor{gray!10}{2.9412} & \cellcolor{gray!10}{0.0269} & \cellcolor{gray!10}{2.6895} & \cellcolor{gray!10}{28.5965}\\
\addlinespace
Handroanthus impetiginosus (Mart. ex DC.) Mattos & 11 & 0.3690 & 2 & 13.75 & 2.6895 & 0.4612 & 3.4733 & 100 & 2.9412 & 0.0269 & 2.6895 & 28.5595\\
\cellcolor{gray!10}{Guapira Aubl.} & \cellcolor{gray!10}{9} & \cellcolor{gray!10}{0.1251} & \cellcolor{gray!10}{2} & \cellcolor{gray!10}{11.25} & \cellcolor{gray!10}{2.2005} & \cellcolor{gray!10}{0.1563} & \cellcolor{gray!10}{1.1772} & \cellcolor{gray!10}{100} & \cellcolor{gray!10}{2.9412} & \cellcolor{gray!10}{0.0220} & \cellcolor{gray!10}{2.2005} & \cellcolor{gray!10}{27.8571}\\
Acrocomia aculeata (Jacq.) Lodd. ex Mart & 7 & 0.1052 & 2 & 8.75 & 1.7115 & 0.1314 & 0.9899 & 100 & 2.9412 & 0.0171 & 1.7115 & 27.2246\\
\cellcolor{gray!10}{Roupala montana Aubl.} & \cellcolor{gray!10}{6} & \cellcolor{gray!10}{0.0990} & \cellcolor{gray!10}{2} & \cellcolor{gray!10}{7.50} & \cellcolor{gray!10}{1.4670} & \cellcolor{gray!10}{0.1237} & \cellcolor{gray!10}{0.9318} & \cellcolor{gray!10}{100} & \cellcolor{gray!10}{2.9412} & \cellcolor{gray!10}{0.0147} & \cellcolor{gray!10}{1.4670} & \cellcolor{gray!10}{26.9096}\\
Eugenia dysenterica (Mart.) DC. & 5 & 0.1051 & 2 & 6.25 & 1.2225 & 0.1314 & 0.9895 & 100 & 2.9412 & 0.0122 & 1.2225 & 26.5984\\
\addlinespace
\cellcolor{gray!10}{Myrciaria tenella (DC.) O.Berg} & \cellcolor{gray!10}{4} & \cellcolor{gray!10}{0.0402} & \cellcolor{gray!10}{2} & \cellcolor{gray!10}{5.00} & \cellcolor{gray!10}{0.9780} & \cellcolor{gray!10}{0.0502} & \cellcolor{gray!10}{0.3781} & \cellcolor{gray!10}{100} & \cellcolor{gray!10}{2.9412} & \cellcolor{gray!10}{0.0098} & \cellcolor{gray!10}{0.9780} & \cellcolor{gray!10}{26.2650}\\
Bowdichia virgilioides Kunth & 3 & 0.1122 & 2 & 3.75 & 0.7335 & 0.1403 & 1.0565 & 100 & 2.9412 & 0.0073 & 0.7335 & 25.9744\\
\cellcolor{gray!10}{Tabebuia aurea (Silva Manso) Benth. \& Hook.f. ex S.Moore} & \cellcolor{gray!10}{3} & \cellcolor{gray!10}{0.0844} & \cellcolor{gray!10}{2} & \cellcolor{gray!10}{3.75} & \cellcolor{gray!10}{0.7335} & \cellcolor{gray!10}{0.1055} & \cellcolor{gray!10}{0.7944} & \cellcolor{gray!10}{100} & \cellcolor{gray!10}{2.9412} & \cellcolor{gray!10}{0.0073} & \cellcolor{gray!10}{0.7335} & \cellcolor{gray!10}{25.9657}\\
Curatella americana L. & 2 & 0.1159 & 2 & 2.50 & 0.4890 & 0.1448 & 1.0907 & 100 & 2.9412 & 0.0049 & 0.4890 & 25.6624\\
\cellcolor{gray!10}{Hymenaea stigonocarpa Mart. ex Hayne} & \cellcolor{gray!10}{2} & \cellcolor{gray!10}{0.0887} & \cellcolor{gray!10}{2} & \cellcolor{gray!10}{2.50} & \cellcolor{gray!10}{0.4890} & \cellcolor{gray!10}{0.1109} & \cellcolor{gray!10}{0.8351} & \cellcolor{gray!10}{100} & \cellcolor{gray!10}{2.9412} & \cellcolor{gray!10}{0.0049} & \cellcolor{gray!10}{0.4890} & \cellcolor{gray!10}{25.6539}\\
\addlinespace
Terminalia argentea Mart. & 5 & 0.1369 & 1 & 6.25 & 1.2225 & 0.1712 & 1.2891 & 50 & 1.4706 & 0.0122 & 1.2225 & 14.1084\\
\cellcolor{gray!10}{Simarouba Aubl.} & \cellcolor{gray!10}{4} & \cellcolor{gray!10}{0.0606} & \cellcolor{gray!10}{1} & \cellcolor{gray!10}{5.00} & \cellcolor{gray!10}{0.9780} & \cellcolor{gray!10}{0.0758} & \cellcolor{gray!10}{0.5706} & \cellcolor{gray!10}{50} & \cellcolor{gray!10}{1.4706} & \cellcolor{gray!10}{0.0098} & \cellcolor{gray!10}{0.9780} & \cellcolor{gray!10}{13.7714}\\
Dilodendron bipinnatum Radlk. & 4 & 0.0563 & 1 & 5.00 & 0.9780 & 0.0703 & 0.5298 & 50 & 1.4706 & 0.0098 & 0.9780 & 13.7700\\
\cellcolor{gray!10}{Guettarda viburnoides Cham. \& Schltdl.} & \cellcolor{gray!10}{4} & \cellcolor{gray!10}{0.0449} & \cellcolor{gray!10}{1} & \cellcolor{gray!10}{5.00} & \cellcolor{gray!10}{0.9780} & \cellcolor{gray!10}{0.0561} & \cellcolor{gray!10}{0.4224} & \cellcolor{gray!10}{50} & \cellcolor{gray!10}{1.4706} & \cellcolor{gray!10}{0.0098} & \cellcolor{gray!10}{0.9780} & \cellcolor{gray!10}{13.7665}\\
Fabaceae 1 & 2 & 0.0532 & 1 & 2.50 & 0.4890 & 0.0665 & 0.5007 & 50 & 1.4706 & 0.0049 & 0.4890 & 13.1428\\
\addlinespace
\cellcolor{gray!10}{Rhamnidium elaeocarpum Reissek} & \cellcolor{gray!10}{2} & \cellcolor{gray!10}{0.0212} & \cellcolor{gray!10}{1} & \cellcolor{gray!10}{2.50} & \cellcolor{gray!10}{0.4890} & \cellcolor{gray!10}{0.0265} & \cellcolor{gray!10}{0.1999} & \cellcolor{gray!10}{50} & \cellcolor{gray!10}{1.4706} & \cellcolor{gray!10}{0.0049} & \cellcolor{gray!10}{0.4890} & \cellcolor{gray!10}{13.1328}\\
Senegalia polyphylla (DC.) Britton \& Rose & 2 & 0.0196 & 1 & 2.50 & 0.4890 & 0.0246 & 0.1850 & 50 & 1.4706 & 0.0049 & 0.4890 & 13.1324\\
\cellcolor{gray!10}{Eugenia dysenterica (Mart.) DC.} & \cellcolor{gray!10}{1} & \cellcolor{gray!10}{0.0509} & \cellcolor{gray!10}{1} & \cellcolor{gray!10}{1.25} & \cellcolor{gray!10}{0.2445} & \cellcolor{gray!10}{0.0637} & \cellcolor{gray!10}{0.4794} & \cellcolor{gray!10}{50} & \cellcolor{gray!10}{1.4706} & \cellcolor{gray!10}{0.0024} & \cellcolor{gray!10}{0.2445} & \cellcolor{gray!10}{12.8290}\\
Maytenus Molina & 1 & 0.0288 & 1 & 1.25 & 0.2445 & 0.0360 & 0.2715 & 50 & 1.4706 & 0.0024 & 0.2445 & 12.8221\\
\cellcolor{gray!10}{Physocalymma scaberrimum Pohl} & \cellcolor{gray!10}{1} & \cellcolor{gray!10}{0.0240} & \cellcolor{gray!10}{1} & \cellcolor{gray!10}{1.25} & \cellcolor{gray!10}{0.2445} & \cellcolor{gray!10}{0.0300} & \cellcolor{gray!10}{0.2258} & \cellcolor{gray!10}{50} & \cellcolor{gray!10}{1.4706} & \cellcolor{gray!10}{0.0024} & \cellcolor{gray!10}{0.2445} & \cellcolor{gray!10}{12.8206}\\
\addlinespace
Guazuma ulmifolia Lam. & 1 & 0.0227 & 1 & 1.25 & 0.2445 & 0.0284 & 0.2136 & 50 & 1.4706 & 0.0024 & 0.2445 & 12.8202\\
\cellcolor{gray!10}{Agonandra brasiliensis Miers ex Benth. \& Hook.f.} & \cellcolor{gray!10}{1} & \cellcolor{gray!10}{0.0225} & \cellcolor{gray!10}{1} & \cellcolor{gray!10}{1.25} & \cellcolor{gray!10}{0.2445} & \cellcolor{gray!10}{0.0282} & \cellcolor{gray!10}{0.2120} & \cellcolor{gray!10}{50} & \cellcolor{gray!10}{1.4706} & \cellcolor{gray!10}{0.0024} & \cellcolor{gray!10}{0.2445} & \cellcolor{gray!10}{12.8202}\\
Handroanthus chrysotrichus (Mart. ex DC.) Mattos & 1 & 0.0201 & 1 & 1.25 & 0.2445 & 0.0251 & 0.1888 & 50 & 1.4706 & 0.0024 & 0.2445 & 12.8194\\
\cellcolor{gray!10}{Peltogyne confertiflora (Mart. ex Hayne) Benth.} & \cellcolor{gray!10}{1} & \cellcolor{gray!10}{0.0170} & \cellcolor{gray!10}{1} & \cellcolor{gray!10}{1.25} & \cellcolor{gray!10}{0.2445} & \cellcolor{gray!10}{0.0212} & \cellcolor{gray!10}{0.1599} & \cellcolor{gray!10}{50} & \cellcolor{gray!10}{1.4706} & \cellcolor{gray!10}{0.0024} & \cellcolor{gray!10}{0.2445} & \cellcolor{gray!10}{12.8184}\\
Lafoensia pacari A.St.-Hil. & 1 & 0.0121 & 1 & 1.25 & 0.2445 & 0.0151 & 0.1139 & 50 & 1.4706 & 0.0024 & 0.2445 & 12.8169\\
\addlinespace
\cellcolor{gray!10}{Machaerium acutifolium Vogel} & \cellcolor{gray!10}{1} & \cellcolor{gray!10}{0.0117} & \cellcolor{gray!10}{1} & \cellcolor{gray!10}{1.25} & \cellcolor{gray!10}{0.2445} & \cellcolor{gray!10}{0.0147} & \cellcolor{gray!10}{0.1105} & \cellcolor{gray!10}{50} & \cellcolor{gray!10}{1.4706} & \cellcolor{gray!10}{0.0024} & \cellcolor{gray!10}{0.2445} & \cellcolor{gray!10}{12.8168}\\
Pouteria torta (Mart.) Radlk. & 1 & 0.0105 & 1 & 1.25 & 0.2445 & 0.0132 & 0.0993 & 50 & 1.4706 & 0.0024 & 0.2445 & 12.8164\\
\cellcolor{gray!10}{Morfoespécie 3} & \cellcolor{gray!10}{1} & \cellcolor{gray!10}{0.0085} & \cellcolor{gray!10}{1} & \cellcolor{gray!10}{1.25} & \cellcolor{gray!10}{0.2445} & \cellcolor{gray!10}{0.0106} & \cellcolor{gray!10}{0.0801} & \cellcolor{gray!10}{50} & \cellcolor{gray!10}{1.4706} & \cellcolor{gray!10}{0.0024} & \cellcolor{gray!10}{0.2445} & \cellcolor{gray!10}{12.8158}\\
\bottomrule
\end{tabular}}
\end{table}
\end{landscape}

Na aplicação prática, a utilização do VIA, que combina parâmetros horizontais e verticais com indicadores de regeneração, fornece uma visão holística da importância das espécies. Isso é fundamental para identificar não apenas as espécies dominantes atuais, mas também aquelas com maior potencial de contribuir para a sustentabilidade futura do ecossistema.

\chapter{Manejo de Florestas Plantadas}\label{manejo-de-florestas-plantadas}

\section{Altura Dominante}\label{altura-dominante}

A altura dominante, que representa a média das alturas das árvores mais altas e/ou mais grossas de um determinado número de árvores por hectare, é menos influenciada por variações de densidade e competições locais entre árvores menores, tornando-a uma medida confiável da qualidade do sítio. Com ela, é possível estimar o potencial de crescimento da floresta, modelar curvas de crescimento e projetar a produção futura, auxiliando no planejamento sustentável e na tomada de decisões estratégicas no manejo florestal.

Dados exemplo:
\href{data/DesafioCalcHdom.csv}{Baixar dados}

\begin{table}
\centering
\begin{tabular}{r|r|r|r|r|l}
\hline
A.Parc. & Parc & n & DAP & Ht & Hdom\\
\hline
720 & 711 & 1 & 19.74 & 14.65 & NA\\
\hline
720 & 711 & 2 & NA & NA & NA\\
\hline
720 & 711 & 3 & 19.19 & 14.44 & NA\\
\hline
720 & 711 & 4 & 17.83 & 13.89 & NA\\
\hline
720 & 711 & 5 & 18.46 & 14.15 & NA\\
\hline
720 & 711 & 6 & 20.05 & 14.78 & NA\\
\hline
\end{tabular}
\end{table}

\subsection{Calculando altura dominante por parcela (Assman)}\label{calculando-altura-dominante-por-parcela-assman}

\begin{Shaded}
\begin{Highlighting}[]
\NormalTok{BD }\OtherTok{\textless{}{-}} \FunctionTok{read.csv2}\NormalTok{(}\StringTok{"data/DesafioCalcHdom.csv"}\NormalTok{)}
\FunctionTok{colnames}\NormalTok{(BD)[}\DecValTok{1}\NormalTok{] }\OtherTok{=} \FunctionTok{c}\NormalTok{(}\StringTok{"areaParc"}\NormalTok{)}
\NormalTok{BD }\OtherTok{\textless{}{-}}\NormalTok{ BD[, }\DecValTok{1}\SpecialCharTok{:}\DecValTok{5}\NormalTok{]}
\NormalTok{BD }\OtherTok{\textless{}{-}}\NormalTok{ BD[}\FunctionTok{order}\NormalTok{(BD}\SpecialCharTok{$}\NormalTok{Parc, BD}\SpecialCharTok{$}\NormalTok{Ht, }\AttributeTok{decreasing =} \ConstantTok{TRUE}\NormalTok{),]}
\FunctionTok{row.names}\NormalTok{(BD) }\OtherTok{\textless{}{-}} \ConstantTok{NULL}

\NormalTok{Parc }\OtherTok{\textless{}{-}} \FunctionTok{unique}\NormalTok{(BD}\SpecialCharTok{$}\NormalTok{Parc)}
\NormalTok{Hdom }\OtherTok{\textless{}{-}} \FunctionTok{unique}\NormalTok{(BD}\SpecialCharTok{$}\NormalTok{Parc)}

\NormalTok{j }\OtherTok{\textless{}{-}} \DecValTok{1}
\ControlFlowTok{for}\NormalTok{ (i }\ControlFlowTok{in}\NormalTok{ Parc) \{}
\NormalTok{  a }\OtherTok{\textless{}{-}} \FunctionTok{round}\NormalTok{(}\FunctionTok{mean}\NormalTok{(BD[}\FunctionTok{which}\NormalTok{(BD}\SpecialCharTok{$}\NormalTok{Parc}\SpecialCharTok{==}\NormalTok{i),}\DecValTok{1}\NormalTok{])}\SpecialCharTok{*}\DecValTok{100}\SpecialCharTok{/}\DecValTok{10000}\NormalTok{, }\DecValTok{0}\NormalTok{)}
\NormalTok{  H }\OtherTok{\textless{}{-}}\NormalTok{ BD[}\FunctionTok{which}\NormalTok{(BD}\SpecialCharTok{$}\NormalTok{Parc}\SpecialCharTok{==}\NormalTok{i),}\DecValTok{5}\NormalTok{]}
\NormalTok{  Hdom[j] }\OtherTok{\textless{}{-}} \FunctionTok{round}\NormalTok{(}\FunctionTok{mean}\NormalTok{(H[}\DecValTok{1}\SpecialCharTok{:}\NormalTok{a]), }\DecValTok{2}\NormalTok{)}
\NormalTok{  j }\OtherTok{\textless{}{-}}\NormalTok{ j}\SpecialCharTok{+}\DecValTok{1}
\NormalTok{\}}

\NormalTok{BD2 }\OtherTok{\textless{}{-}} \FunctionTok{as.data.frame}\NormalTok{(}\FunctionTok{cbind}\NormalTok{(Parc, Hdom))}
\NormalTok{BD }\OtherTok{\textless{}{-}} \FunctionTok{merge}\NormalTok{(BD, BD2)}

\FunctionTok{rm}\NormalTok{(a, H, Hdom, i, j, Parc)}
\end{Highlighting}
\end{Shaded}

Resultado:

\begin{table}
\centering
\begin{tabular}{r|r}
\hline
Parc & Hdom\\
\hline
711 & 14.58\\
\hline
623 & 14.02\\
\hline
622 & 13.54\\
\hline
621 & 14.62\\
\hline
534 & 15.00\\
\hline
533 & 14.62\\
\hline
532 & 14.76\\
\hline
531 & 14.95\\
\hline
453 & 14.51\\
\hline
452 & 14.53\\
\hline
451 & 14.52\\
\hline
344 & 15.06\\
\hline
343 & 14.48\\
\hline
342 & 14.78\\
\hline
341 & 14.96\\
\hline
262 & 13.28\\
\hline
261 & 14.33\\
\hline
173 & 14.10\\
\hline
172 & 14.22\\
\hline
171 & 14.80\\
\hline
\end{tabular}
\end{table}

\chapter*{Referências}\label{referuxeancias}
\addcontentsline{toc}{chapter}{Referências}

  \bibliography{book.bib}

\end{document}
